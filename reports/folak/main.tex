\documentclass{mod-comjnl}
\usepackage[utf8]{inputenc}
\usepackage[T1]{fontenc}
\usepackage[english]{babel}
\usepackage{csquotes}
\usepackage{hyperref}
\usepackage{url}
\usepackage{siunitx}
\usepackage[gen]{eurosym}
\usepackage[acronym]{glossaries}
\usepackage{blindtext}
\usepackage{booktabs}
\usepackage{pgf}
\usepackage[tableposition=top,font=small]{caption}
\usepackage{subcaption}
\usepackage{float}
\usepackage{stfloats}
\usepackage[top=2cm, left=1.65cm, right=1.65cm, bottom=0.75cm]{geometry}
\usepackage[flushmargin]{footmisc}
\usepackage{xcolor}


\hypersetup{hidelinks}

\makeatletter
\setlength{\@fptop}{0pt}
\makeatother

%% These two lines are needed to get the correct paper size
%% in TeX Live 2016
\let\pdfpageheight\paperheight
\let\pdfpagewidth\paperwidth

\glsdisablehyper

\newacronym{he}{H\&E}{hematoxylin and eosin}
\newacronym{wsi}{WSI}{whole slide imaging}
\newacronym{cnn}{CNN}{convolutional neural network}
\newacronym{qwk}{QWK}{quadratic weighted Cohen’s kappa}
\newacronym{pca}{PCa}{prostate cancer}
\newacronym{panda}{PANDA}{Prostate cANcer graDe Assessment}
\newacronym{isup}{ISUP}{International Society of Urological Pathology}


\sisetup{
  group-four-digits = true,
  group-separator = {,}
}

% custom vancouver styling
\usepackage[style=numeric,
    backend=biber,
    sorting=none,
    url=false,
    isbn=false,
    terseinits=true,
    giveninits=true,
    minnames=6,
    maxnames=6]{biblatex}

% Remove unwanted punctuations
\renewcommand*{\revsdnamepunct}{}
\renewcommand*{\finentrypunct}{}
\renewcommand*{\bibpagespunct}{}

% Dot instead av brackets in references
\DeclareFieldFormat{labelnumberwidth}{\mkbibbold{#1\adddot}}

% Lastname followed by initials format
\DeclareNameAlias{sortname}{family-given}
\DeclareNameAlias{default}{family-given}
\renewcommand*{\revsdnamepunct}{}

\DeclareSourcemap{%
    \maps[datatype=bibtex]{
        % Journal abbreviations
        \map[overwrite]{
            \step[fieldsource=shortjournal]
            \step[fieldset=journaltitle,origfieldval]
        }
    }
}

% remove in
\renewbibmacro{in:}{}
% remove pp
\DeclareFieldFormat*{pages}{#1}
% reformat doi
\DeclareFieldFormat*{doi}{\url{https://doi.org/#1}}
%remove quotation marks around title
\DeclareFieldFormat*{title}{#1}


\DeclareFieldFormat{journaltitle}{\mkbibemph{#1}\isdot}

% Provide three letter month names
\newcommand*{\shortmonth}[1]{
    \ifthenelse{\NOT\equal{#1}{}}{
        \ifcase#1\relax
        \or Jan
        \or Feb
        \or Mar
        \or Apr
        \or May
        \or Jun
        \or Jul
        \or Aug
        \or Sep
        \or Oct
        \or Nov
        \or Dec
        \fi
    }
}

\DeclareFieldFormat*{number}{\mkbibparens{#1}}

\DeclareFieldFormat*{date}{\thefield{year}}

% Code adapted from biblatex-nejm package

\renewbibmacro*{volume+number+eid}{
    \printfield{volume}%
    \setunit{}%
    \printfield{number}%
    \addcolon%
    \printfield{eid}%
}

\renewbibmacro*{issue+date}{
    \usebibmacro{date}
}

\renewbibmacro*{journal+issuetitle}{
    \usebibmacro{journal}%
    \iffieldundef{series}%
    \adddot%
    {}
    {\newunit%
        \printfield{series}}%
    \setunit{\addspace}%
    \usebibmacro{issue+date}%
    \setunit{\addsemicolon}%
    \usebibmacro{volume+number+eid}%
    \usebibmacro{issue}%
    \newunit}

% compress page numbers. E.g. XYZ-XAB -> XYZ-AB
\DeclareFieldFormat{postnote}{\mkcomprange[{\mkpageprefix[pagination]}]{#1}}
\DeclareFieldFormat{pages}{\mkcomprange{#1}}

% Compress ranges where lower limit > 100
\setcounter{mincomprange}{100}

% Don't compress beyond the fourth digit
\setcounter{maxcomprange}{1000}

% Display compressed upper limit with at least two digits,
% unless leading digit is zero
\setcounter{mincompwidth}{10}

\addbibresource{main.bib}
 
\begin{document}

\title[Efficient CNN Training for ISUP Classification of Prostate Biopsies]{Efficient Convolutional Neural Network Training for ISUP Classification of Prostate Biopsies}
\author{Kelvin Szolnoky}
\shortauthors{K. Szolnoky}
\affiliation{Department of Medical Epidemiology and Biostatistics,\\ Karolinska Institutet,\\ Stockholm, Sweden}
\email{kelvin.szolnoky@ki.se}
\supervisor{Martin Eklund}

\begin{abstract}
  \Acrfull{pca} is one of the most frequently occurring cancers in the world and in combination with it being a rather expensive cancer to diagnose and treat,  \acrshort{pca} becomes an extensive economic burden on society. This burden may be lifted by introducing automation systems where possible as a diagnostic tool. This can be done by scanning in the biopsies using \acrlong{wsi} and using a hot topic in science, machine learning. In this article, we show that classifying, and thus diagnosing, prostate biopsies is possible with a simple model. The model achieves a state-of-the-art like performance with a \acrfull{qwk} of 0.909 on an internal test set. The result is validated on an external test set and scoring a \acrshort{qwk} of 0.882. Our results show that it is possible to create high performing ISUP grader with a simple model and thus minimising the resources required for training and inference.
\end{abstract}

\maketitle

\section{Introduction}
In men, \acrfull{pca} is, compared to other types of cancers, the most frequently diagnosed in 112 countries and the leading cause of death in 48 countries \cite{sung_global_2021}. In a 2020 article by Shuang Hao et. al., the estimated economic burden of \acrshort{pca} for Sweden was estimated to be \SI{280781820}[\euro]{} \cite{hao_economic_2020}. This sum includes everything from direct healthcare costs to estimated cost due to productivity loss. Thus, all advancements and improvements within diagnostics and treatment have a large opportunity to reduce the societal burden.

Two areas that have a large development potential are diagnostics and prognostics. A pathologist's assessment through a microscope on \acrfull{he} stained biopsy is currently the most important marker for both \acrshort{pca} diagnostics and prognostics \cite{epstein_update_2010}. However, this assessment process has remained rather unchanged for the past century \cite{boyce_update_2017}. Since the approval of \acrfull{wsi} by the Food and Drug Administration (FDA) \cite{boyce_update_2017} has the typical prostate viewing by microscope been switched out for the digital \acrshort{wsi}. It is safe to say that the digitalisation of pathology has been slow when comparing to medical disciplines such as radiology \cite{zippel_rise_2021}.

However, the standardised approach of pathologists grading prostate biopsies has several flaws. Firstly, it is very costly as the average cost of a pathologist's assessment on a prostate biopsy is \SI{516}[\euro]{} \cite{hao_economic_2020}, making it one of the more expensive procedures in the treatment pathway. Secondly, it suffers from significant inter- and intraobserver variability \cite{egevad_standardization_2013, allsbrook_interobserver_general_2001}, however, specialized uropathologists show higher concordance rates \cite{allsbrook_interobserver_uro_2001}.

The ISUP grading was introduced in 2016 to combat the rater variability \cite{egevad_international_2016}. It is done by grouping Gleason scores into five groups, ISUP grades 1-5. 

In later years, a hot topic within several fields of science, including medicine, is the application of machine learning. Machine learning within medicine has shown a lot of promise, especially \acrfullpl{cnn} within the field of image analysis. An article within the field of dermatology has demonstrated that a \acrshort{cnn} achieves higher performance compared to the majority of dermatologists tested \cite{haenssle_man_2018}. Furthermore, results show that not only can \acrshortpl{cnn} perform better than pathologists grading prostate biopsies, they also show that in symbiosis pathologists and \acrshortpl{cnn} systems achieve higher performance than solitarily \cite{bulten_artificial_2021}.

Despite the performance of these systems, their application into clinical practice has been slow and remains a question. This is due to the fact that the small datasets trained and validated on miss wide variance of clinical samples \cite{campanella_clinical-grade_2019}. Another issue with \acrshortpl{cnn} is the sheer amount of resources required to train them. The computer chip shortage has made it harder to access the resources needed to train and use \acrshortpl{cnn}. Lightweight models that require less resources for training and inference but still achieve cutting edge results are needed to continue the development and application of \acrshortpl{cnn}. The \textit{EfficientNet} family are a great example of this, for their time they were 8x smaller and 6x faster than the best existing \acrshortpl{cnn} and yet today they are one of the top-performing networks on ImageNet.

In 2020, an online challenge (\acrfull{panda} Challenge\footnote{\url{https://www.kaggle.com/c/prostate-cancer-grade-assessment/overview}}) to create models that classify \acrshort{pca} in biopsies was hosted by a joint collaboration of Radboud University Medical Center and Karolinska Institute. With roughly 11 000 \acrshortpl{wsi}, this made the \acrshort{panda} challenge and its dataset the largest publicly available \acrshort{wsi} collection.

The aim of the study was to train a \textit{simple} but yet state-of-the-art \acrshort{isup} grading model on an open dataset of prostate biopsies (the \acrshort{panda} dataset) while using as little resources as possible. Furthermore, we validate these results across both an internal and external dataset.

\section{Materials and Methods}
All code used in this study is publicly available online at \url{https://github.com/kelszo/isup-grader} under the \textit{GNU Affero General Public License v3.0}.

\subsection{Dataset}
The training dataset was retrieved from the \acrfull{panda} Challenge hosted on Kaggle\footnote{\url{https://www.kaggle.com/c/prostate-cancer-grade-assessment}} under the \textit{Attribution-NonCommercial-ShareAlike 4.0 International} license. The study's data was approved by the institutional review board of Radboud University Medical Center (IRB \textit{2016–2275}), Stockholm regional ethics committee (permits \textit{2012/572-31/1}, \textit{2012/438-31/3}, and \textit{2018/845-32}). Informed consent was waived due to the usage of de-identified prostate specimens in a retrospective setting.

The dataset consisted of 10 616 \acrshort{wsi} images in tiff format used for training and an additional 938 slides used for testing. The training dataset was split into five folds for cross-validation. The testing dataset was used as an internal validation set.

330 WSI of prostate biopsies consisting of both benign and cancerous tissue of varying ISUP grades was fetched from \textit{Karolinska University Hospital} to use as an external validation set.

\subsection{Data pre-processing}
\begin{figure}[!t]
  \centering
  \includegraphics[width=0.4\textwidth]{figures/tiling.png}
  \caption{The process from WSI to input image for the network. Starts with finding the axis for the biopsy and then optimally extracting non-overlapping patches to include as much data as possible from the biopsy with a minimal amount of background data in each patch. Finishes with glueing together 36 randomly selected patches into a 1:1 resolution image.}
  \label{fig:tiling}
\end{figure}

The pre-processing done on the \acrshortpl{wsi} was to split each image up into smaller sub-images (patches) to remove white space and reduce the total resolution of each input image. This was done by using an optimised algorithm finding the axis of each biopsy and then extracting as many non-overlapping patches as possible from the biopsy as seen in figure \ref{fig:tiling}. Each patch was extracted at level 1 (downsampling of 4) with a resolution of 256x256.

To exclude noisy labels, an open pre-made dataset that already eliminated noisy labels was used\footnote{\url{https://github.com/kentaroy47/Kaggle-PANDA-1st-place-solution}}.

\subsection{Model}
The model consisted of a single EfficientNetB0 model \cite{tan_efficientnet_2020} pre-trained on ImageNet. All parameters remained tunable for a total of 5.3 million tunable parameters. The model's input was a 1536x1536 resolution image with 3 channels; the image was composed of 36 random patches selected from the pre-processing step, if the biopsy comprised of less than 36 patches white patches were appended to reach 36 patches. The patches were glued together in random order. Each time a biopsy was fetched, a new 36 random patches were selected and glued together to form a stitched image.

Different types of augmentations were made on both a patch and stitched image level. On patch level, flipping in both horisontal and vertical axis, blurring, jittering the colours, and cutout were done. On the stitched image level, flipping in both horisontal and vertical axis and normalising the image by the ImageNet values. These augmentations were done to decrease the possibility of overfitting. Additionally, AdamW was chosen as the optimizer to help combat overfitting. Larger batch sizes of 8 were selected to utilise the complete effect of batch normalization in the network.

Several additional methods were used to increase training speed. Most notable include cyclic learning rate scheduler \cite{smith_cyclical_2017} and mixed-precision training \cite{micikevicius_mixed_2018}. Mixed-precision training was also used to reduce the required amount of resources needed. Worth noting is that ensembling was not done to speed up training and inference time.

Test time augmentations were done to try to mimic the effects of ensembles during testing without needing to train complete ensembles. It was also done to include more patches in testing, due to the nature of the tiling algorithm it can select more than 36 patches. Thus to include all these tiles in the final prediction, predictions were made several times with different tiles and augmentations and the final prediction was chosen through majority voting.

\subsection{Statistical Analysis}
The model's ISUP-grading performance will be assessed using the \acrfull{qwk} and accuracy. \acrshort{qwk} is a metric that measures the agreement between two raters (inter and intra) of categorical items. If the two series are at a random agreement the \acrshort{qwk} will be 0, on the other hand, if the two series are in complete agreement the \acrshort{qwk} is 1. \acrshort{qwk} can also be less than 0 if the two series are at less agreement than expected by pure randomness. Accuracy is simply the percentage of outcomes that the network predicts the same as ground truth.

Choice of treatment is often based on grouped ISUP grades; the groups being: benign, ISUP 1, ISUP 2-3, and ISUP4-5. The performance will thus also be graded in how well it performs (both \acrshort{qwk} and accuracy) on these grouped ISUP grades.

\section{Results}
\begin{figure*}[t]
  \hspace{-2cm}
  \centering
  \begin{subfigure}[b]{.5\linewidth}
    \centering
    \begin{subfigure}[b]{.5\linewidth}
      \centering
      \resizebox{1.5\textwidth}{!}{%% Creator: Matplotlib, PGF backend
%%
%% To include the figure in your LaTeX document, write
%%   \input{<filename>.pgf}
%%
%% Make sure the required packages are loaded in your preamble
%%   \usepackage{pgf}
%%
%% Figures using additional raster images can only be included by \input if
%% they are in the same directory as the main LaTeX file. For loading figures
%% from other directories you can use the `import` package
%%   \usepackage{import}
%%
%% and then include the figures with
%%   \import{<path to file>}{<filename>.pgf}
%%
%% Matplotlib used the following preamble
%%
\begingroup%
\makeatletter%
\begin{pgfpicture}%
\pgfpathrectangle{\pgfpointorigin}{\pgfqpoint{6.000000in}{4.000000in}}%
\pgfusepath{use as bounding box, clip}%
\begin{pgfscope}%
\pgfsetbuttcap%
\pgfsetmiterjoin%
\pgfsetlinewidth{0.000000pt}%
\definecolor{currentstroke}{rgb}{1.000000,1.000000,1.000000}%
\pgfsetstrokecolor{currentstroke}%
\pgfsetstrokeopacity{0.000000}%
\pgfsetdash{}{0pt}%
\pgfpathmoveto{\pgfqpoint{0.000000in}{0.000000in}}%
\pgfpathlineto{\pgfqpoint{6.000000in}{0.000000in}}%
\pgfpathlineto{\pgfqpoint{6.000000in}{4.000000in}}%
\pgfpathlineto{\pgfqpoint{0.000000in}{4.000000in}}%
\pgfpathclose%
\pgfusepath{}%
\end{pgfscope}%
\begin{pgfscope}%
\pgfsetbuttcap%
\pgfsetmiterjoin%
\definecolor{currentfill}{rgb}{1.000000,1.000000,1.000000}%
\pgfsetfillcolor{currentfill}%
\pgfsetlinewidth{0.000000pt}%
\definecolor{currentstroke}{rgb}{0.000000,0.000000,0.000000}%
\pgfsetstrokecolor{currentstroke}%
\pgfsetstrokeopacity{0.000000}%
\pgfsetdash{}{0pt}%
\pgfpathmoveto{\pgfqpoint{1.565000in}{0.500000in}}%
\pgfpathlineto{\pgfqpoint{4.585000in}{0.500000in}}%
\pgfpathlineto{\pgfqpoint{4.585000in}{3.520000in}}%
\pgfpathlineto{\pgfqpoint{1.565000in}{3.520000in}}%
\pgfpathclose%
\pgfusepath{fill}%
\end{pgfscope}%
\begin{pgfscope}%
\pgfpathrectangle{\pgfqpoint{1.565000in}{0.500000in}}{\pgfqpoint{3.020000in}{3.020000in}}%
\pgfusepath{clip}%
\pgfsetbuttcap%
\pgfsetroundjoin%
\definecolor{currentfill}{rgb}{0.472656,0.031250,0.289062}%
\pgfsetfillcolor{currentfill}%
\pgfsetlinewidth{0.000000pt}%
\definecolor{currentstroke}{rgb}{1.000000,1.000000,1.000000}%
\pgfsetstrokecolor{currentstroke}%
\pgfsetdash{}{0pt}%
\pgfpathmoveto{\pgfqpoint{1.565000in}{3.520000in}}%
\pgfpathlineto{\pgfqpoint{2.068333in}{3.520000in}}%
\pgfpathlineto{\pgfqpoint{2.068333in}{3.016667in}}%
\pgfpathlineto{\pgfqpoint{1.565000in}{3.016667in}}%
\pgfpathlineto{\pgfqpoint{1.565000in}{3.520000in}}%
\pgfusepath{fill}%
\end{pgfscope}%
\begin{pgfscope}%
\pgfpathrectangle{\pgfqpoint{1.565000in}{0.500000in}}{\pgfqpoint{3.020000in}{3.020000in}}%
\pgfusepath{clip}%
\pgfsetbuttcap%
\pgfsetroundjoin%
\definecolor{currentfill}{rgb}{0.790824,0.617724,0.718827}%
\pgfsetfillcolor{currentfill}%
\pgfsetlinewidth{0.000000pt}%
\definecolor{currentstroke}{rgb}{1.000000,1.000000,1.000000}%
\pgfsetstrokecolor{currentstroke}%
\pgfsetdash{}{0pt}%
\pgfpathmoveto{\pgfqpoint{2.068333in}{3.520000in}}%
\pgfpathlineto{\pgfqpoint{2.571667in}{3.520000in}}%
\pgfpathlineto{\pgfqpoint{2.571667in}{3.016667in}}%
\pgfpathlineto{\pgfqpoint{2.068333in}{3.016667in}}%
\pgfpathlineto{\pgfqpoint{2.068333in}{3.520000in}}%
\pgfusepath{fill}%
\end{pgfscope}%
\begin{pgfscope}%
\pgfpathrectangle{\pgfqpoint{1.565000in}{0.500000in}}{\pgfqpoint{3.020000in}{3.020000in}}%
\pgfusepath{clip}%
\pgfsetbuttcap%
\pgfsetroundjoin%
\definecolor{currentfill}{rgb}{0.968724,0.945644,0.959125}%
\pgfsetfillcolor{currentfill}%
\pgfsetlinewidth{0.000000pt}%
\definecolor{currentstroke}{rgb}{1.000000,1.000000,1.000000}%
\pgfsetstrokecolor{currentstroke}%
\pgfsetdash{}{0pt}%
\pgfpathmoveto{\pgfqpoint{2.571667in}{3.520000in}}%
\pgfpathlineto{\pgfqpoint{3.075000in}{3.520000in}}%
\pgfpathlineto{\pgfqpoint{3.075000in}{3.016667in}}%
\pgfpathlineto{\pgfqpoint{2.571667in}{3.016667in}}%
\pgfpathlineto{\pgfqpoint{2.571667in}{3.520000in}}%
\pgfusepath{fill}%
\end{pgfscope}%
\begin{pgfscope}%
\pgfpathrectangle{\pgfqpoint{1.565000in}{0.500000in}}{\pgfqpoint{3.020000in}{3.020000in}}%
\pgfusepath{clip}%
\pgfsetbuttcap%
\pgfsetroundjoin%
\definecolor{currentfill}{rgb}{0.996094,0.996094,0.996094}%
\pgfsetfillcolor{currentfill}%
\pgfsetlinewidth{0.000000pt}%
\definecolor{currentstroke}{rgb}{1.000000,1.000000,1.000000}%
\pgfsetstrokecolor{currentstroke}%
\pgfsetdash{}{0pt}%
\pgfpathmoveto{\pgfqpoint{3.075000in}{3.520000in}}%
\pgfpathlineto{\pgfqpoint{3.578333in}{3.520000in}}%
\pgfpathlineto{\pgfqpoint{3.578333in}{3.016667in}}%
\pgfpathlineto{\pgfqpoint{3.075000in}{3.016667in}}%
\pgfpathlineto{\pgfqpoint{3.075000in}{3.520000in}}%
\pgfusepath{fill}%
\end{pgfscope}%
\begin{pgfscope}%
\pgfpathrectangle{\pgfqpoint{1.565000in}{0.500000in}}{\pgfqpoint{3.020000in}{3.020000in}}%
\pgfusepath{clip}%
\pgfsetbuttcap%
\pgfsetroundjoin%
\definecolor{currentfill}{rgb}{0.996094,0.996094,0.996094}%
\pgfsetfillcolor{currentfill}%
\pgfsetlinewidth{0.000000pt}%
\definecolor{currentstroke}{rgb}{1.000000,1.000000,1.000000}%
\pgfsetstrokecolor{currentstroke}%
\pgfsetdash{}{0pt}%
\pgfpathmoveto{\pgfqpoint{3.578333in}{3.520000in}}%
\pgfpathlineto{\pgfqpoint{4.081667in}{3.520000in}}%
\pgfpathlineto{\pgfqpoint{4.081667in}{3.016667in}}%
\pgfpathlineto{\pgfqpoint{3.578333in}{3.016667in}}%
\pgfpathlineto{\pgfqpoint{3.578333in}{3.520000in}}%
\pgfusepath{fill}%
\end{pgfscope}%
\begin{pgfscope}%
\pgfpathrectangle{\pgfqpoint{1.565000in}{0.500000in}}{\pgfqpoint{3.020000in}{3.020000in}}%
\pgfusepath{clip}%
\pgfsetbuttcap%
\pgfsetroundjoin%
\definecolor{currentfill}{rgb}{0.996094,0.996094,0.996094}%
\pgfsetfillcolor{currentfill}%
\pgfsetlinewidth{0.000000pt}%
\definecolor{currentstroke}{rgb}{1.000000,1.000000,1.000000}%
\pgfsetstrokecolor{currentstroke}%
\pgfsetdash{}{0pt}%
\pgfpathmoveto{\pgfqpoint{4.081667in}{3.520000in}}%
\pgfpathlineto{\pgfqpoint{4.585000in}{3.520000in}}%
\pgfpathlineto{\pgfqpoint{4.585000in}{3.016667in}}%
\pgfpathlineto{\pgfqpoint{4.081667in}{3.016667in}}%
\pgfpathlineto{\pgfqpoint{4.081667in}{3.520000in}}%
\pgfusepath{fill}%
\end{pgfscope}%
\begin{pgfscope}%
\pgfpathrectangle{\pgfqpoint{1.565000in}{0.500000in}}{\pgfqpoint{3.020000in}{3.020000in}}%
\pgfusepath{clip}%
\pgfsetbuttcap%
\pgfsetroundjoin%
\definecolor{currentfill}{rgb}{0.927671,0.869970,0.903671}%
\pgfsetfillcolor{currentfill}%
\pgfsetlinewidth{0.000000pt}%
\definecolor{currentstroke}{rgb}{1.000000,1.000000,1.000000}%
\pgfsetstrokecolor{currentstroke}%
\pgfsetdash{}{0pt}%
\pgfpathmoveto{\pgfqpoint{1.565000in}{3.016667in}}%
\pgfpathlineto{\pgfqpoint{2.068333in}{3.016667in}}%
\pgfpathlineto{\pgfqpoint{2.068333in}{2.513333in}}%
\pgfpathlineto{\pgfqpoint{1.565000in}{2.513333in}}%
\pgfpathlineto{\pgfqpoint{1.565000in}{3.016667in}}%
\pgfusepath{fill}%
\end{pgfscope}%
\begin{pgfscope}%
\pgfpathrectangle{\pgfqpoint{1.565000in}{0.500000in}}{\pgfqpoint{3.020000in}{3.020000in}}%
\pgfusepath{clip}%
\pgfsetbuttcap%
\pgfsetroundjoin%
\definecolor{currentfill}{rgb}{0.472656,0.031250,0.289062}%
\pgfsetfillcolor{currentfill}%
\pgfsetlinewidth{0.000000pt}%
\definecolor{currentstroke}{rgb}{1.000000,1.000000,1.000000}%
\pgfsetstrokecolor{currentstroke}%
\pgfsetdash{}{0pt}%
\pgfpathmoveto{\pgfqpoint{2.068333in}{3.016667in}}%
\pgfpathlineto{\pgfqpoint{2.571667in}{3.016667in}}%
\pgfpathlineto{\pgfqpoint{2.571667in}{2.513333in}}%
\pgfpathlineto{\pgfqpoint{2.068333in}{2.513333in}}%
\pgfpathlineto{\pgfqpoint{2.068333in}{3.016667in}}%
\pgfusepath{fill}%
\end{pgfscope}%
\begin{pgfscope}%
\pgfpathrectangle{\pgfqpoint{1.565000in}{0.500000in}}{\pgfqpoint{3.020000in}{3.020000in}}%
\pgfusepath{clip}%
\pgfsetbuttcap%
\pgfsetroundjoin%
\definecolor{currentfill}{rgb}{0.681347,0.415926,0.570951}%
\pgfsetfillcolor{currentfill}%
\pgfsetlinewidth{0.000000pt}%
\definecolor{currentstroke}{rgb}{1.000000,1.000000,1.000000}%
\pgfsetstrokecolor{currentstroke}%
\pgfsetdash{}{0pt}%
\pgfpathmoveto{\pgfqpoint{2.571667in}{3.016667in}}%
\pgfpathlineto{\pgfqpoint{3.075000in}{3.016667in}}%
\pgfpathlineto{\pgfqpoint{3.075000in}{2.513333in}}%
\pgfpathlineto{\pgfqpoint{2.571667in}{2.513333in}}%
\pgfpathlineto{\pgfqpoint{2.571667in}{3.016667in}}%
\pgfusepath{fill}%
\end{pgfscope}%
\begin{pgfscope}%
\pgfpathrectangle{\pgfqpoint{1.565000in}{0.500000in}}{\pgfqpoint{3.020000in}{3.020000in}}%
\pgfusepath{clip}%
\pgfsetbuttcap%
\pgfsetroundjoin%
\definecolor{currentfill}{rgb}{0.968724,0.945644,0.959125}%
\pgfsetfillcolor{currentfill}%
\pgfsetlinewidth{0.000000pt}%
\definecolor{currentstroke}{rgb}{1.000000,1.000000,1.000000}%
\pgfsetstrokecolor{currentstroke}%
\pgfsetdash{}{0pt}%
\pgfpathmoveto{\pgfqpoint{3.075000in}{3.016667in}}%
\pgfpathlineto{\pgfqpoint{3.578333in}{3.016667in}}%
\pgfpathlineto{\pgfqpoint{3.578333in}{2.513333in}}%
\pgfpathlineto{\pgfqpoint{3.075000in}{2.513333in}}%
\pgfpathlineto{\pgfqpoint{3.075000in}{3.016667in}}%
\pgfusepath{fill}%
\end{pgfscope}%
\begin{pgfscope}%
\pgfpathrectangle{\pgfqpoint{1.565000in}{0.500000in}}{\pgfqpoint{3.020000in}{3.020000in}}%
\pgfusepath{clip}%
\pgfsetbuttcap%
\pgfsetroundjoin%
\definecolor{currentfill}{rgb}{0.982409,0.970869,0.977609}%
\pgfsetfillcolor{currentfill}%
\pgfsetlinewidth{0.000000pt}%
\definecolor{currentstroke}{rgb}{1.000000,1.000000,1.000000}%
\pgfsetstrokecolor{currentstroke}%
\pgfsetdash{}{0pt}%
\pgfpathmoveto{\pgfqpoint{3.578333in}{3.016667in}}%
\pgfpathlineto{\pgfqpoint{4.081667in}{3.016667in}}%
\pgfpathlineto{\pgfqpoint{4.081667in}{2.513333in}}%
\pgfpathlineto{\pgfqpoint{3.578333in}{2.513333in}}%
\pgfpathlineto{\pgfqpoint{3.578333in}{3.016667in}}%
\pgfusepath{fill}%
\end{pgfscope}%
\begin{pgfscope}%
\pgfpathrectangle{\pgfqpoint{1.565000in}{0.500000in}}{\pgfqpoint{3.020000in}{3.020000in}}%
\pgfusepath{clip}%
\pgfsetbuttcap%
\pgfsetroundjoin%
\definecolor{currentfill}{rgb}{0.996094,0.996094,0.996094}%
\pgfsetfillcolor{currentfill}%
\pgfsetlinewidth{0.000000pt}%
\definecolor{currentstroke}{rgb}{1.000000,1.000000,1.000000}%
\pgfsetstrokecolor{currentstroke}%
\pgfsetdash{}{0pt}%
\pgfpathmoveto{\pgfqpoint{4.081667in}{3.016667in}}%
\pgfpathlineto{\pgfqpoint{4.585000in}{3.016667in}}%
\pgfpathlineto{\pgfqpoint{4.585000in}{2.513333in}}%
\pgfpathlineto{\pgfqpoint{4.081667in}{2.513333in}}%
\pgfpathlineto{\pgfqpoint{4.081667in}{3.016667in}}%
\pgfusepath{fill}%
\end{pgfscope}%
\begin{pgfscope}%
\pgfpathrectangle{\pgfqpoint{1.565000in}{0.500000in}}{\pgfqpoint{3.020000in}{3.020000in}}%
\pgfusepath{clip}%
\pgfsetbuttcap%
\pgfsetroundjoin%
\definecolor{currentfill}{rgb}{0.996094,0.996094,0.996094}%
\pgfsetfillcolor{currentfill}%
\pgfsetlinewidth{0.000000pt}%
\definecolor{currentstroke}{rgb}{1.000000,1.000000,1.000000}%
\pgfsetstrokecolor{currentstroke}%
\pgfsetdash{}{0pt}%
\pgfpathmoveto{\pgfqpoint{1.565000in}{2.513333in}}%
\pgfpathlineto{\pgfqpoint{2.068333in}{2.513333in}}%
\pgfpathlineto{\pgfqpoint{2.068333in}{2.010000in}}%
\pgfpathlineto{\pgfqpoint{1.565000in}{2.010000in}}%
\pgfpathlineto{\pgfqpoint{1.565000in}{2.513333in}}%
\pgfusepath{fill}%
\end{pgfscope}%
\begin{pgfscope}%
\pgfpathrectangle{\pgfqpoint{1.565000in}{0.500000in}}{\pgfqpoint{3.020000in}{3.020000in}}%
\pgfusepath{clip}%
\pgfsetbuttcap%
\pgfsetroundjoin%
\definecolor{currentfill}{rgb}{0.749770,0.542050,0.663373}%
\pgfsetfillcolor{currentfill}%
\pgfsetlinewidth{0.000000pt}%
\definecolor{currentstroke}{rgb}{1.000000,1.000000,1.000000}%
\pgfsetstrokecolor{currentstroke}%
\pgfsetdash{}{0pt}%
\pgfpathmoveto{\pgfqpoint{2.068333in}{2.513333in}}%
\pgfpathlineto{\pgfqpoint{2.571667in}{2.513333in}}%
\pgfpathlineto{\pgfqpoint{2.571667in}{2.010000in}}%
\pgfpathlineto{\pgfqpoint{2.068333in}{2.010000in}}%
\pgfpathlineto{\pgfqpoint{2.068333in}{2.513333in}}%
\pgfusepath{fill}%
\end{pgfscope}%
\begin{pgfscope}%
\pgfpathrectangle{\pgfqpoint{1.565000in}{0.500000in}}{\pgfqpoint{3.020000in}{3.020000in}}%
\pgfusepath{clip}%
\pgfsetbuttcap%
\pgfsetroundjoin%
\definecolor{currentfill}{rgb}{0.472656,0.031250,0.289062}%
\pgfsetfillcolor{currentfill}%
\pgfsetlinewidth{0.000000pt}%
\definecolor{currentstroke}{rgb}{1.000000,1.000000,1.000000}%
\pgfsetstrokecolor{currentstroke}%
\pgfsetdash{}{0pt}%
\pgfpathmoveto{\pgfqpoint{2.571667in}{2.513333in}}%
\pgfpathlineto{\pgfqpoint{3.075000in}{2.513333in}}%
\pgfpathlineto{\pgfqpoint{3.075000in}{2.010000in}}%
\pgfpathlineto{\pgfqpoint{2.571667in}{2.010000in}}%
\pgfpathlineto{\pgfqpoint{2.571667in}{2.513333in}}%
\pgfusepath{fill}%
\end{pgfscope}%
\begin{pgfscope}%
\pgfpathrectangle{\pgfqpoint{1.565000in}{0.500000in}}{\pgfqpoint{3.020000in}{3.020000in}}%
\pgfusepath{clip}%
\pgfsetbuttcap%
\pgfsetroundjoin%
\definecolor{currentfill}{rgb}{0.749770,0.542050,0.663373}%
\pgfsetfillcolor{currentfill}%
\pgfsetlinewidth{0.000000pt}%
\definecolor{currentstroke}{rgb}{1.000000,1.000000,1.000000}%
\pgfsetstrokecolor{currentstroke}%
\pgfsetdash{}{0pt}%
\pgfpathmoveto{\pgfqpoint{3.075000in}{2.513333in}}%
\pgfpathlineto{\pgfqpoint{3.578333in}{2.513333in}}%
\pgfpathlineto{\pgfqpoint{3.578333in}{2.010000in}}%
\pgfpathlineto{\pgfqpoint{3.075000in}{2.010000in}}%
\pgfpathlineto{\pgfqpoint{3.075000in}{2.513333in}}%
\pgfusepath{fill}%
\end{pgfscope}%
\begin{pgfscope}%
\pgfpathrectangle{\pgfqpoint{1.565000in}{0.500000in}}{\pgfqpoint{3.020000in}{3.020000in}}%
\pgfusepath{clip}%
\pgfsetbuttcap%
\pgfsetroundjoin%
\definecolor{currentfill}{rgb}{0.955040,0.920420,0.940640}%
\pgfsetfillcolor{currentfill}%
\pgfsetlinewidth{0.000000pt}%
\definecolor{currentstroke}{rgb}{1.000000,1.000000,1.000000}%
\pgfsetstrokecolor{currentstroke}%
\pgfsetdash{}{0pt}%
\pgfpathmoveto{\pgfqpoint{3.578333in}{2.513333in}}%
\pgfpathlineto{\pgfqpoint{4.081667in}{2.513333in}}%
\pgfpathlineto{\pgfqpoint{4.081667in}{2.010000in}}%
\pgfpathlineto{\pgfqpoint{3.578333in}{2.010000in}}%
\pgfpathlineto{\pgfqpoint{3.578333in}{2.513333in}}%
\pgfusepath{fill}%
\end{pgfscope}%
\begin{pgfscope}%
\pgfpathrectangle{\pgfqpoint{1.565000in}{0.500000in}}{\pgfqpoint{3.020000in}{3.020000in}}%
\pgfusepath{clip}%
\pgfsetbuttcap%
\pgfsetroundjoin%
\definecolor{currentfill}{rgb}{0.996094,0.996094,0.996094}%
\pgfsetfillcolor{currentfill}%
\pgfsetlinewidth{0.000000pt}%
\definecolor{currentstroke}{rgb}{1.000000,1.000000,1.000000}%
\pgfsetstrokecolor{currentstroke}%
\pgfsetdash{}{0pt}%
\pgfpathmoveto{\pgfqpoint{4.081667in}{2.513333in}}%
\pgfpathlineto{\pgfqpoint{4.585000in}{2.513333in}}%
\pgfpathlineto{\pgfqpoint{4.585000in}{2.010000in}}%
\pgfpathlineto{\pgfqpoint{4.081667in}{2.010000in}}%
\pgfpathlineto{\pgfqpoint{4.081667in}{2.513333in}}%
\pgfusepath{fill}%
\end{pgfscope}%
\begin{pgfscope}%
\pgfpathrectangle{\pgfqpoint{1.565000in}{0.500000in}}{\pgfqpoint{3.020000in}{3.020000in}}%
\pgfusepath{clip}%
\pgfsetbuttcap%
\pgfsetroundjoin%
\definecolor{currentfill}{rgb}{0.996094,0.996094,0.996094}%
\pgfsetfillcolor{currentfill}%
\pgfsetlinewidth{0.000000pt}%
\definecolor{currentstroke}{rgb}{1.000000,1.000000,1.000000}%
\pgfsetstrokecolor{currentstroke}%
\pgfsetdash{}{0pt}%
\pgfpathmoveto{\pgfqpoint{1.565000in}{2.010000in}}%
\pgfpathlineto{\pgfqpoint{2.068333in}{2.010000in}}%
\pgfpathlineto{\pgfqpoint{2.068333in}{1.506667in}}%
\pgfpathlineto{\pgfqpoint{1.565000in}{1.506667in}}%
\pgfpathlineto{\pgfqpoint{1.565000in}{2.010000in}}%
\pgfusepath{fill}%
\end{pgfscope}%
\begin{pgfscope}%
\pgfpathrectangle{\pgfqpoint{1.565000in}{0.500000in}}{\pgfqpoint{3.020000in}{3.020000in}}%
\pgfusepath{clip}%
\pgfsetbuttcap%
\pgfsetroundjoin%
\definecolor{currentfill}{rgb}{0.996094,0.996094,0.996094}%
\pgfsetfillcolor{currentfill}%
\pgfsetlinewidth{0.000000pt}%
\definecolor{currentstroke}{rgb}{1.000000,1.000000,1.000000}%
\pgfsetstrokecolor{currentstroke}%
\pgfsetdash{}{0pt}%
\pgfpathmoveto{\pgfqpoint{2.068333in}{2.010000in}}%
\pgfpathlineto{\pgfqpoint{2.571667in}{2.010000in}}%
\pgfpathlineto{\pgfqpoint{2.571667in}{1.506667in}}%
\pgfpathlineto{\pgfqpoint{2.068333in}{1.506667in}}%
\pgfpathlineto{\pgfqpoint{2.068333in}{2.010000in}}%
\pgfusepath{fill}%
\end{pgfscope}%
\begin{pgfscope}%
\pgfpathrectangle{\pgfqpoint{1.565000in}{0.500000in}}{\pgfqpoint{3.020000in}{3.020000in}}%
\pgfusepath{clip}%
\pgfsetbuttcap%
\pgfsetroundjoin%
\definecolor{currentfill}{rgb}{0.927671,0.869970,0.903671}%
\pgfsetfillcolor{currentfill}%
\pgfsetlinewidth{0.000000pt}%
\definecolor{currentstroke}{rgb}{1.000000,1.000000,1.000000}%
\pgfsetstrokecolor{currentstroke}%
\pgfsetdash{}{0pt}%
\pgfpathmoveto{\pgfqpoint{2.571667in}{2.010000in}}%
\pgfpathlineto{\pgfqpoint{3.075000in}{2.010000in}}%
\pgfpathlineto{\pgfqpoint{3.075000in}{1.506667in}}%
\pgfpathlineto{\pgfqpoint{2.571667in}{1.506667in}}%
\pgfpathlineto{\pgfqpoint{2.571667in}{2.010000in}}%
\pgfusepath{fill}%
\end{pgfscope}%
\begin{pgfscope}%
\pgfpathrectangle{\pgfqpoint{1.565000in}{0.500000in}}{\pgfqpoint{3.020000in}{3.020000in}}%
\pgfusepath{clip}%
\pgfsetbuttcap%
\pgfsetroundjoin%
\definecolor{currentfill}{rgb}{0.544501,0.163680,0.386106}%
\pgfsetfillcolor{currentfill}%
\pgfsetlinewidth{0.000000pt}%
\definecolor{currentstroke}{rgb}{1.000000,1.000000,1.000000}%
\pgfsetstrokecolor{currentstroke}%
\pgfsetdash{}{0pt}%
\pgfpathmoveto{\pgfqpoint{3.075000in}{2.010000in}}%
\pgfpathlineto{\pgfqpoint{3.578333in}{2.010000in}}%
\pgfpathlineto{\pgfqpoint{3.578333in}{1.506667in}}%
\pgfpathlineto{\pgfqpoint{3.075000in}{1.506667in}}%
\pgfpathlineto{\pgfqpoint{3.075000in}{2.010000in}}%
\pgfusepath{fill}%
\end{pgfscope}%
\begin{pgfscope}%
\pgfpathrectangle{\pgfqpoint{1.565000in}{0.500000in}}{\pgfqpoint{3.020000in}{3.020000in}}%
\pgfusepath{clip}%
\pgfsetbuttcap%
\pgfsetroundjoin%
\definecolor{currentfill}{rgb}{0.736086,0.516825,0.644889}%
\pgfsetfillcolor{currentfill}%
\pgfsetlinewidth{0.000000pt}%
\definecolor{currentstroke}{rgb}{1.000000,1.000000,1.000000}%
\pgfsetstrokecolor{currentstroke}%
\pgfsetdash{}{0pt}%
\pgfpathmoveto{\pgfqpoint{3.578333in}{2.010000in}}%
\pgfpathlineto{\pgfqpoint{4.081667in}{2.010000in}}%
\pgfpathlineto{\pgfqpoint{4.081667in}{1.506667in}}%
\pgfpathlineto{\pgfqpoint{3.578333in}{1.506667in}}%
\pgfpathlineto{\pgfqpoint{3.578333in}{2.010000in}}%
\pgfusepath{fill}%
\end{pgfscope}%
\begin{pgfscope}%
\pgfpathrectangle{\pgfqpoint{1.565000in}{0.500000in}}{\pgfqpoint{3.020000in}{3.020000in}}%
\pgfusepath{clip}%
\pgfsetbuttcap%
\pgfsetroundjoin%
\definecolor{currentfill}{rgb}{0.968724,0.945644,0.959125}%
\pgfsetfillcolor{currentfill}%
\pgfsetlinewidth{0.000000pt}%
\definecolor{currentstroke}{rgb}{1.000000,1.000000,1.000000}%
\pgfsetstrokecolor{currentstroke}%
\pgfsetdash{}{0pt}%
\pgfpathmoveto{\pgfqpoint{4.081667in}{2.010000in}}%
\pgfpathlineto{\pgfqpoint{4.585000in}{2.010000in}}%
\pgfpathlineto{\pgfqpoint{4.585000in}{1.506667in}}%
\pgfpathlineto{\pgfqpoint{4.081667in}{1.506667in}}%
\pgfpathlineto{\pgfqpoint{4.081667in}{2.010000in}}%
\pgfusepath{fill}%
\end{pgfscope}%
\begin{pgfscope}%
\pgfpathrectangle{\pgfqpoint{1.565000in}{0.500000in}}{\pgfqpoint{3.020000in}{3.020000in}}%
\pgfusepath{clip}%
\pgfsetbuttcap%
\pgfsetroundjoin%
\definecolor{currentfill}{rgb}{0.996094,0.996094,0.996094}%
\pgfsetfillcolor{currentfill}%
\pgfsetlinewidth{0.000000pt}%
\definecolor{currentstroke}{rgb}{1.000000,1.000000,1.000000}%
\pgfsetstrokecolor{currentstroke}%
\pgfsetdash{}{0pt}%
\pgfpathmoveto{\pgfqpoint{1.565000in}{1.506667in}}%
\pgfpathlineto{\pgfqpoint{2.068333in}{1.506667in}}%
\pgfpathlineto{\pgfqpoint{2.068333in}{1.003333in}}%
\pgfpathlineto{\pgfqpoint{1.565000in}{1.003333in}}%
\pgfpathlineto{\pgfqpoint{1.565000in}{1.506667in}}%
\pgfusepath{fill}%
\end{pgfscope}%
\begin{pgfscope}%
\pgfpathrectangle{\pgfqpoint{1.565000in}{0.500000in}}{\pgfqpoint{3.020000in}{3.020000in}}%
\pgfusepath{clip}%
\pgfsetbuttcap%
\pgfsetroundjoin%
\definecolor{currentfill}{rgb}{0.982409,0.970869,0.977609}%
\pgfsetfillcolor{currentfill}%
\pgfsetlinewidth{0.000000pt}%
\definecolor{currentstroke}{rgb}{1.000000,1.000000,1.000000}%
\pgfsetstrokecolor{currentstroke}%
\pgfsetdash{}{0pt}%
\pgfpathmoveto{\pgfqpoint{2.068333in}{1.506667in}}%
\pgfpathlineto{\pgfqpoint{2.571667in}{1.506667in}}%
\pgfpathlineto{\pgfqpoint{2.571667in}{1.003333in}}%
\pgfpathlineto{\pgfqpoint{2.068333in}{1.003333in}}%
\pgfpathlineto{\pgfqpoint{2.068333in}{1.506667in}}%
\pgfusepath{fill}%
\end{pgfscope}%
\begin{pgfscope}%
\pgfpathrectangle{\pgfqpoint{1.565000in}{0.500000in}}{\pgfqpoint{3.020000in}{3.020000in}}%
\pgfusepath{clip}%
\pgfsetbuttcap%
\pgfsetroundjoin%
\definecolor{currentfill}{rgb}{0.900301,0.819521,0.866702}%
\pgfsetfillcolor{currentfill}%
\pgfsetlinewidth{0.000000pt}%
\definecolor{currentstroke}{rgb}{1.000000,1.000000,1.000000}%
\pgfsetstrokecolor{currentstroke}%
\pgfsetdash{}{0pt}%
\pgfpathmoveto{\pgfqpoint{2.571667in}{1.506667in}}%
\pgfpathlineto{\pgfqpoint{3.075000in}{1.506667in}}%
\pgfpathlineto{\pgfqpoint{3.075000in}{1.003333in}}%
\pgfpathlineto{\pgfqpoint{2.571667in}{1.003333in}}%
\pgfpathlineto{\pgfqpoint{2.571667in}{1.506667in}}%
\pgfusepath{fill}%
\end{pgfscope}%
\begin{pgfscope}%
\pgfpathrectangle{\pgfqpoint{1.565000in}{0.500000in}}{\pgfqpoint{3.020000in}{3.020000in}}%
\pgfusepath{clip}%
\pgfsetbuttcap%
\pgfsetroundjoin%
\definecolor{currentfill}{rgb}{0.927671,0.869970,0.903671}%
\pgfsetfillcolor{currentfill}%
\pgfsetlinewidth{0.000000pt}%
\definecolor{currentstroke}{rgb}{1.000000,1.000000,1.000000}%
\pgfsetstrokecolor{currentstroke}%
\pgfsetdash{}{0pt}%
\pgfpathmoveto{\pgfqpoint{3.075000in}{1.506667in}}%
\pgfpathlineto{\pgfqpoint{3.578333in}{1.506667in}}%
\pgfpathlineto{\pgfqpoint{3.578333in}{1.003333in}}%
\pgfpathlineto{\pgfqpoint{3.075000in}{1.003333in}}%
\pgfpathlineto{\pgfqpoint{3.075000in}{1.506667in}}%
\pgfusepath{fill}%
\end{pgfscope}%
\begin{pgfscope}%
\pgfpathrectangle{\pgfqpoint{1.565000in}{0.500000in}}{\pgfqpoint{3.020000in}{3.020000in}}%
\pgfusepath{clip}%
\pgfsetbuttcap%
\pgfsetroundjoin%
\definecolor{currentfill}{rgb}{0.472656,0.031250,0.289062}%
\pgfsetfillcolor{currentfill}%
\pgfsetlinewidth{0.000000pt}%
\definecolor{currentstroke}{rgb}{1.000000,1.000000,1.000000}%
\pgfsetstrokecolor{currentstroke}%
\pgfsetdash{}{0pt}%
\pgfpathmoveto{\pgfqpoint{3.578333in}{1.506667in}}%
\pgfpathlineto{\pgfqpoint{4.081667in}{1.506667in}}%
\pgfpathlineto{\pgfqpoint{4.081667in}{1.003333in}}%
\pgfpathlineto{\pgfqpoint{3.578333in}{1.003333in}}%
\pgfpathlineto{\pgfqpoint{3.578333in}{1.506667in}}%
\pgfusepath{fill}%
\end{pgfscope}%
\begin{pgfscope}%
\pgfpathrectangle{\pgfqpoint{1.565000in}{0.500000in}}{\pgfqpoint{3.020000in}{3.020000in}}%
\pgfusepath{clip}%
\pgfsetbuttcap%
\pgfsetroundjoin%
\definecolor{currentfill}{rgb}{0.900301,0.819521,0.866702}%
\pgfsetfillcolor{currentfill}%
\pgfsetlinewidth{0.000000pt}%
\definecolor{currentstroke}{rgb}{1.000000,1.000000,1.000000}%
\pgfsetstrokecolor{currentstroke}%
\pgfsetdash{}{0pt}%
\pgfpathmoveto{\pgfqpoint{4.081667in}{1.506667in}}%
\pgfpathlineto{\pgfqpoint{4.585000in}{1.506667in}}%
\pgfpathlineto{\pgfqpoint{4.585000in}{1.003333in}}%
\pgfpathlineto{\pgfqpoint{4.081667in}{1.003333in}}%
\pgfpathlineto{\pgfqpoint{4.081667in}{1.506667in}}%
\pgfusepath{fill}%
\end{pgfscope}%
\begin{pgfscope}%
\pgfpathrectangle{\pgfqpoint{1.565000in}{0.500000in}}{\pgfqpoint{3.020000in}{3.020000in}}%
\pgfusepath{clip}%
\pgfsetbuttcap%
\pgfsetroundjoin%
\definecolor{currentfill}{rgb}{0.982409,0.970869,0.977609}%
\pgfsetfillcolor{currentfill}%
\pgfsetlinewidth{0.000000pt}%
\definecolor{currentstroke}{rgb}{1.000000,1.000000,1.000000}%
\pgfsetstrokecolor{currentstroke}%
\pgfsetdash{}{0pt}%
\pgfpathmoveto{\pgfqpoint{1.565000in}{1.003333in}}%
\pgfpathlineto{\pgfqpoint{2.068333in}{1.003333in}}%
\pgfpathlineto{\pgfqpoint{2.068333in}{0.500000in}}%
\pgfpathlineto{\pgfqpoint{1.565000in}{0.500000in}}%
\pgfpathlineto{\pgfqpoint{1.565000in}{1.003333in}}%
\pgfusepath{fill}%
\end{pgfscope}%
\begin{pgfscope}%
\pgfpathrectangle{\pgfqpoint{1.565000in}{0.500000in}}{\pgfqpoint{3.020000in}{3.020000in}}%
\pgfusepath{clip}%
\pgfsetbuttcap%
\pgfsetroundjoin%
\definecolor{currentfill}{rgb}{0.996094,0.996094,0.996094}%
\pgfsetfillcolor{currentfill}%
\pgfsetlinewidth{0.000000pt}%
\definecolor{currentstroke}{rgb}{1.000000,1.000000,1.000000}%
\pgfsetstrokecolor{currentstroke}%
\pgfsetdash{}{0pt}%
\pgfpathmoveto{\pgfqpoint{2.068333in}{1.003333in}}%
\pgfpathlineto{\pgfqpoint{2.571667in}{1.003333in}}%
\pgfpathlineto{\pgfqpoint{2.571667in}{0.500000in}}%
\pgfpathlineto{\pgfqpoint{2.068333in}{0.500000in}}%
\pgfpathlineto{\pgfqpoint{2.068333in}{1.003333in}}%
\pgfusepath{fill}%
\end{pgfscope}%
\begin{pgfscope}%
\pgfpathrectangle{\pgfqpoint{1.565000in}{0.500000in}}{\pgfqpoint{3.020000in}{3.020000in}}%
\pgfusepath{clip}%
\pgfsetbuttcap%
\pgfsetroundjoin%
\definecolor{currentfill}{rgb}{0.927671,0.869970,0.903671}%
\pgfsetfillcolor{currentfill}%
\pgfsetlinewidth{0.000000pt}%
\definecolor{currentstroke}{rgb}{1.000000,1.000000,1.000000}%
\pgfsetstrokecolor{currentstroke}%
\pgfsetdash{}{0pt}%
\pgfpathmoveto{\pgfqpoint{2.571667in}{1.003333in}}%
\pgfpathlineto{\pgfqpoint{3.075000in}{1.003333in}}%
\pgfpathlineto{\pgfqpoint{3.075000in}{0.500000in}}%
\pgfpathlineto{\pgfqpoint{2.571667in}{0.500000in}}%
\pgfpathlineto{\pgfqpoint{2.571667in}{1.003333in}}%
\pgfusepath{fill}%
\end{pgfscope}%
\begin{pgfscope}%
\pgfpathrectangle{\pgfqpoint{1.565000in}{0.500000in}}{\pgfqpoint{3.020000in}{3.020000in}}%
\pgfusepath{clip}%
\pgfsetbuttcap%
\pgfsetroundjoin%
\definecolor{currentfill}{rgb}{0.941355,0.895195,0.922156}%
\pgfsetfillcolor{currentfill}%
\pgfsetlinewidth{0.000000pt}%
\definecolor{currentstroke}{rgb}{1.000000,1.000000,1.000000}%
\pgfsetstrokecolor{currentstroke}%
\pgfsetdash{}{0pt}%
\pgfpathmoveto{\pgfqpoint{3.075000in}{1.003333in}}%
\pgfpathlineto{\pgfqpoint{3.578333in}{1.003333in}}%
\pgfpathlineto{\pgfqpoint{3.578333in}{0.500000in}}%
\pgfpathlineto{\pgfqpoint{3.075000in}{0.500000in}}%
\pgfpathlineto{\pgfqpoint{3.075000in}{1.003333in}}%
\pgfusepath{fill}%
\end{pgfscope}%
\begin{pgfscope}%
\pgfpathrectangle{\pgfqpoint{1.565000in}{0.500000in}}{\pgfqpoint{3.020000in}{3.020000in}}%
\pgfusepath{clip}%
\pgfsetbuttcap%
\pgfsetroundjoin%
\definecolor{currentfill}{rgb}{0.626608,0.315028,0.497013}%
\pgfsetfillcolor{currentfill}%
\pgfsetlinewidth{0.000000pt}%
\definecolor{currentstroke}{rgb}{1.000000,1.000000,1.000000}%
\pgfsetstrokecolor{currentstroke}%
\pgfsetdash{}{0pt}%
\pgfpathmoveto{\pgfqpoint{3.578333in}{1.003333in}}%
\pgfpathlineto{\pgfqpoint{4.081667in}{1.003333in}}%
\pgfpathlineto{\pgfqpoint{4.081667in}{0.500000in}}%
\pgfpathlineto{\pgfqpoint{3.578333in}{0.500000in}}%
\pgfpathlineto{\pgfqpoint{3.578333in}{1.003333in}}%
\pgfusepath{fill}%
\end{pgfscope}%
\begin{pgfscope}%
\pgfpathrectangle{\pgfqpoint{1.565000in}{0.500000in}}{\pgfqpoint{3.020000in}{3.020000in}}%
\pgfusepath{clip}%
\pgfsetbuttcap%
\pgfsetroundjoin%
\definecolor{currentfill}{rgb}{0.476077,0.037556,0.293684}%
\pgfsetfillcolor{currentfill}%
\pgfsetlinewidth{0.000000pt}%
\definecolor{currentstroke}{rgb}{1.000000,1.000000,1.000000}%
\pgfsetstrokecolor{currentstroke}%
\pgfsetdash{}{0pt}%
\pgfpathmoveto{\pgfqpoint{4.081667in}{1.003333in}}%
\pgfpathlineto{\pgfqpoint{4.585000in}{1.003333in}}%
\pgfpathlineto{\pgfqpoint{4.585000in}{0.500000in}}%
\pgfpathlineto{\pgfqpoint{4.081667in}{0.500000in}}%
\pgfpathlineto{\pgfqpoint{4.081667in}{1.003333in}}%
\pgfusepath{fill}%
\end{pgfscope}%
\begin{pgfscope}%
\definecolor{textcolor}{rgb}{0.000000,0.000000,0.000000}%
\pgfsetstrokecolor{textcolor}%
\pgfsetfillcolor{textcolor}%
\pgftext[x=1.816667in,y=0.402778in,,top]{\color{textcolor}\rmfamily\fontsize{10.000000}{12.000000}\selectfont 0}%
\end{pgfscope}%
\begin{pgfscope}%
\definecolor{textcolor}{rgb}{0.000000,0.000000,0.000000}%
\pgfsetstrokecolor{textcolor}%
\pgfsetfillcolor{textcolor}%
\pgftext[x=2.320000in,y=0.402778in,,top]{\color{textcolor}\rmfamily\fontsize{10.000000}{12.000000}\selectfont 1}%
\end{pgfscope}%
\begin{pgfscope}%
\definecolor{textcolor}{rgb}{0.000000,0.000000,0.000000}%
\pgfsetstrokecolor{textcolor}%
\pgfsetfillcolor{textcolor}%
\pgftext[x=2.823333in,y=0.402778in,,top]{\color{textcolor}\rmfamily\fontsize{10.000000}{12.000000}\selectfont 2}%
\end{pgfscope}%
\begin{pgfscope}%
\definecolor{textcolor}{rgb}{0.000000,0.000000,0.000000}%
\pgfsetstrokecolor{textcolor}%
\pgfsetfillcolor{textcolor}%
\pgftext[x=3.326667in,y=0.402778in,,top]{\color{textcolor}\rmfamily\fontsize{10.000000}{12.000000}\selectfont 3}%
\end{pgfscope}%
\begin{pgfscope}%
\definecolor{textcolor}{rgb}{0.000000,0.000000,0.000000}%
\pgfsetstrokecolor{textcolor}%
\pgfsetfillcolor{textcolor}%
\pgftext[x=3.830000in,y=0.402778in,,top]{\color{textcolor}\rmfamily\fontsize{10.000000}{12.000000}\selectfont 4}%
\end{pgfscope}%
\begin{pgfscope}%
\definecolor{textcolor}{rgb}{0.000000,0.000000,0.000000}%
\pgfsetstrokecolor{textcolor}%
\pgfsetfillcolor{textcolor}%
\pgftext[x=4.333333in,y=0.402778in,,top]{\color{textcolor}\rmfamily\fontsize{10.000000}{12.000000}\selectfont 5}%
\end{pgfscope}%
\begin{pgfscope}%
\definecolor{textcolor}{rgb}{0.000000,0.000000,0.000000}%
\pgfsetstrokecolor{textcolor}%
\pgfsetfillcolor{textcolor}%
\pgftext[x=3.075000in,y=0.195988in,,top]{\color{textcolor}\rmfamily\fontsize{15.000000}{18.000000}\selectfont Predicted ISUP grade}%
\end{pgfscope}%
\begin{pgfscope}%
\definecolor{textcolor}{rgb}{0.000000,0.000000,0.000000}%
\pgfsetstrokecolor{textcolor}%
\pgfsetfillcolor{textcolor}%
\pgftext[x=1.440772in, y=3.247114in, left, base,rotate=90.000000]{\color{textcolor}\rmfamily\fontsize{10.000000}{12.000000}\selectfont 0}%
\end{pgfscope}%
\begin{pgfscope}%
\definecolor{textcolor}{rgb}{0.000000,0.000000,0.000000}%
\pgfsetstrokecolor{textcolor}%
\pgfsetfillcolor{textcolor}%
\pgftext[x=1.440772in, y=2.743781in, left, base,rotate=90.000000]{\color{textcolor}\rmfamily\fontsize{10.000000}{12.000000}\selectfont 1}%
\end{pgfscope}%
\begin{pgfscope}%
\definecolor{textcolor}{rgb}{0.000000,0.000000,0.000000}%
\pgfsetstrokecolor{textcolor}%
\pgfsetfillcolor{textcolor}%
\pgftext[x=1.440772in, y=2.240447in, left, base,rotate=90.000000]{\color{textcolor}\rmfamily\fontsize{10.000000}{12.000000}\selectfont 2}%
\end{pgfscope}%
\begin{pgfscope}%
\definecolor{textcolor}{rgb}{0.000000,0.000000,0.000000}%
\pgfsetstrokecolor{textcolor}%
\pgfsetfillcolor{textcolor}%
\pgftext[x=1.440772in, y=1.737114in, left, base,rotate=90.000000]{\color{textcolor}\rmfamily\fontsize{10.000000}{12.000000}\selectfont 3}%
\end{pgfscope}%
\begin{pgfscope}%
\definecolor{textcolor}{rgb}{0.000000,0.000000,0.000000}%
\pgfsetstrokecolor{textcolor}%
\pgfsetfillcolor{textcolor}%
\pgftext[x=1.440772in, y=1.233781in, left, base,rotate=90.000000]{\color{textcolor}\rmfamily\fontsize{10.000000}{12.000000}\selectfont 4}%
\end{pgfscope}%
\begin{pgfscope}%
\definecolor{textcolor}{rgb}{0.000000,0.000000,0.000000}%
\pgfsetstrokecolor{textcolor}%
\pgfsetfillcolor{textcolor}%
\pgftext[x=1.440772in, y=0.730447in, left, base,rotate=90.000000]{\color{textcolor}\rmfamily\fontsize{10.000000}{12.000000}\selectfont 5}%
\end{pgfscope}%
\begin{pgfscope}%
\definecolor{textcolor}{rgb}{0.000000,0.000000,0.000000}%
\pgfsetstrokecolor{textcolor}%
\pgfsetfillcolor{textcolor}%
\pgftext[x=1.260988in,y=2.010000in,,bottom,rotate=90.000000]{\color{textcolor}\rmfamily\fontsize{15.000000}{18.000000}\selectfont Ground truth ISUP grade}%
\end{pgfscope}%
\begin{pgfscope}%
\pgfsetrectcap%
\pgfsetmiterjoin%
\pgfsetlinewidth{0.803000pt}%
\definecolor{currentstroke}{rgb}{0.000000,0.000000,0.000000}%
\pgfsetstrokecolor{currentstroke}%
\pgfsetdash{}{0pt}%
\pgfpathmoveto{\pgfqpoint{1.565000in}{0.500000in}}%
\pgfpathlineto{\pgfqpoint{1.565000in}{3.520000in}}%
\pgfusepath{stroke}%
\end{pgfscope}%
\begin{pgfscope}%
\pgfsetrectcap%
\pgfsetmiterjoin%
\pgfsetlinewidth{0.803000pt}%
\definecolor{currentstroke}{rgb}{0.000000,0.000000,0.000000}%
\pgfsetstrokecolor{currentstroke}%
\pgfsetdash{}{0pt}%
\pgfpathmoveto{\pgfqpoint{4.585000in}{0.500000in}}%
\pgfpathlineto{\pgfqpoint{4.585000in}{3.520000in}}%
\pgfusepath{stroke}%
\end{pgfscope}%
\begin{pgfscope}%
\pgfsetrectcap%
\pgfsetmiterjoin%
\pgfsetlinewidth{0.803000pt}%
\definecolor{currentstroke}{rgb}{0.000000,0.000000,0.000000}%
\pgfsetstrokecolor{currentstroke}%
\pgfsetdash{}{0pt}%
\pgfpathmoveto{\pgfqpoint{1.565000in}{0.500000in}}%
\pgfpathlineto{\pgfqpoint{4.585000in}{0.500000in}}%
\pgfusepath{stroke}%
\end{pgfscope}%
\begin{pgfscope}%
\pgfsetrectcap%
\pgfsetmiterjoin%
\pgfsetlinewidth{0.803000pt}%
\definecolor{currentstroke}{rgb}{0.000000,0.000000,0.000000}%
\pgfsetstrokecolor{currentstroke}%
\pgfsetdash{}{0pt}%
\pgfpathmoveto{\pgfqpoint{1.565000in}{3.520000in}}%
\pgfpathlineto{\pgfqpoint{4.585000in}{3.520000in}}%
\pgfusepath{stroke}%
\end{pgfscope}%
\begin{pgfscope}%
\definecolor{textcolor}{rgb}{1.000000,1.000000,1.000000}%
\pgfsetstrokecolor{textcolor}%
\pgfsetfillcolor{textcolor}%
\pgftext[x=1.816667in,y=3.268333in,,]{\color{textcolor}\rmfamily\fontsize{10.000000}{12.000000}\selectfont 348}%
\end{pgfscope}%
\begin{pgfscope}%
\definecolor{textcolor}{rgb}{1.000000,1.000000,1.000000}%
\pgfsetstrokecolor{textcolor}%
\pgfsetfillcolor{textcolor}%
\pgftext[x=2.320000in,y=3.268333in,,]{\color{textcolor}\rmfamily\fontsize{10.000000}{12.000000}\selectfont 21}%
\end{pgfscope}%
\begin{pgfscope}%
\definecolor{textcolor}{rgb}{0.150000,0.150000,0.150000}%
\pgfsetstrokecolor{textcolor}%
\pgfsetfillcolor{textcolor}%
\pgftext[x=2.823333in,y=3.268333in,,]{\color{textcolor}\rmfamily\fontsize{10.000000}{12.000000}\selectfont 3}%
\end{pgfscope}%
\begin{pgfscope}%
\definecolor{textcolor}{rgb}{0.150000,0.150000,0.150000}%
\pgfsetstrokecolor{textcolor}%
\pgfsetfillcolor{textcolor}%
\pgftext[x=3.326667in,y=3.268333in,,]{\color{textcolor}\rmfamily\fontsize{10.000000}{12.000000}\selectfont 1}%
\end{pgfscope}%
\begin{pgfscope}%
\definecolor{textcolor}{rgb}{0.150000,0.150000,0.150000}%
\pgfsetstrokecolor{textcolor}%
\pgfsetfillcolor{textcolor}%
\pgftext[x=3.830000in,y=3.268333in,,]{\color{textcolor}\rmfamily\fontsize{10.000000}{12.000000}\selectfont 1}%
\end{pgfscope}%
\begin{pgfscope}%
\definecolor{textcolor}{rgb}{0.150000,0.150000,0.150000}%
\pgfsetstrokecolor{textcolor}%
\pgfsetfillcolor{textcolor}%
\pgftext[x=4.333333in,y=3.268333in,,]{\color{textcolor}\rmfamily\fontsize{10.000000}{12.000000}\selectfont 0}%
\end{pgfscope}%
\begin{pgfscope}%
\definecolor{textcolor}{rgb}{0.150000,0.150000,0.150000}%
\pgfsetstrokecolor{textcolor}%
\pgfsetfillcolor{textcolor}%
\pgftext[x=1.816667in,y=2.765000in,,]{\color{textcolor}\rmfamily\fontsize{10.000000}{12.000000}\selectfont 7}%
\end{pgfscope}%
\begin{pgfscope}%
\definecolor{textcolor}{rgb}{1.000000,1.000000,1.000000}%
\pgfsetstrokecolor{textcolor}%
\pgfsetfillcolor{textcolor}%
\pgftext[x=2.320000in,y=2.765000in,,]{\color{textcolor}\rmfamily\fontsize{10.000000}{12.000000}\selectfont 129}%
\end{pgfscope}%
\begin{pgfscope}%
\definecolor{textcolor}{rgb}{1.000000,1.000000,1.000000}%
\pgfsetstrokecolor{textcolor}%
\pgfsetfillcolor{textcolor}%
\pgftext[x=2.823333in,y=2.765000in,,]{\color{textcolor}\rmfamily\fontsize{10.000000}{12.000000}\selectfont 32}%
\end{pgfscope}%
\begin{pgfscope}%
\definecolor{textcolor}{rgb}{0.150000,0.150000,0.150000}%
\pgfsetstrokecolor{textcolor}%
\pgfsetfillcolor{textcolor}%
\pgftext[x=3.326667in,y=2.765000in,,]{\color{textcolor}\rmfamily\fontsize{10.000000}{12.000000}\selectfont 3}%
\end{pgfscope}%
\begin{pgfscope}%
\definecolor{textcolor}{rgb}{0.150000,0.150000,0.150000}%
\pgfsetstrokecolor{textcolor}%
\pgfsetfillcolor{textcolor}%
\pgftext[x=3.830000in,y=2.765000in,,]{\color{textcolor}\rmfamily\fontsize{10.000000}{12.000000}\selectfont 2}%
\end{pgfscope}%
\begin{pgfscope}%
\definecolor{textcolor}{rgb}{0.150000,0.150000,0.150000}%
\pgfsetstrokecolor{textcolor}%
\pgfsetfillcolor{textcolor}%
\pgftext[x=4.333333in,y=2.765000in,,]{\color{textcolor}\rmfamily\fontsize{10.000000}{12.000000}\selectfont 0}%
\end{pgfscope}%
\begin{pgfscope}%
\definecolor{textcolor}{rgb}{0.150000,0.150000,0.150000}%
\pgfsetstrokecolor{textcolor}%
\pgfsetfillcolor{textcolor}%
\pgftext[x=1.816667in,y=2.261667in,,]{\color{textcolor}\rmfamily\fontsize{10.000000}{12.000000}\selectfont 1}%
\end{pgfscope}%
\begin{pgfscope}%
\definecolor{textcolor}{rgb}{1.000000,1.000000,1.000000}%
\pgfsetstrokecolor{textcolor}%
\pgfsetfillcolor{textcolor}%
\pgftext[x=2.320000in,y=2.261667in,,]{\color{textcolor}\rmfamily\fontsize{10.000000}{12.000000}\selectfont 25}%
\end{pgfscope}%
\begin{pgfscope}%
\definecolor{textcolor}{rgb}{1.000000,1.000000,1.000000}%
\pgfsetstrokecolor{textcolor}%
\pgfsetfillcolor{textcolor}%
\pgftext[x=2.823333in,y=2.261667in,,]{\color{textcolor}\rmfamily\fontsize{10.000000}{12.000000}\selectfont 60}%
\end{pgfscope}%
\begin{pgfscope}%
\definecolor{textcolor}{rgb}{1.000000,1.000000,1.000000}%
\pgfsetstrokecolor{textcolor}%
\pgfsetfillcolor{textcolor}%
\pgftext[x=3.326667in,y=2.261667in,,]{\color{textcolor}\rmfamily\fontsize{10.000000}{12.000000}\selectfont 25}%
\end{pgfscope}%
\begin{pgfscope}%
\definecolor{textcolor}{rgb}{0.150000,0.150000,0.150000}%
\pgfsetstrokecolor{textcolor}%
\pgfsetfillcolor{textcolor}%
\pgftext[x=3.830000in,y=2.261667in,,]{\color{textcolor}\rmfamily\fontsize{10.000000}{12.000000}\selectfont 5}%
\end{pgfscope}%
\begin{pgfscope}%
\definecolor{textcolor}{rgb}{0.150000,0.150000,0.150000}%
\pgfsetstrokecolor{textcolor}%
\pgfsetfillcolor{textcolor}%
\pgftext[x=4.333333in,y=2.261667in,,]{\color{textcolor}\rmfamily\fontsize{10.000000}{12.000000}\selectfont 0}%
\end{pgfscope}%
\begin{pgfscope}%
\definecolor{textcolor}{rgb}{0.150000,0.150000,0.150000}%
\pgfsetstrokecolor{textcolor}%
\pgfsetfillcolor{textcolor}%
\pgftext[x=1.816667in,y=1.758333in,,]{\color{textcolor}\rmfamily\fontsize{10.000000}{12.000000}\selectfont 1}%
\end{pgfscope}%
\begin{pgfscope}%
\definecolor{textcolor}{rgb}{0.150000,0.150000,0.150000}%
\pgfsetstrokecolor{textcolor}%
\pgfsetfillcolor{textcolor}%
\pgftext[x=2.320000in,y=1.758333in,,]{\color{textcolor}\rmfamily\fontsize{10.000000}{12.000000}\selectfont 0}%
\end{pgfscope}%
\begin{pgfscope}%
\definecolor{textcolor}{rgb}{0.150000,0.150000,0.150000}%
\pgfsetstrokecolor{textcolor}%
\pgfsetfillcolor{textcolor}%
\pgftext[x=2.823333in,y=1.758333in,,]{\color{textcolor}\rmfamily\fontsize{10.000000}{12.000000}\selectfont 8}%
\end{pgfscope}%
\begin{pgfscope}%
\definecolor{textcolor}{rgb}{1.000000,1.000000,1.000000}%
\pgfsetstrokecolor{textcolor}%
\pgfsetfillcolor{textcolor}%
\pgftext[x=3.326667in,y=1.758333in,,]{\color{textcolor}\rmfamily\fontsize{10.000000}{12.000000}\selectfont 45}%
\end{pgfscope}%
\begin{pgfscope}%
\definecolor{textcolor}{rgb}{1.000000,1.000000,1.000000}%
\pgfsetstrokecolor{textcolor}%
\pgfsetfillcolor{textcolor}%
\pgftext[x=3.830000in,y=1.758333in,,]{\color{textcolor}\rmfamily\fontsize{10.000000}{12.000000}\selectfont 26}%
\end{pgfscope}%
\begin{pgfscope}%
\definecolor{textcolor}{rgb}{0.150000,0.150000,0.150000}%
\pgfsetstrokecolor{textcolor}%
\pgfsetfillcolor{textcolor}%
\pgftext[x=4.333333in,y=1.758333in,,]{\color{textcolor}\rmfamily\fontsize{10.000000}{12.000000}\selectfont 3}%
\end{pgfscope}%
\begin{pgfscope}%
\definecolor{textcolor}{rgb}{0.150000,0.150000,0.150000}%
\pgfsetstrokecolor{textcolor}%
\pgfsetfillcolor{textcolor}%
\pgftext[x=1.816667in,y=1.255000in,,]{\color{textcolor}\rmfamily\fontsize{10.000000}{12.000000}\selectfont 0}%
\end{pgfscope}%
\begin{pgfscope}%
\definecolor{textcolor}{rgb}{0.150000,0.150000,0.150000}%
\pgfsetstrokecolor{textcolor}%
\pgfsetfillcolor{textcolor}%
\pgftext[x=2.320000in,y=1.255000in,,]{\color{textcolor}\rmfamily\fontsize{10.000000}{12.000000}\selectfont 2}%
\end{pgfscope}%
\begin{pgfscope}%
\definecolor{textcolor}{rgb}{0.150000,0.150000,0.150000}%
\pgfsetstrokecolor{textcolor}%
\pgfsetfillcolor{textcolor}%
\pgftext[x=2.823333in,y=1.255000in,,]{\color{textcolor}\rmfamily\fontsize{10.000000}{12.000000}\selectfont 10}%
\end{pgfscope}%
\begin{pgfscope}%
\definecolor{textcolor}{rgb}{0.150000,0.150000,0.150000}%
\pgfsetstrokecolor{textcolor}%
\pgfsetfillcolor{textcolor}%
\pgftext[x=3.326667in,y=1.255000in,,]{\color{textcolor}\rmfamily\fontsize{10.000000}{12.000000}\selectfont 8}%
\end{pgfscope}%
\begin{pgfscope}%
\definecolor{textcolor}{rgb}{1.000000,1.000000,1.000000}%
\pgfsetstrokecolor{textcolor}%
\pgfsetfillcolor{textcolor}%
\pgftext[x=3.830000in,y=1.255000in,,]{\color{textcolor}\rmfamily\fontsize{10.000000}{12.000000}\selectfont 57}%
\end{pgfscope}%
\begin{pgfscope}%
\definecolor{textcolor}{rgb}{0.150000,0.150000,0.150000}%
\pgfsetstrokecolor{textcolor}%
\pgfsetfillcolor{textcolor}%
\pgftext[x=4.333333in,y=1.255000in,,]{\color{textcolor}\rmfamily\fontsize{10.000000}{12.000000}\selectfont 10}%
\end{pgfscope}%
\begin{pgfscope}%
\definecolor{textcolor}{rgb}{0.150000,0.150000,0.150000}%
\pgfsetstrokecolor{textcolor}%
\pgfsetfillcolor{textcolor}%
\pgftext[x=1.816667in,y=0.751667in,,]{\color{textcolor}\rmfamily\fontsize{10.000000}{12.000000}\selectfont 2}%
\end{pgfscope}%
\begin{pgfscope}%
\definecolor{textcolor}{rgb}{0.150000,0.150000,0.150000}%
\pgfsetstrokecolor{textcolor}%
\pgfsetfillcolor{textcolor}%
\pgftext[x=2.320000in,y=0.751667in,,]{\color{textcolor}\rmfamily\fontsize{10.000000}{12.000000}\selectfont 0}%
\end{pgfscope}%
\begin{pgfscope}%
\definecolor{textcolor}{rgb}{0.150000,0.150000,0.150000}%
\pgfsetstrokecolor{textcolor}%
\pgfsetfillcolor{textcolor}%
\pgftext[x=2.823333in,y=0.751667in,,]{\color{textcolor}\rmfamily\fontsize{10.000000}{12.000000}\selectfont 7}%
\end{pgfscope}%
\begin{pgfscope}%
\definecolor{textcolor}{rgb}{0.150000,0.150000,0.150000}%
\pgfsetstrokecolor{textcolor}%
\pgfsetfillcolor{textcolor}%
\pgftext[x=3.326667in,y=0.751667in,,]{\color{textcolor}\rmfamily\fontsize{10.000000}{12.000000}\selectfont 6}%
\end{pgfscope}%
\begin{pgfscope}%
\definecolor{textcolor}{rgb}{1.000000,1.000000,1.000000}%
\pgfsetstrokecolor{textcolor}%
\pgfsetfillcolor{textcolor}%
\pgftext[x=3.830000in,y=0.751667in,,]{\color{textcolor}\rmfamily\fontsize{10.000000}{12.000000}\selectfont 37}%
\end{pgfscope}%
\begin{pgfscope}%
\definecolor{textcolor}{rgb}{1.000000,1.000000,1.000000}%
\pgfsetstrokecolor{textcolor}%
\pgfsetfillcolor{textcolor}%
\pgftext[x=4.333333in,y=0.751667in,,]{\color{textcolor}\rmfamily\fontsize{10.000000}{12.000000}\selectfont 53}%
\end{pgfscope}%
\begin{pgfscope}%
\definecolor{textcolor}{rgb}{0.000000,0.000000,0.000000}%
\pgfsetstrokecolor{textcolor}%
\pgfsetfillcolor{textcolor}%
\pgftext[x=3.075000in,y=3.686667in,,base]{\color{textcolor}\rmfamily\fontsize{18.000000}{21.600000}\selectfont Internal test set}%
\end{pgfscope}%
\end{pgfpicture}%
\makeatother%
\endgroup%
}
    \end{subfigure}%
    \begin{subfigure}[b]{.5\linewidth}
      \centering
      \resizebox{1.5\textwidth}{!}{%% Creator: Matplotlib, PGF backend
%%
%% To include the figure in your LaTeX document, write
%%   \input{<filename>.pgf}
%%
%% Make sure the required packages are loaded in your preamble
%%   \usepackage{pgf}
%%
%% Figures using additional raster images can only be included by \input if
%% they are in the same directory as the main LaTeX file. For loading figures
%% from other directories you can use the `import` package
%%   \usepackage{import}
%%
%% and then include the figures with
%%   \import{<path to file>}{<filename>.pgf}
%%
%% Matplotlib used the following preamble
%%
\begingroup%
\makeatletter%
\begin{pgfpicture}%
\pgfpathrectangle{\pgfpointorigin}{\pgfqpoint{6.000000in}{4.000000in}}%
\pgfusepath{use as bounding box, clip}%
\begin{pgfscope}%
\pgfsetbuttcap%
\pgfsetmiterjoin%
\pgfsetlinewidth{0.000000pt}%
\definecolor{currentstroke}{rgb}{1.000000,1.000000,1.000000}%
\pgfsetstrokecolor{currentstroke}%
\pgfsetstrokeopacity{0.000000}%
\pgfsetdash{}{0pt}%
\pgfpathmoveto{\pgfqpoint{0.000000in}{0.000000in}}%
\pgfpathlineto{\pgfqpoint{6.000000in}{0.000000in}}%
\pgfpathlineto{\pgfqpoint{6.000000in}{4.000000in}}%
\pgfpathlineto{\pgfqpoint{0.000000in}{4.000000in}}%
\pgfpathclose%
\pgfusepath{}%
\end{pgfscope}%
\begin{pgfscope}%
\pgfsetbuttcap%
\pgfsetmiterjoin%
\definecolor{currentfill}{rgb}{1.000000,1.000000,1.000000}%
\pgfsetfillcolor{currentfill}%
\pgfsetlinewidth{0.000000pt}%
\definecolor{currentstroke}{rgb}{0.000000,0.000000,0.000000}%
\pgfsetstrokecolor{currentstroke}%
\pgfsetstrokeopacity{0.000000}%
\pgfsetdash{}{0pt}%
\pgfpathmoveto{\pgfqpoint{1.565000in}{0.500000in}}%
\pgfpathlineto{\pgfqpoint{4.585000in}{0.500000in}}%
\pgfpathlineto{\pgfqpoint{4.585000in}{3.520000in}}%
\pgfpathlineto{\pgfqpoint{1.565000in}{3.520000in}}%
\pgfpathclose%
\pgfusepath{fill}%
\end{pgfscope}%
\begin{pgfscope}%
\pgfpathrectangle{\pgfqpoint{1.565000in}{0.500000in}}{\pgfqpoint{3.020000in}{3.020000in}}%
\pgfusepath{clip}%
\pgfsetbuttcap%
\pgfsetroundjoin%
\definecolor{currentfill}{rgb}{0.472656,0.031250,0.289062}%
\pgfsetfillcolor{currentfill}%
\pgfsetlinewidth{0.000000pt}%
\definecolor{currentstroke}{rgb}{1.000000,1.000000,1.000000}%
\pgfsetstrokecolor{currentstroke}%
\pgfsetdash{}{0pt}%
\pgfpathmoveto{\pgfqpoint{1.565000in}{3.520000in}}%
\pgfpathlineto{\pgfqpoint{2.068333in}{3.520000in}}%
\pgfpathlineto{\pgfqpoint{2.068333in}{3.016667in}}%
\pgfpathlineto{\pgfqpoint{1.565000in}{3.016667in}}%
\pgfpathlineto{\pgfqpoint{1.565000in}{3.520000in}}%
\pgfusepath{fill}%
\end{pgfscope}%
\begin{pgfscope}%
\pgfpathrectangle{\pgfqpoint{1.565000in}{0.500000in}}{\pgfqpoint{3.020000in}{3.020000in}}%
\pgfusepath{clip}%
\pgfsetbuttcap%
\pgfsetroundjoin%
\definecolor{currentfill}{rgb}{0.965303,0.939338,0.954504}%
\pgfsetfillcolor{currentfill}%
\pgfsetlinewidth{0.000000pt}%
\definecolor{currentstroke}{rgb}{1.000000,1.000000,1.000000}%
\pgfsetstrokecolor{currentstroke}%
\pgfsetdash{}{0pt}%
\pgfpathmoveto{\pgfqpoint{2.068333in}{3.520000in}}%
\pgfpathlineto{\pgfqpoint{2.571667in}{3.520000in}}%
\pgfpathlineto{\pgfqpoint{2.571667in}{3.016667in}}%
\pgfpathlineto{\pgfqpoint{2.068333in}{3.016667in}}%
\pgfpathlineto{\pgfqpoint{2.068333in}{3.520000in}}%
\pgfusepath{fill}%
\end{pgfscope}%
\begin{pgfscope}%
\pgfpathrectangle{\pgfqpoint{1.565000in}{0.500000in}}{\pgfqpoint{3.020000in}{3.020000in}}%
\pgfusepath{clip}%
\pgfsetbuttcap%
\pgfsetroundjoin%
\definecolor{currentfill}{rgb}{0.991988,0.988526,0.990548}%
\pgfsetfillcolor{currentfill}%
\pgfsetlinewidth{0.000000pt}%
\definecolor{currentstroke}{rgb}{1.000000,1.000000,1.000000}%
\pgfsetstrokecolor{currentstroke}%
\pgfsetdash{}{0pt}%
\pgfpathmoveto{\pgfqpoint{2.571667in}{3.520000in}}%
\pgfpathlineto{\pgfqpoint{3.075000in}{3.520000in}}%
\pgfpathlineto{\pgfqpoint{3.075000in}{3.016667in}}%
\pgfpathlineto{\pgfqpoint{2.571667in}{3.016667in}}%
\pgfpathlineto{\pgfqpoint{2.571667in}{3.520000in}}%
\pgfusepath{fill}%
\end{pgfscope}%
\begin{pgfscope}%
\pgfpathrectangle{\pgfqpoint{1.565000in}{0.500000in}}{\pgfqpoint{3.020000in}{3.020000in}}%
\pgfusepath{clip}%
\pgfsetbuttcap%
\pgfsetroundjoin%
\definecolor{currentfill}{rgb}{0.996094,0.996094,0.996094}%
\pgfsetfillcolor{currentfill}%
\pgfsetlinewidth{0.000000pt}%
\definecolor{currentstroke}{rgb}{1.000000,1.000000,1.000000}%
\pgfsetstrokecolor{currentstroke}%
\pgfsetdash{}{0pt}%
\pgfpathmoveto{\pgfqpoint{3.075000in}{3.520000in}}%
\pgfpathlineto{\pgfqpoint{3.578333in}{3.520000in}}%
\pgfpathlineto{\pgfqpoint{3.578333in}{3.016667in}}%
\pgfpathlineto{\pgfqpoint{3.075000in}{3.016667in}}%
\pgfpathlineto{\pgfqpoint{3.075000in}{3.520000in}}%
\pgfusepath{fill}%
\end{pgfscope}%
\begin{pgfscope}%
\pgfpathrectangle{\pgfqpoint{1.565000in}{0.500000in}}{\pgfqpoint{3.020000in}{3.020000in}}%
\pgfusepath{clip}%
\pgfsetbuttcap%
\pgfsetroundjoin%
\definecolor{currentfill}{rgb}{0.996094,0.996094,0.996094}%
\pgfsetfillcolor{currentfill}%
\pgfsetlinewidth{0.000000pt}%
\definecolor{currentstroke}{rgb}{1.000000,1.000000,1.000000}%
\pgfsetstrokecolor{currentstroke}%
\pgfsetdash{}{0pt}%
\pgfpathmoveto{\pgfqpoint{3.578333in}{3.520000in}}%
\pgfpathlineto{\pgfqpoint{4.081667in}{3.520000in}}%
\pgfpathlineto{\pgfqpoint{4.081667in}{3.016667in}}%
\pgfpathlineto{\pgfqpoint{3.578333in}{3.016667in}}%
\pgfpathlineto{\pgfqpoint{3.578333in}{3.520000in}}%
\pgfusepath{fill}%
\end{pgfscope}%
\begin{pgfscope}%
\pgfpathrectangle{\pgfqpoint{1.565000in}{0.500000in}}{\pgfqpoint{3.020000in}{3.020000in}}%
\pgfusepath{clip}%
\pgfsetbuttcap%
\pgfsetroundjoin%
\definecolor{currentfill}{rgb}{0.996094,0.996094,0.996094}%
\pgfsetfillcolor{currentfill}%
\pgfsetlinewidth{0.000000pt}%
\definecolor{currentstroke}{rgb}{1.000000,1.000000,1.000000}%
\pgfsetstrokecolor{currentstroke}%
\pgfsetdash{}{0pt}%
\pgfpathmoveto{\pgfqpoint{4.081667in}{3.520000in}}%
\pgfpathlineto{\pgfqpoint{4.585000in}{3.520000in}}%
\pgfpathlineto{\pgfqpoint{4.585000in}{3.016667in}}%
\pgfpathlineto{\pgfqpoint{4.081667in}{3.016667in}}%
\pgfpathlineto{\pgfqpoint{4.081667in}{3.520000in}}%
\pgfusepath{fill}%
\end{pgfscope}%
\begin{pgfscope}%
\pgfpathrectangle{\pgfqpoint{1.565000in}{0.500000in}}{\pgfqpoint{3.020000in}{3.020000in}}%
\pgfusepath{clip}%
\pgfsetbuttcap%
\pgfsetroundjoin%
\definecolor{currentfill}{rgb}{0.973514,0.954473,0.965594}%
\pgfsetfillcolor{currentfill}%
\pgfsetlinewidth{0.000000pt}%
\definecolor{currentstroke}{rgb}{1.000000,1.000000,1.000000}%
\pgfsetstrokecolor{currentstroke}%
\pgfsetdash{}{0pt}%
\pgfpathmoveto{\pgfqpoint{1.565000in}{3.016667in}}%
\pgfpathlineto{\pgfqpoint{2.068333in}{3.016667in}}%
\pgfpathlineto{\pgfqpoint{2.068333in}{2.513333in}}%
\pgfpathlineto{\pgfqpoint{1.565000in}{2.513333in}}%
\pgfpathlineto{\pgfqpoint{1.565000in}{3.016667in}}%
\pgfusepath{fill}%
\end{pgfscope}%
\begin{pgfscope}%
\pgfpathrectangle{\pgfqpoint{1.565000in}{0.500000in}}{\pgfqpoint{3.020000in}{3.020000in}}%
\pgfusepath{clip}%
\pgfsetbuttcap%
\pgfsetroundjoin%
\definecolor{currentfill}{rgb}{0.575291,0.220435,0.427696}%
\pgfsetfillcolor{currentfill}%
\pgfsetlinewidth{0.000000pt}%
\definecolor{currentstroke}{rgb}{1.000000,1.000000,1.000000}%
\pgfsetstrokecolor{currentstroke}%
\pgfsetdash{}{0pt}%
\pgfpathmoveto{\pgfqpoint{2.068333in}{3.016667in}}%
\pgfpathlineto{\pgfqpoint{2.571667in}{3.016667in}}%
\pgfpathlineto{\pgfqpoint{2.571667in}{2.513333in}}%
\pgfpathlineto{\pgfqpoint{2.068333in}{2.513333in}}%
\pgfpathlineto{\pgfqpoint{2.068333in}{3.016667in}}%
\pgfusepath{fill}%
\end{pgfscope}%
\begin{pgfscope}%
\pgfpathrectangle{\pgfqpoint{1.565000in}{0.500000in}}{\pgfqpoint{3.020000in}{3.020000in}}%
\pgfusepath{clip}%
\pgfsetbuttcap%
\pgfsetroundjoin%
\definecolor{currentfill}{rgb}{0.893459,0.806909,0.857460}%
\pgfsetfillcolor{currentfill}%
\pgfsetlinewidth{0.000000pt}%
\definecolor{currentstroke}{rgb}{1.000000,1.000000,1.000000}%
\pgfsetstrokecolor{currentstroke}%
\pgfsetdash{}{0pt}%
\pgfpathmoveto{\pgfqpoint{2.571667in}{3.016667in}}%
\pgfpathlineto{\pgfqpoint{3.075000in}{3.016667in}}%
\pgfpathlineto{\pgfqpoint{3.075000in}{2.513333in}}%
\pgfpathlineto{\pgfqpoint{2.571667in}{2.513333in}}%
\pgfpathlineto{\pgfqpoint{2.571667in}{3.016667in}}%
\pgfusepath{fill}%
\end{pgfscope}%
\begin{pgfscope}%
\pgfpathrectangle{\pgfqpoint{1.565000in}{0.500000in}}{\pgfqpoint{3.020000in}{3.020000in}}%
\pgfusepath{clip}%
\pgfsetbuttcap%
\pgfsetroundjoin%
\definecolor{currentfill}{rgb}{0.987883,0.980959,0.985003}%
\pgfsetfillcolor{currentfill}%
\pgfsetlinewidth{0.000000pt}%
\definecolor{currentstroke}{rgb}{1.000000,1.000000,1.000000}%
\pgfsetstrokecolor{currentstroke}%
\pgfsetdash{}{0pt}%
\pgfpathmoveto{\pgfqpoint{3.075000in}{3.016667in}}%
\pgfpathlineto{\pgfqpoint{3.578333in}{3.016667in}}%
\pgfpathlineto{\pgfqpoint{3.578333in}{2.513333in}}%
\pgfpathlineto{\pgfqpoint{3.075000in}{2.513333in}}%
\pgfpathlineto{\pgfqpoint{3.075000in}{3.016667in}}%
\pgfusepath{fill}%
\end{pgfscope}%
\begin{pgfscope}%
\pgfpathrectangle{\pgfqpoint{1.565000in}{0.500000in}}{\pgfqpoint{3.020000in}{3.020000in}}%
\pgfusepath{clip}%
\pgfsetbuttcap%
\pgfsetroundjoin%
\definecolor{currentfill}{rgb}{0.989936,0.984743,0.987776}%
\pgfsetfillcolor{currentfill}%
\pgfsetlinewidth{0.000000pt}%
\definecolor{currentstroke}{rgb}{1.000000,1.000000,1.000000}%
\pgfsetstrokecolor{currentstroke}%
\pgfsetdash{}{0pt}%
\pgfpathmoveto{\pgfqpoint{3.578333in}{3.016667in}}%
\pgfpathlineto{\pgfqpoint{4.081667in}{3.016667in}}%
\pgfpathlineto{\pgfqpoint{4.081667in}{2.513333in}}%
\pgfpathlineto{\pgfqpoint{3.578333in}{2.513333in}}%
\pgfpathlineto{\pgfqpoint{3.578333in}{3.016667in}}%
\pgfusepath{fill}%
\end{pgfscope}%
\begin{pgfscope}%
\pgfpathrectangle{\pgfqpoint{1.565000in}{0.500000in}}{\pgfqpoint{3.020000in}{3.020000in}}%
\pgfusepath{clip}%
\pgfsetbuttcap%
\pgfsetroundjoin%
\definecolor{currentfill}{rgb}{0.996094,0.996094,0.996094}%
\pgfsetfillcolor{currentfill}%
\pgfsetlinewidth{0.000000pt}%
\definecolor{currentstroke}{rgb}{1.000000,1.000000,1.000000}%
\pgfsetstrokecolor{currentstroke}%
\pgfsetdash{}{0pt}%
\pgfpathmoveto{\pgfqpoint{4.081667in}{3.016667in}}%
\pgfpathlineto{\pgfqpoint{4.585000in}{3.016667in}}%
\pgfpathlineto{\pgfqpoint{4.585000in}{2.513333in}}%
\pgfpathlineto{\pgfqpoint{4.081667in}{2.513333in}}%
\pgfpathlineto{\pgfqpoint{4.081667in}{3.016667in}}%
\pgfusepath{fill}%
\end{pgfscope}%
\begin{pgfscope}%
\pgfpathrectangle{\pgfqpoint{1.565000in}{0.500000in}}{\pgfqpoint{3.020000in}{3.020000in}}%
\pgfusepath{clip}%
\pgfsetbuttcap%
\pgfsetroundjoin%
\definecolor{currentfill}{rgb}{0.991988,0.988526,0.990548}%
\pgfsetfillcolor{currentfill}%
\pgfsetlinewidth{0.000000pt}%
\definecolor{currentstroke}{rgb}{1.000000,1.000000,1.000000}%
\pgfsetstrokecolor{currentstroke}%
\pgfsetdash{}{0pt}%
\pgfpathmoveto{\pgfqpoint{1.565000in}{2.513333in}}%
\pgfpathlineto{\pgfqpoint{2.068333in}{2.513333in}}%
\pgfpathlineto{\pgfqpoint{2.068333in}{2.010000in}}%
\pgfpathlineto{\pgfqpoint{1.565000in}{2.010000in}}%
\pgfpathlineto{\pgfqpoint{1.565000in}{2.513333in}}%
\pgfusepath{fill}%
\end{pgfscope}%
\begin{pgfscope}%
\pgfpathrectangle{\pgfqpoint{1.565000in}{0.500000in}}{\pgfqpoint{3.020000in}{3.020000in}}%
\pgfusepath{clip}%
\pgfsetbuttcap%
\pgfsetroundjoin%
\definecolor{currentfill}{rgb}{0.874985,0.772855,0.832506}%
\pgfsetfillcolor{currentfill}%
\pgfsetlinewidth{0.000000pt}%
\definecolor{currentstroke}{rgb}{1.000000,1.000000,1.000000}%
\pgfsetstrokecolor{currentstroke}%
\pgfsetdash{}{0pt}%
\pgfpathmoveto{\pgfqpoint{2.068333in}{2.513333in}}%
\pgfpathlineto{\pgfqpoint{2.571667in}{2.513333in}}%
\pgfpathlineto{\pgfqpoint{2.571667in}{2.010000in}}%
\pgfpathlineto{\pgfqpoint{2.068333in}{2.010000in}}%
\pgfpathlineto{\pgfqpoint{2.068333in}{2.513333in}}%
\pgfusepath{fill}%
\end{pgfscope}%
\begin{pgfscope}%
\pgfpathrectangle{\pgfqpoint{1.565000in}{0.500000in}}{\pgfqpoint{3.020000in}{3.020000in}}%
\pgfusepath{clip}%
\pgfsetbuttcap%
\pgfsetroundjoin%
\definecolor{currentfill}{rgb}{0.704611,0.458808,0.602374}%
\pgfsetfillcolor{currentfill}%
\pgfsetlinewidth{0.000000pt}%
\definecolor{currentstroke}{rgb}{1.000000,1.000000,1.000000}%
\pgfsetstrokecolor{currentstroke}%
\pgfsetdash{}{0pt}%
\pgfpathmoveto{\pgfqpoint{2.571667in}{2.513333in}}%
\pgfpathlineto{\pgfqpoint{3.075000in}{2.513333in}}%
\pgfpathlineto{\pgfqpoint{3.075000in}{2.010000in}}%
\pgfpathlineto{\pgfqpoint{2.571667in}{2.010000in}}%
\pgfpathlineto{\pgfqpoint{2.571667in}{2.513333in}}%
\pgfusepath{fill}%
\end{pgfscope}%
\begin{pgfscope}%
\pgfpathrectangle{\pgfqpoint{1.565000in}{0.500000in}}{\pgfqpoint{3.020000in}{3.020000in}}%
\pgfusepath{clip}%
\pgfsetbuttcap%
\pgfsetroundjoin%
\definecolor{currentfill}{rgb}{0.874985,0.772855,0.832506}%
\pgfsetfillcolor{currentfill}%
\pgfsetlinewidth{0.000000pt}%
\definecolor{currentstroke}{rgb}{1.000000,1.000000,1.000000}%
\pgfsetstrokecolor{currentstroke}%
\pgfsetdash{}{0pt}%
\pgfpathmoveto{\pgfqpoint{3.075000in}{2.513333in}}%
\pgfpathlineto{\pgfqpoint{3.578333in}{2.513333in}}%
\pgfpathlineto{\pgfqpoint{3.578333in}{2.010000in}}%
\pgfpathlineto{\pgfqpoint{3.075000in}{2.010000in}}%
\pgfpathlineto{\pgfqpoint{3.075000in}{2.513333in}}%
\pgfusepath{fill}%
\end{pgfscope}%
\begin{pgfscope}%
\pgfpathrectangle{\pgfqpoint{1.565000in}{0.500000in}}{\pgfqpoint{3.020000in}{3.020000in}}%
\pgfusepath{clip}%
\pgfsetbuttcap%
\pgfsetroundjoin%
\definecolor{currentfill}{rgb}{0.973514,0.954473,0.965594}%
\pgfsetfillcolor{currentfill}%
\pgfsetlinewidth{0.000000pt}%
\definecolor{currentstroke}{rgb}{1.000000,1.000000,1.000000}%
\pgfsetstrokecolor{currentstroke}%
\pgfsetdash{}{0pt}%
\pgfpathmoveto{\pgfqpoint{3.578333in}{2.513333in}}%
\pgfpathlineto{\pgfqpoint{4.081667in}{2.513333in}}%
\pgfpathlineto{\pgfqpoint{4.081667in}{2.010000in}}%
\pgfpathlineto{\pgfqpoint{3.578333in}{2.010000in}}%
\pgfpathlineto{\pgfqpoint{3.578333in}{2.513333in}}%
\pgfusepath{fill}%
\end{pgfscope}%
\begin{pgfscope}%
\pgfpathrectangle{\pgfqpoint{1.565000in}{0.500000in}}{\pgfqpoint{3.020000in}{3.020000in}}%
\pgfusepath{clip}%
\pgfsetbuttcap%
\pgfsetroundjoin%
\definecolor{currentfill}{rgb}{0.996094,0.996094,0.996094}%
\pgfsetfillcolor{currentfill}%
\pgfsetlinewidth{0.000000pt}%
\definecolor{currentstroke}{rgb}{1.000000,1.000000,1.000000}%
\pgfsetstrokecolor{currentstroke}%
\pgfsetdash{}{0pt}%
\pgfpathmoveto{\pgfqpoint{4.081667in}{2.513333in}}%
\pgfpathlineto{\pgfqpoint{4.585000in}{2.513333in}}%
\pgfpathlineto{\pgfqpoint{4.585000in}{2.010000in}}%
\pgfpathlineto{\pgfqpoint{4.081667in}{2.010000in}}%
\pgfpathlineto{\pgfqpoint{4.081667in}{2.513333in}}%
\pgfusepath{fill}%
\end{pgfscope}%
\begin{pgfscope}%
\pgfpathrectangle{\pgfqpoint{1.565000in}{0.500000in}}{\pgfqpoint{3.020000in}{3.020000in}}%
\pgfusepath{clip}%
\pgfsetbuttcap%
\pgfsetroundjoin%
\definecolor{currentfill}{rgb}{0.989936,0.984743,0.987776}%
\pgfsetfillcolor{currentfill}%
\pgfsetlinewidth{0.000000pt}%
\definecolor{currentstroke}{rgb}{1.000000,1.000000,1.000000}%
\pgfsetstrokecolor{currentstroke}%
\pgfsetdash{}{0pt}%
\pgfpathmoveto{\pgfqpoint{1.565000in}{2.010000in}}%
\pgfpathlineto{\pgfqpoint{2.068333in}{2.010000in}}%
\pgfpathlineto{\pgfqpoint{2.068333in}{1.506667in}}%
\pgfpathlineto{\pgfqpoint{1.565000in}{1.506667in}}%
\pgfpathlineto{\pgfqpoint{1.565000in}{2.010000in}}%
\pgfusepath{fill}%
\end{pgfscope}%
\begin{pgfscope}%
\pgfpathrectangle{\pgfqpoint{1.565000in}{0.500000in}}{\pgfqpoint{3.020000in}{3.020000in}}%
\pgfusepath{clip}%
\pgfsetbuttcap%
\pgfsetroundjoin%
\definecolor{currentfill}{rgb}{0.996094,0.996094,0.996094}%
\pgfsetfillcolor{currentfill}%
\pgfsetlinewidth{0.000000pt}%
\definecolor{currentstroke}{rgb}{1.000000,1.000000,1.000000}%
\pgfsetstrokecolor{currentstroke}%
\pgfsetdash{}{0pt}%
\pgfpathmoveto{\pgfqpoint{2.068333in}{2.010000in}}%
\pgfpathlineto{\pgfqpoint{2.571667in}{2.010000in}}%
\pgfpathlineto{\pgfqpoint{2.571667in}{1.506667in}}%
\pgfpathlineto{\pgfqpoint{2.068333in}{1.506667in}}%
\pgfpathlineto{\pgfqpoint{2.068333in}{2.010000in}}%
\pgfusepath{fill}%
\end{pgfscope}%
\begin{pgfscope}%
\pgfpathrectangle{\pgfqpoint{1.565000in}{0.500000in}}{\pgfqpoint{3.020000in}{3.020000in}}%
\pgfusepath{clip}%
\pgfsetbuttcap%
\pgfsetroundjoin%
\definecolor{currentfill}{rgb}{0.942724,0.897718,0.924004}%
\pgfsetfillcolor{currentfill}%
\pgfsetlinewidth{0.000000pt}%
\definecolor{currentstroke}{rgb}{1.000000,1.000000,1.000000}%
\pgfsetstrokecolor{currentstroke}%
\pgfsetdash{}{0pt}%
\pgfpathmoveto{\pgfqpoint{2.571667in}{2.010000in}}%
\pgfpathlineto{\pgfqpoint{3.075000in}{2.010000in}}%
\pgfpathlineto{\pgfqpoint{3.075000in}{1.506667in}}%
\pgfpathlineto{\pgfqpoint{2.571667in}{1.506667in}}%
\pgfpathlineto{\pgfqpoint{2.571667in}{2.010000in}}%
\pgfusepath{fill}%
\end{pgfscope}%
\begin{pgfscope}%
\pgfpathrectangle{\pgfqpoint{1.565000in}{0.500000in}}{\pgfqpoint{3.020000in}{3.020000in}}%
\pgfusepath{clip}%
\pgfsetbuttcap%
\pgfsetroundjoin%
\definecolor{currentfill}{rgb}{0.690242,0.432322,0.582966}%
\pgfsetfillcolor{currentfill}%
\pgfsetlinewidth{0.000000pt}%
\definecolor{currentstroke}{rgb}{1.000000,1.000000,1.000000}%
\pgfsetstrokecolor{currentstroke}%
\pgfsetdash{}{0pt}%
\pgfpathmoveto{\pgfqpoint{3.075000in}{2.010000in}}%
\pgfpathlineto{\pgfqpoint{3.578333in}{2.010000in}}%
\pgfpathlineto{\pgfqpoint{3.578333in}{1.506667in}}%
\pgfpathlineto{\pgfqpoint{3.075000in}{1.506667in}}%
\pgfpathlineto{\pgfqpoint{3.075000in}{2.010000in}}%
\pgfusepath{fill}%
\end{pgfscope}%
\begin{pgfscope}%
\pgfpathrectangle{\pgfqpoint{1.565000in}{0.500000in}}{\pgfqpoint{3.020000in}{3.020000in}}%
\pgfusepath{clip}%
\pgfsetbuttcap%
\pgfsetroundjoin%
\definecolor{currentfill}{rgb}{0.819562,0.670695,0.757644}%
\pgfsetfillcolor{currentfill}%
\pgfsetlinewidth{0.000000pt}%
\definecolor{currentstroke}{rgb}{1.000000,1.000000,1.000000}%
\pgfsetstrokecolor{currentstroke}%
\pgfsetdash{}{0pt}%
\pgfpathmoveto{\pgfqpoint{3.578333in}{2.010000in}}%
\pgfpathlineto{\pgfqpoint{4.081667in}{2.010000in}}%
\pgfpathlineto{\pgfqpoint{4.081667in}{1.506667in}}%
\pgfpathlineto{\pgfqpoint{3.578333in}{1.506667in}}%
\pgfpathlineto{\pgfqpoint{3.578333in}{2.010000in}}%
\pgfusepath{fill}%
\end{pgfscope}%
\begin{pgfscope}%
\pgfpathrectangle{\pgfqpoint{1.565000in}{0.500000in}}{\pgfqpoint{3.020000in}{3.020000in}}%
\pgfusepath{clip}%
\pgfsetbuttcap%
\pgfsetroundjoin%
\definecolor{currentfill}{rgb}{0.977619,0.962040,0.971140}%
\pgfsetfillcolor{currentfill}%
\pgfsetlinewidth{0.000000pt}%
\definecolor{currentstroke}{rgb}{1.000000,1.000000,1.000000}%
\pgfsetstrokecolor{currentstroke}%
\pgfsetdash{}{0pt}%
\pgfpathmoveto{\pgfqpoint{4.081667in}{2.010000in}}%
\pgfpathlineto{\pgfqpoint{4.585000in}{2.010000in}}%
\pgfpathlineto{\pgfqpoint{4.585000in}{1.506667in}}%
\pgfpathlineto{\pgfqpoint{4.081667in}{1.506667in}}%
\pgfpathlineto{\pgfqpoint{4.081667in}{2.010000in}}%
\pgfusepath{fill}%
\end{pgfscope}%
\begin{pgfscope}%
\pgfpathrectangle{\pgfqpoint{1.565000in}{0.500000in}}{\pgfqpoint{3.020000in}{3.020000in}}%
\pgfusepath{clip}%
\pgfsetbuttcap%
\pgfsetroundjoin%
\definecolor{currentfill}{rgb}{0.996094,0.996094,0.996094}%
\pgfsetfillcolor{currentfill}%
\pgfsetlinewidth{0.000000pt}%
\definecolor{currentstroke}{rgb}{1.000000,1.000000,1.000000}%
\pgfsetstrokecolor{currentstroke}%
\pgfsetdash{}{0pt}%
\pgfpathmoveto{\pgfqpoint{1.565000in}{1.506667in}}%
\pgfpathlineto{\pgfqpoint{2.068333in}{1.506667in}}%
\pgfpathlineto{\pgfqpoint{2.068333in}{1.003333in}}%
\pgfpathlineto{\pgfqpoint{1.565000in}{1.003333in}}%
\pgfpathlineto{\pgfqpoint{1.565000in}{1.506667in}}%
\pgfusepath{fill}%
\end{pgfscope}%
\begin{pgfscope}%
\pgfpathrectangle{\pgfqpoint{1.565000in}{0.500000in}}{\pgfqpoint{3.020000in}{3.020000in}}%
\pgfusepath{clip}%
\pgfsetbuttcap%
\pgfsetroundjoin%
\definecolor{currentfill}{rgb}{0.983778,0.973392,0.979458}%
\pgfsetfillcolor{currentfill}%
\pgfsetlinewidth{0.000000pt}%
\definecolor{currentstroke}{rgb}{1.000000,1.000000,1.000000}%
\pgfsetstrokecolor{currentstroke}%
\pgfsetdash{}{0pt}%
\pgfpathmoveto{\pgfqpoint{2.068333in}{1.506667in}}%
\pgfpathlineto{\pgfqpoint{2.571667in}{1.506667in}}%
\pgfpathlineto{\pgfqpoint{2.571667in}{1.003333in}}%
\pgfpathlineto{\pgfqpoint{2.068333in}{1.003333in}}%
\pgfpathlineto{\pgfqpoint{2.068333in}{1.506667in}}%
\pgfusepath{fill}%
\end{pgfscope}%
\begin{pgfscope}%
\pgfpathrectangle{\pgfqpoint{1.565000in}{0.500000in}}{\pgfqpoint{3.020000in}{3.020000in}}%
\pgfusepath{clip}%
\pgfsetbuttcap%
\pgfsetroundjoin%
\definecolor{currentfill}{rgb}{0.932460,0.878799,0.910141}%
\pgfsetfillcolor{currentfill}%
\pgfsetlinewidth{0.000000pt}%
\definecolor{currentstroke}{rgb}{1.000000,1.000000,1.000000}%
\pgfsetstrokecolor{currentstroke}%
\pgfsetdash{}{0pt}%
\pgfpathmoveto{\pgfqpoint{2.571667in}{1.506667in}}%
\pgfpathlineto{\pgfqpoint{3.075000in}{1.506667in}}%
\pgfpathlineto{\pgfqpoint{3.075000in}{1.003333in}}%
\pgfpathlineto{\pgfqpoint{2.571667in}{1.003333in}}%
\pgfpathlineto{\pgfqpoint{2.571667in}{1.506667in}}%
\pgfusepath{fill}%
\end{pgfscope}%
\begin{pgfscope}%
\pgfpathrectangle{\pgfqpoint{1.565000in}{0.500000in}}{\pgfqpoint{3.020000in}{3.020000in}}%
\pgfusepath{clip}%
\pgfsetbuttcap%
\pgfsetroundjoin%
\definecolor{currentfill}{rgb}{0.944776,0.901501,0.926777}%
\pgfsetfillcolor{currentfill}%
\pgfsetlinewidth{0.000000pt}%
\definecolor{currentstroke}{rgb}{1.000000,1.000000,1.000000}%
\pgfsetstrokecolor{currentstroke}%
\pgfsetdash{}{0pt}%
\pgfpathmoveto{\pgfqpoint{3.075000in}{1.506667in}}%
\pgfpathlineto{\pgfqpoint{3.578333in}{1.506667in}}%
\pgfpathlineto{\pgfqpoint{3.578333in}{1.003333in}}%
\pgfpathlineto{\pgfqpoint{3.075000in}{1.003333in}}%
\pgfpathlineto{\pgfqpoint{3.075000in}{1.506667in}}%
\pgfusepath{fill}%
\end{pgfscope}%
\begin{pgfscope}%
\pgfpathrectangle{\pgfqpoint{1.565000in}{0.500000in}}{\pgfqpoint{3.020000in}{3.020000in}}%
\pgfusepath{clip}%
\pgfsetbuttcap%
\pgfsetroundjoin%
\definecolor{currentfill}{rgb}{0.626608,0.315028,0.497013}%
\pgfsetfillcolor{currentfill}%
\pgfsetlinewidth{0.000000pt}%
\definecolor{currentstroke}{rgb}{1.000000,1.000000,1.000000}%
\pgfsetstrokecolor{currentstroke}%
\pgfsetdash{}{0pt}%
\pgfpathmoveto{\pgfqpoint{3.578333in}{1.506667in}}%
\pgfpathlineto{\pgfqpoint{4.081667in}{1.506667in}}%
\pgfpathlineto{\pgfqpoint{4.081667in}{1.003333in}}%
\pgfpathlineto{\pgfqpoint{3.578333in}{1.003333in}}%
\pgfpathlineto{\pgfqpoint{3.578333in}{1.506667in}}%
\pgfusepath{fill}%
\end{pgfscope}%
\begin{pgfscope}%
\pgfpathrectangle{\pgfqpoint{1.565000in}{0.500000in}}{\pgfqpoint{3.020000in}{3.020000in}}%
\pgfusepath{clip}%
\pgfsetbuttcap%
\pgfsetroundjoin%
\definecolor{currentfill}{rgb}{0.932460,0.878799,0.910141}%
\pgfsetfillcolor{currentfill}%
\pgfsetlinewidth{0.000000pt}%
\definecolor{currentstroke}{rgb}{1.000000,1.000000,1.000000}%
\pgfsetstrokecolor{currentstroke}%
\pgfsetdash{}{0pt}%
\pgfpathmoveto{\pgfqpoint{4.081667in}{1.506667in}}%
\pgfpathlineto{\pgfqpoint{4.585000in}{1.506667in}}%
\pgfpathlineto{\pgfqpoint{4.585000in}{1.003333in}}%
\pgfpathlineto{\pgfqpoint{4.081667in}{1.003333in}}%
\pgfpathlineto{\pgfqpoint{4.081667in}{1.506667in}}%
\pgfusepath{fill}%
\end{pgfscope}%
\begin{pgfscope}%
\pgfpathrectangle{\pgfqpoint{1.565000in}{0.500000in}}{\pgfqpoint{3.020000in}{3.020000in}}%
\pgfusepath{clip}%
\pgfsetbuttcap%
\pgfsetroundjoin%
\definecolor{currentfill}{rgb}{0.985830,0.977175,0.982230}%
\pgfsetfillcolor{currentfill}%
\pgfsetlinewidth{0.000000pt}%
\definecolor{currentstroke}{rgb}{1.000000,1.000000,1.000000}%
\pgfsetstrokecolor{currentstroke}%
\pgfsetdash{}{0pt}%
\pgfpathmoveto{\pgfqpoint{1.565000in}{1.003333in}}%
\pgfpathlineto{\pgfqpoint{2.068333in}{1.003333in}}%
\pgfpathlineto{\pgfqpoint{2.068333in}{0.500000in}}%
\pgfpathlineto{\pgfqpoint{1.565000in}{0.500000in}}%
\pgfpathlineto{\pgfqpoint{1.565000in}{1.003333in}}%
\pgfusepath{fill}%
\end{pgfscope}%
\begin{pgfscope}%
\pgfpathrectangle{\pgfqpoint{1.565000in}{0.500000in}}{\pgfqpoint{3.020000in}{3.020000in}}%
\pgfusepath{clip}%
\pgfsetbuttcap%
\pgfsetroundjoin%
\definecolor{currentfill}{rgb}{0.996094,0.996094,0.996094}%
\pgfsetfillcolor{currentfill}%
\pgfsetlinewidth{0.000000pt}%
\definecolor{currentstroke}{rgb}{1.000000,1.000000,1.000000}%
\pgfsetstrokecolor{currentstroke}%
\pgfsetdash{}{0pt}%
\pgfpathmoveto{\pgfqpoint{2.068333in}{1.003333in}}%
\pgfpathlineto{\pgfqpoint{2.571667in}{1.003333in}}%
\pgfpathlineto{\pgfqpoint{2.571667in}{0.500000in}}%
\pgfpathlineto{\pgfqpoint{2.068333in}{0.500000in}}%
\pgfpathlineto{\pgfqpoint{2.068333in}{1.003333in}}%
\pgfusepath{fill}%
\end{pgfscope}%
\begin{pgfscope}%
\pgfpathrectangle{\pgfqpoint{1.565000in}{0.500000in}}{\pgfqpoint{3.020000in}{3.020000in}}%
\pgfusepath{clip}%
\pgfsetbuttcap%
\pgfsetroundjoin%
\definecolor{currentfill}{rgb}{0.959145,0.927987,0.946186}%
\pgfsetfillcolor{currentfill}%
\pgfsetlinewidth{0.000000pt}%
\definecolor{currentstroke}{rgb}{1.000000,1.000000,1.000000}%
\pgfsetstrokecolor{currentstroke}%
\pgfsetdash{}{0pt}%
\pgfpathmoveto{\pgfqpoint{2.571667in}{1.003333in}}%
\pgfpathlineto{\pgfqpoint{3.075000in}{1.003333in}}%
\pgfpathlineto{\pgfqpoint{3.075000in}{0.500000in}}%
\pgfpathlineto{\pgfqpoint{2.571667in}{0.500000in}}%
\pgfpathlineto{\pgfqpoint{2.571667in}{1.003333in}}%
\pgfusepath{fill}%
\end{pgfscope}%
\begin{pgfscope}%
\pgfpathrectangle{\pgfqpoint{1.565000in}{0.500000in}}{\pgfqpoint{3.020000in}{3.020000in}}%
\pgfusepath{clip}%
\pgfsetbuttcap%
\pgfsetroundjoin%
\definecolor{currentfill}{rgb}{0.965303,0.939338,0.954504}%
\pgfsetfillcolor{currentfill}%
\pgfsetlinewidth{0.000000pt}%
\definecolor{currentstroke}{rgb}{1.000000,1.000000,1.000000}%
\pgfsetstrokecolor{currentstroke}%
\pgfsetdash{}{0pt}%
\pgfpathmoveto{\pgfqpoint{3.075000in}{1.003333in}}%
\pgfpathlineto{\pgfqpoint{3.578333in}{1.003333in}}%
\pgfpathlineto{\pgfqpoint{3.578333in}{0.500000in}}%
\pgfpathlineto{\pgfqpoint{3.075000in}{0.500000in}}%
\pgfpathlineto{\pgfqpoint{3.075000in}{1.003333in}}%
\pgfusepath{fill}%
\end{pgfscope}%
\begin{pgfscope}%
\pgfpathrectangle{\pgfqpoint{1.565000in}{0.500000in}}{\pgfqpoint{3.020000in}{3.020000in}}%
\pgfusepath{clip}%
\pgfsetbuttcap%
\pgfsetroundjoin%
\definecolor{currentfill}{rgb}{0.799035,0.632858,0.729917}%
\pgfsetfillcolor{currentfill}%
\pgfsetlinewidth{0.000000pt}%
\definecolor{currentstroke}{rgb}{1.000000,1.000000,1.000000}%
\pgfsetstrokecolor{currentstroke}%
\pgfsetdash{}{0pt}%
\pgfpathmoveto{\pgfqpoint{3.578333in}{1.003333in}}%
\pgfpathlineto{\pgfqpoint{4.081667in}{1.003333in}}%
\pgfpathlineto{\pgfqpoint{4.081667in}{0.500000in}}%
\pgfpathlineto{\pgfqpoint{3.578333in}{0.500000in}}%
\pgfpathlineto{\pgfqpoint{3.578333in}{1.003333in}}%
\pgfusepath{fill}%
\end{pgfscope}%
\begin{pgfscope}%
\pgfpathrectangle{\pgfqpoint{1.565000in}{0.500000in}}{\pgfqpoint{3.020000in}{3.020000in}}%
\pgfusepath{clip}%
\pgfsetbuttcap%
\pgfsetroundjoin%
\definecolor{currentfill}{rgb}{0.712822,0.473943,0.613465}%
\pgfsetfillcolor{currentfill}%
\pgfsetlinewidth{0.000000pt}%
\definecolor{currentstroke}{rgb}{1.000000,1.000000,1.000000}%
\pgfsetstrokecolor{currentstroke}%
\pgfsetdash{}{0pt}%
\pgfpathmoveto{\pgfqpoint{4.081667in}{1.003333in}}%
\pgfpathlineto{\pgfqpoint{4.585000in}{1.003333in}}%
\pgfpathlineto{\pgfqpoint{4.585000in}{0.500000in}}%
\pgfpathlineto{\pgfqpoint{4.081667in}{0.500000in}}%
\pgfpathlineto{\pgfqpoint{4.081667in}{1.003333in}}%
\pgfusepath{fill}%
\end{pgfscope}%
\begin{pgfscope}%
\definecolor{textcolor}{rgb}{0.000000,0.000000,0.000000}%
\pgfsetstrokecolor{textcolor}%
\pgfsetfillcolor{textcolor}%
\pgftext[x=1.816667in,y=0.402778in,,top]{\color{textcolor}\rmfamily\fontsize{10.000000}{12.000000}\selectfont 0}%
\end{pgfscope}%
\begin{pgfscope}%
\definecolor{textcolor}{rgb}{0.000000,0.000000,0.000000}%
\pgfsetstrokecolor{textcolor}%
\pgfsetfillcolor{textcolor}%
\pgftext[x=2.320000in,y=0.402778in,,top]{\color{textcolor}\rmfamily\fontsize{10.000000}{12.000000}\selectfont 1}%
\end{pgfscope}%
\begin{pgfscope}%
\definecolor{textcolor}{rgb}{0.000000,0.000000,0.000000}%
\pgfsetstrokecolor{textcolor}%
\pgfsetfillcolor{textcolor}%
\pgftext[x=2.823333in,y=0.402778in,,top]{\color{textcolor}\rmfamily\fontsize{10.000000}{12.000000}\selectfont 2}%
\end{pgfscope}%
\begin{pgfscope}%
\definecolor{textcolor}{rgb}{0.000000,0.000000,0.000000}%
\pgfsetstrokecolor{textcolor}%
\pgfsetfillcolor{textcolor}%
\pgftext[x=3.326667in,y=0.402778in,,top]{\color{textcolor}\rmfamily\fontsize{10.000000}{12.000000}\selectfont 3}%
\end{pgfscope}%
\begin{pgfscope}%
\definecolor{textcolor}{rgb}{0.000000,0.000000,0.000000}%
\pgfsetstrokecolor{textcolor}%
\pgfsetfillcolor{textcolor}%
\pgftext[x=3.830000in,y=0.402778in,,top]{\color{textcolor}\rmfamily\fontsize{10.000000}{12.000000}\selectfont 4}%
\end{pgfscope}%
\begin{pgfscope}%
\definecolor{textcolor}{rgb}{0.000000,0.000000,0.000000}%
\pgfsetstrokecolor{textcolor}%
\pgfsetfillcolor{textcolor}%
\pgftext[x=4.333333in,y=0.402778in,,top]{\color{textcolor}\rmfamily\fontsize{10.000000}{12.000000}\selectfont 5}%
\end{pgfscope}%
\begin{pgfscope}%
\definecolor{textcolor}{rgb}{0.000000,0.000000,0.000000}%
\pgfsetstrokecolor{textcolor}%
\pgfsetfillcolor{textcolor}%
\pgftext[x=3.075000in,y=0.195988in,,top]{\color{textcolor}\rmfamily\fontsize{15.000000}{18.000000}\selectfont Predicted ISUP grade}%
\end{pgfscope}%
\begin{pgfscope}%
\definecolor{textcolor}{rgb}{0.000000,0.000000,0.000000}%
\pgfsetstrokecolor{textcolor}%
\pgfsetfillcolor{textcolor}%
\pgftext[x=1.440772in, y=3.247114in, left, base,rotate=90.000000]{\color{textcolor}\rmfamily\fontsize{10.000000}{12.000000}\selectfont 0}%
\end{pgfscope}%
\begin{pgfscope}%
\definecolor{textcolor}{rgb}{0.000000,0.000000,0.000000}%
\pgfsetstrokecolor{textcolor}%
\pgfsetfillcolor{textcolor}%
\pgftext[x=1.440772in, y=2.743781in, left, base,rotate=90.000000]{\color{textcolor}\rmfamily\fontsize{10.000000}{12.000000}\selectfont 1}%
\end{pgfscope}%
\begin{pgfscope}%
\definecolor{textcolor}{rgb}{0.000000,0.000000,0.000000}%
\pgfsetstrokecolor{textcolor}%
\pgfsetfillcolor{textcolor}%
\pgftext[x=1.440772in, y=2.240447in, left, base,rotate=90.000000]{\color{textcolor}\rmfamily\fontsize{10.000000}{12.000000}\selectfont 2}%
\end{pgfscope}%
\begin{pgfscope}%
\definecolor{textcolor}{rgb}{0.000000,0.000000,0.000000}%
\pgfsetstrokecolor{textcolor}%
\pgfsetfillcolor{textcolor}%
\pgftext[x=1.440772in, y=1.737114in, left, base,rotate=90.000000]{\color{textcolor}\rmfamily\fontsize{10.000000}{12.000000}\selectfont 3}%
\end{pgfscope}%
\begin{pgfscope}%
\definecolor{textcolor}{rgb}{0.000000,0.000000,0.000000}%
\pgfsetstrokecolor{textcolor}%
\pgfsetfillcolor{textcolor}%
\pgftext[x=1.440772in, y=1.233781in, left, base,rotate=90.000000]{\color{textcolor}\rmfamily\fontsize{10.000000}{12.000000}\selectfont 4}%
\end{pgfscope}%
\begin{pgfscope}%
\definecolor{textcolor}{rgb}{0.000000,0.000000,0.000000}%
\pgfsetstrokecolor{textcolor}%
\pgfsetfillcolor{textcolor}%
\pgftext[x=1.440772in, y=0.730447in, left, base,rotate=90.000000]{\color{textcolor}\rmfamily\fontsize{10.000000}{12.000000}\selectfont 5}%
\end{pgfscope}%
\begin{pgfscope}%
\definecolor{textcolor}{rgb}{0.000000,0.000000,0.000000}%
\pgfsetstrokecolor{textcolor}%
\pgfsetfillcolor{textcolor}%
\pgftext[x=1.260988in,y=2.010000in,,bottom,rotate=90.000000]{\color{textcolor}\rmfamily\fontsize{15.000000}{18.000000}\selectfont Ground truth ISUP grade}%
\end{pgfscope}%
\begin{pgfscope}%
\pgfsetrectcap%
\pgfsetmiterjoin%
\pgfsetlinewidth{0.803000pt}%
\definecolor{currentstroke}{rgb}{0.000000,0.000000,0.000000}%
\pgfsetstrokecolor{currentstroke}%
\pgfsetdash{}{0pt}%
\pgfpathmoveto{\pgfqpoint{1.565000in}{0.500000in}}%
\pgfpathlineto{\pgfqpoint{1.565000in}{3.520000in}}%
\pgfusepath{stroke}%
\end{pgfscope}%
\begin{pgfscope}%
\pgfsetrectcap%
\pgfsetmiterjoin%
\pgfsetlinewidth{0.803000pt}%
\definecolor{currentstroke}{rgb}{0.000000,0.000000,0.000000}%
\pgfsetstrokecolor{currentstroke}%
\pgfsetdash{}{0pt}%
\pgfpathmoveto{\pgfqpoint{4.585000in}{0.500000in}}%
\pgfpathlineto{\pgfqpoint{4.585000in}{3.520000in}}%
\pgfusepath{stroke}%
\end{pgfscope}%
\begin{pgfscope}%
\pgfsetrectcap%
\pgfsetmiterjoin%
\pgfsetlinewidth{0.803000pt}%
\definecolor{currentstroke}{rgb}{0.000000,0.000000,0.000000}%
\pgfsetstrokecolor{currentstroke}%
\pgfsetdash{}{0pt}%
\pgfpathmoveto{\pgfqpoint{1.565000in}{0.500000in}}%
\pgfpathlineto{\pgfqpoint{4.585000in}{0.500000in}}%
\pgfusepath{stroke}%
\end{pgfscope}%
\begin{pgfscope}%
\pgfsetrectcap%
\pgfsetmiterjoin%
\pgfsetlinewidth{0.803000pt}%
\definecolor{currentstroke}{rgb}{0.000000,0.000000,0.000000}%
\pgfsetstrokecolor{currentstroke}%
\pgfsetdash{}{0pt}%
\pgfpathmoveto{\pgfqpoint{1.565000in}{3.520000in}}%
\pgfpathlineto{\pgfqpoint{4.585000in}{3.520000in}}%
\pgfusepath{stroke}%
\end{pgfscope}%
\begin{pgfscope}%
\definecolor{textcolor}{rgb}{1.000000,1.000000,1.000000}%
\pgfsetstrokecolor{textcolor}%
\pgfsetfillcolor{textcolor}%
\pgftext[x=1.816667in,y=3.268333in,,]{\color{textcolor}\rmfamily\fontsize{10.000000}{12.000000}\selectfont 93\%}%
\end{pgfscope}%
\begin{pgfscope}%
\definecolor{textcolor}{rgb}{0.150000,0.150000,0.150000}%
\pgfsetstrokecolor{textcolor}%
\pgfsetfillcolor{textcolor}%
\pgftext[x=2.320000in,y=3.268333in,,]{\color{textcolor}\rmfamily\fontsize{10.000000}{12.000000}\selectfont 6\%}%
\end{pgfscope}%
\begin{pgfscope}%
\definecolor{textcolor}{rgb}{0.150000,0.150000,0.150000}%
\pgfsetstrokecolor{textcolor}%
\pgfsetfillcolor{textcolor}%
\pgftext[x=2.823333in,y=3.268333in,,]{\color{textcolor}\rmfamily\fontsize{10.000000}{12.000000}\selectfont 1\%}%
\end{pgfscope}%
\begin{pgfscope}%
\definecolor{textcolor}{rgb}{0.150000,0.150000,0.150000}%
\pgfsetstrokecolor{textcolor}%
\pgfsetfillcolor{textcolor}%
\pgftext[x=3.326667in,y=3.268333in,,]{\color{textcolor}\rmfamily\fontsize{10.000000}{12.000000}\selectfont 0\%}%
\end{pgfscope}%
\begin{pgfscope}%
\definecolor{textcolor}{rgb}{0.150000,0.150000,0.150000}%
\pgfsetstrokecolor{textcolor}%
\pgfsetfillcolor{textcolor}%
\pgftext[x=3.830000in,y=3.268333in,,]{\color{textcolor}\rmfamily\fontsize{10.000000}{12.000000}\selectfont 0\%}%
\end{pgfscope}%
\begin{pgfscope}%
\definecolor{textcolor}{rgb}{0.150000,0.150000,0.150000}%
\pgfsetstrokecolor{textcolor}%
\pgfsetfillcolor{textcolor}%
\pgftext[x=4.333333in,y=3.268333in,,]{\color{textcolor}\rmfamily\fontsize{10.000000}{12.000000}\selectfont 0\%}%
\end{pgfscope}%
\begin{pgfscope}%
\definecolor{textcolor}{rgb}{0.150000,0.150000,0.150000}%
\pgfsetstrokecolor{textcolor}%
\pgfsetfillcolor{textcolor}%
\pgftext[x=1.816667in,y=2.765000in,,]{\color{textcolor}\rmfamily\fontsize{10.000000}{12.000000}\selectfont 4\%}%
\end{pgfscope}%
\begin{pgfscope}%
\definecolor{textcolor}{rgb}{1.000000,1.000000,1.000000}%
\pgfsetstrokecolor{textcolor}%
\pgfsetfillcolor{textcolor}%
\pgftext[x=2.320000in,y=2.765000in,,]{\color{textcolor}\rmfamily\fontsize{10.000000}{12.000000}\selectfont 75\%}%
\end{pgfscope}%
\begin{pgfscope}%
\definecolor{textcolor}{rgb}{0.150000,0.150000,0.150000}%
\pgfsetstrokecolor{textcolor}%
\pgfsetfillcolor{textcolor}%
\pgftext[x=2.823333in,y=2.765000in,,]{\color{textcolor}\rmfamily\fontsize{10.000000}{12.000000}\selectfont 18\%}%
\end{pgfscope}%
\begin{pgfscope}%
\definecolor{textcolor}{rgb}{0.150000,0.150000,0.150000}%
\pgfsetstrokecolor{textcolor}%
\pgfsetfillcolor{textcolor}%
\pgftext[x=3.326667in,y=2.765000in,,]{\color{textcolor}\rmfamily\fontsize{10.000000}{12.000000}\selectfont 2\%}%
\end{pgfscope}%
\begin{pgfscope}%
\definecolor{textcolor}{rgb}{0.150000,0.150000,0.150000}%
\pgfsetstrokecolor{textcolor}%
\pgfsetfillcolor{textcolor}%
\pgftext[x=3.830000in,y=2.765000in,,]{\color{textcolor}\rmfamily\fontsize{10.000000}{12.000000}\selectfont 1\%}%
\end{pgfscope}%
\begin{pgfscope}%
\definecolor{textcolor}{rgb}{0.150000,0.150000,0.150000}%
\pgfsetstrokecolor{textcolor}%
\pgfsetfillcolor{textcolor}%
\pgftext[x=4.333333in,y=2.765000in,,]{\color{textcolor}\rmfamily\fontsize{10.000000}{12.000000}\selectfont 0\%}%
\end{pgfscope}%
\begin{pgfscope}%
\definecolor{textcolor}{rgb}{0.150000,0.150000,0.150000}%
\pgfsetstrokecolor{textcolor}%
\pgfsetfillcolor{textcolor}%
\pgftext[x=1.816667in,y=2.261667in,,]{\color{textcolor}\rmfamily\fontsize{10.000000}{12.000000}\selectfont 1\%}%
\end{pgfscope}%
\begin{pgfscope}%
\definecolor{textcolor}{rgb}{0.150000,0.150000,0.150000}%
\pgfsetstrokecolor{textcolor}%
\pgfsetfillcolor{textcolor}%
\pgftext[x=2.320000in,y=2.261667in,,]{\color{textcolor}\rmfamily\fontsize{10.000000}{12.000000}\selectfont 22\%}%
\end{pgfscope}%
\begin{pgfscope}%
\definecolor{textcolor}{rgb}{1.000000,1.000000,1.000000}%
\pgfsetstrokecolor{textcolor}%
\pgfsetfillcolor{textcolor}%
\pgftext[x=2.823333in,y=2.261667in,,]{\color{textcolor}\rmfamily\fontsize{10.000000}{12.000000}\selectfont 52\%}%
\end{pgfscope}%
\begin{pgfscope}%
\definecolor{textcolor}{rgb}{0.150000,0.150000,0.150000}%
\pgfsetstrokecolor{textcolor}%
\pgfsetfillcolor{textcolor}%
\pgftext[x=3.326667in,y=2.261667in,,]{\color{textcolor}\rmfamily\fontsize{10.000000}{12.000000}\selectfont 22\%}%
\end{pgfscope}%
\begin{pgfscope}%
\definecolor{textcolor}{rgb}{0.150000,0.150000,0.150000}%
\pgfsetstrokecolor{textcolor}%
\pgfsetfillcolor{textcolor}%
\pgftext[x=3.830000in,y=2.261667in,,]{\color{textcolor}\rmfamily\fontsize{10.000000}{12.000000}\selectfont 4\%}%
\end{pgfscope}%
\begin{pgfscope}%
\definecolor{textcolor}{rgb}{0.150000,0.150000,0.150000}%
\pgfsetstrokecolor{textcolor}%
\pgfsetfillcolor{textcolor}%
\pgftext[x=4.333333in,y=2.261667in,,]{\color{textcolor}\rmfamily\fontsize{10.000000}{12.000000}\selectfont 0\%}%
\end{pgfscope}%
\begin{pgfscope}%
\definecolor{textcolor}{rgb}{0.150000,0.150000,0.150000}%
\pgfsetstrokecolor{textcolor}%
\pgfsetfillcolor{textcolor}%
\pgftext[x=1.816667in,y=1.758333in,,]{\color{textcolor}\rmfamily\fontsize{10.000000}{12.000000}\selectfont 1\%}%
\end{pgfscope}%
\begin{pgfscope}%
\definecolor{textcolor}{rgb}{0.150000,0.150000,0.150000}%
\pgfsetstrokecolor{textcolor}%
\pgfsetfillcolor{textcolor}%
\pgftext[x=2.320000in,y=1.758333in,,]{\color{textcolor}\rmfamily\fontsize{10.000000}{12.000000}\selectfont 0\%}%
\end{pgfscope}%
\begin{pgfscope}%
\definecolor{textcolor}{rgb}{0.150000,0.150000,0.150000}%
\pgfsetstrokecolor{textcolor}%
\pgfsetfillcolor{textcolor}%
\pgftext[x=2.823333in,y=1.758333in,,]{\color{textcolor}\rmfamily\fontsize{10.000000}{12.000000}\selectfont 10\%}%
\end{pgfscope}%
\begin{pgfscope}%
\definecolor{textcolor}{rgb}{1.000000,1.000000,1.000000}%
\pgfsetstrokecolor{textcolor}%
\pgfsetfillcolor{textcolor}%
\pgftext[x=3.326667in,y=1.758333in,,]{\color{textcolor}\rmfamily\fontsize{10.000000}{12.000000}\selectfont 54\%}%
\end{pgfscope}%
\begin{pgfscope}%
\definecolor{textcolor}{rgb}{0.150000,0.150000,0.150000}%
\pgfsetstrokecolor{textcolor}%
\pgfsetfillcolor{textcolor}%
\pgftext[x=3.830000in,y=1.758333in,,]{\color{textcolor}\rmfamily\fontsize{10.000000}{12.000000}\selectfont 31\%}%
\end{pgfscope}%
\begin{pgfscope}%
\definecolor{textcolor}{rgb}{0.150000,0.150000,0.150000}%
\pgfsetstrokecolor{textcolor}%
\pgfsetfillcolor{textcolor}%
\pgftext[x=4.333333in,y=1.758333in,,]{\color{textcolor}\rmfamily\fontsize{10.000000}{12.000000}\selectfont 4\%}%
\end{pgfscope}%
\begin{pgfscope}%
\definecolor{textcolor}{rgb}{0.150000,0.150000,0.150000}%
\pgfsetstrokecolor{textcolor}%
\pgfsetfillcolor{textcolor}%
\pgftext[x=1.816667in,y=1.255000in,,]{\color{textcolor}\rmfamily\fontsize{10.000000}{12.000000}\selectfont 0\%}%
\end{pgfscope}%
\begin{pgfscope}%
\definecolor{textcolor}{rgb}{0.150000,0.150000,0.150000}%
\pgfsetstrokecolor{textcolor}%
\pgfsetfillcolor{textcolor}%
\pgftext[x=2.320000in,y=1.255000in,,]{\color{textcolor}\rmfamily\fontsize{10.000000}{12.000000}\selectfont 2\%}%
\end{pgfscope}%
\begin{pgfscope}%
\definecolor{textcolor}{rgb}{0.150000,0.150000,0.150000}%
\pgfsetstrokecolor{textcolor}%
\pgfsetfillcolor{textcolor}%
\pgftext[x=2.823333in,y=1.255000in,,]{\color{textcolor}\rmfamily\fontsize{10.000000}{12.000000}\selectfont 11\%}%
\end{pgfscope}%
\begin{pgfscope}%
\definecolor{textcolor}{rgb}{0.150000,0.150000,0.150000}%
\pgfsetstrokecolor{textcolor}%
\pgfsetfillcolor{textcolor}%
\pgftext[x=3.326667in,y=1.255000in,,]{\color{textcolor}\rmfamily\fontsize{10.000000}{12.000000}\selectfont 9\%}%
\end{pgfscope}%
\begin{pgfscope}%
\definecolor{textcolor}{rgb}{1.000000,1.000000,1.000000}%
\pgfsetstrokecolor{textcolor}%
\pgfsetfillcolor{textcolor}%
\pgftext[x=3.830000in,y=1.255000in,,]{\color{textcolor}\rmfamily\fontsize{10.000000}{12.000000}\selectfont 66\%}%
\end{pgfscope}%
\begin{pgfscope}%
\definecolor{textcolor}{rgb}{0.150000,0.150000,0.150000}%
\pgfsetstrokecolor{textcolor}%
\pgfsetfillcolor{textcolor}%
\pgftext[x=4.333333in,y=1.255000in,,]{\color{textcolor}\rmfamily\fontsize{10.000000}{12.000000}\selectfont 11\%}%
\end{pgfscope}%
\begin{pgfscope}%
\definecolor{textcolor}{rgb}{0.150000,0.150000,0.150000}%
\pgfsetstrokecolor{textcolor}%
\pgfsetfillcolor{textcolor}%
\pgftext[x=1.816667in,y=0.751667in,,]{\color{textcolor}\rmfamily\fontsize{10.000000}{12.000000}\selectfont 2\%}%
\end{pgfscope}%
\begin{pgfscope}%
\definecolor{textcolor}{rgb}{0.150000,0.150000,0.150000}%
\pgfsetstrokecolor{textcolor}%
\pgfsetfillcolor{textcolor}%
\pgftext[x=2.320000in,y=0.751667in,,]{\color{textcolor}\rmfamily\fontsize{10.000000}{12.000000}\selectfont 0\%}%
\end{pgfscope}%
\begin{pgfscope}%
\definecolor{textcolor}{rgb}{0.150000,0.150000,0.150000}%
\pgfsetstrokecolor{textcolor}%
\pgfsetfillcolor{textcolor}%
\pgftext[x=2.823333in,y=0.751667in,,]{\color{textcolor}\rmfamily\fontsize{10.000000}{12.000000}\selectfont 7\%}%
\end{pgfscope}%
\begin{pgfscope}%
\definecolor{textcolor}{rgb}{0.150000,0.150000,0.150000}%
\pgfsetstrokecolor{textcolor}%
\pgfsetfillcolor{textcolor}%
\pgftext[x=3.326667in,y=0.751667in,,]{\color{textcolor}\rmfamily\fontsize{10.000000}{12.000000}\selectfont 6\%}%
\end{pgfscope}%
\begin{pgfscope}%
\definecolor{textcolor}{rgb}{0.150000,0.150000,0.150000}%
\pgfsetstrokecolor{textcolor}%
\pgfsetfillcolor{textcolor}%
\pgftext[x=3.830000in,y=0.751667in,,]{\color{textcolor}\rmfamily\fontsize{10.000000}{12.000000}\selectfont 35\%}%
\end{pgfscope}%
\begin{pgfscope}%
\definecolor{textcolor}{rgb}{1.000000,1.000000,1.000000}%
\pgfsetstrokecolor{textcolor}%
\pgfsetfillcolor{textcolor}%
\pgftext[x=4.333333in,y=0.751667in,,]{\color{textcolor}\rmfamily\fontsize{10.000000}{12.000000}\selectfont 50\%}%
\end{pgfscope}%
\begin{pgfscope}%
\definecolor{textcolor}{rgb}{0.000000,0.000000,0.000000}%
\pgfsetstrokecolor{textcolor}%
\pgfsetfillcolor{textcolor}%
\pgftext[x=3.075000in,y=3.686667in,,base]{\color{textcolor}\rmfamily\fontsize{18.000000}{21.600000}\selectfont Internal test set}%
\end{pgfscope}%
\end{pgfpicture}%
\makeatother%
\endgroup%
}
    \end{subfigure}%
    \caption{}
  \end{subfigure}%
  \begin{subfigure}[b]{.5\linewidth}
    \centering
    \begin{subfigure}[b]{.5\linewidth}
      \centering
      \resizebox{1.5\textwidth}{!}{%% Creator: Matplotlib, PGF backend
%%
%% To include the figure in your LaTeX document, write
%%   \input{<filename>.pgf}
%%
%% Make sure the required packages are loaded in your preamble
%%   \usepackage{pgf}
%%
%% Figures using additional raster images can only be included by \input if
%% they are in the same directory as the main LaTeX file. For loading figures
%% from other directories you can use the `import` package
%%   \usepackage{import}
%%
%% and then include the figures with
%%   \import{<path to file>}{<filename>.pgf}
%%
%% Matplotlib used the following preamble
%%
\begingroup%
\makeatletter%
\begin{pgfpicture}%
\pgfpathrectangle{\pgfpointorigin}{\pgfqpoint{6.000000in}{4.000000in}}%
\pgfusepath{use as bounding box, clip}%
\begin{pgfscope}%
\pgfsetbuttcap%
\pgfsetmiterjoin%
\pgfsetlinewidth{0.000000pt}%
\definecolor{currentstroke}{rgb}{1.000000,1.000000,1.000000}%
\pgfsetstrokecolor{currentstroke}%
\pgfsetstrokeopacity{0.000000}%
\pgfsetdash{}{0pt}%
\pgfpathmoveto{\pgfqpoint{0.000000in}{0.000000in}}%
\pgfpathlineto{\pgfqpoint{6.000000in}{0.000000in}}%
\pgfpathlineto{\pgfqpoint{6.000000in}{4.000000in}}%
\pgfpathlineto{\pgfqpoint{0.000000in}{4.000000in}}%
\pgfpathclose%
\pgfusepath{}%
\end{pgfscope}%
\begin{pgfscope}%
\pgfsetbuttcap%
\pgfsetmiterjoin%
\definecolor{currentfill}{rgb}{1.000000,1.000000,1.000000}%
\pgfsetfillcolor{currentfill}%
\pgfsetlinewidth{0.000000pt}%
\definecolor{currentstroke}{rgb}{0.000000,0.000000,0.000000}%
\pgfsetstrokecolor{currentstroke}%
\pgfsetstrokeopacity{0.000000}%
\pgfsetdash{}{0pt}%
\pgfpathmoveto{\pgfqpoint{1.565000in}{0.500000in}}%
\pgfpathlineto{\pgfqpoint{4.585000in}{0.500000in}}%
\pgfpathlineto{\pgfqpoint{4.585000in}{3.520000in}}%
\pgfpathlineto{\pgfqpoint{1.565000in}{3.520000in}}%
\pgfpathclose%
\pgfusepath{fill}%
\end{pgfscope}%
\begin{pgfscope}%
\pgfpathrectangle{\pgfqpoint{1.565000in}{0.500000in}}{\pgfqpoint{3.020000in}{3.020000in}}%
\pgfusepath{clip}%
\pgfsetbuttcap%
\pgfsetroundjoin%
\definecolor{currentfill}{rgb}{0.472656,0.031250,0.289062}%
\pgfsetfillcolor{currentfill}%
\pgfsetlinewidth{0.000000pt}%
\definecolor{currentstroke}{rgb}{1.000000,1.000000,1.000000}%
\pgfsetstrokecolor{currentstroke}%
\pgfsetdash{}{0pt}%
\pgfpathmoveto{\pgfqpoint{1.565000in}{3.520000in}}%
\pgfpathlineto{\pgfqpoint{2.068333in}{3.520000in}}%
\pgfpathlineto{\pgfqpoint{2.068333in}{3.016667in}}%
\pgfpathlineto{\pgfqpoint{1.565000in}{3.016667in}}%
\pgfpathlineto{\pgfqpoint{1.565000in}{3.520000in}}%
\pgfusepath{fill}%
\end{pgfscope}%
\begin{pgfscope}%
\pgfpathrectangle{\pgfqpoint{1.565000in}{0.500000in}}{\pgfqpoint{3.020000in}{3.020000in}}%
\pgfusepath{clip}%
\pgfsetbuttcap%
\pgfsetroundjoin%
\definecolor{currentfill}{rgb}{0.722401,0.491600,0.626404}%
\pgfsetfillcolor{currentfill}%
\pgfsetlinewidth{0.000000pt}%
\definecolor{currentstroke}{rgb}{1.000000,1.000000,1.000000}%
\pgfsetstrokecolor{currentstroke}%
\pgfsetdash{}{0pt}%
\pgfpathmoveto{\pgfqpoint{2.068333in}{3.520000in}}%
\pgfpathlineto{\pgfqpoint{2.571667in}{3.520000in}}%
\pgfpathlineto{\pgfqpoint{2.571667in}{3.016667in}}%
\pgfpathlineto{\pgfqpoint{2.068333in}{3.016667in}}%
\pgfpathlineto{\pgfqpoint{2.068333in}{3.520000in}}%
\pgfusepath{fill}%
\end{pgfscope}%
\begin{pgfscope}%
\pgfpathrectangle{\pgfqpoint{1.565000in}{0.500000in}}{\pgfqpoint{3.020000in}{3.020000in}}%
\pgfusepath{clip}%
\pgfsetbuttcap%
\pgfsetroundjoin%
\definecolor{currentfill}{rgb}{0.968724,0.945644,0.959125}%
\pgfsetfillcolor{currentfill}%
\pgfsetlinewidth{0.000000pt}%
\definecolor{currentstroke}{rgb}{1.000000,1.000000,1.000000}%
\pgfsetstrokecolor{currentstroke}%
\pgfsetdash{}{0pt}%
\pgfpathmoveto{\pgfqpoint{2.571667in}{3.520000in}}%
\pgfpathlineto{\pgfqpoint{3.075000in}{3.520000in}}%
\pgfpathlineto{\pgfqpoint{3.075000in}{3.016667in}}%
\pgfpathlineto{\pgfqpoint{2.571667in}{3.016667in}}%
\pgfpathlineto{\pgfqpoint{2.571667in}{3.520000in}}%
\pgfusepath{fill}%
\end{pgfscope}%
\begin{pgfscope}%
\pgfpathrectangle{\pgfqpoint{1.565000in}{0.500000in}}{\pgfqpoint{3.020000in}{3.020000in}}%
\pgfusepath{clip}%
\pgfsetbuttcap%
\pgfsetroundjoin%
\definecolor{currentfill}{rgb}{0.996094,0.996094,0.996094}%
\pgfsetfillcolor{currentfill}%
\pgfsetlinewidth{0.000000pt}%
\definecolor{currentstroke}{rgb}{1.000000,1.000000,1.000000}%
\pgfsetstrokecolor{currentstroke}%
\pgfsetdash{}{0pt}%
\pgfpathmoveto{\pgfqpoint{3.075000in}{3.520000in}}%
\pgfpathlineto{\pgfqpoint{3.578333in}{3.520000in}}%
\pgfpathlineto{\pgfqpoint{3.578333in}{3.016667in}}%
\pgfpathlineto{\pgfqpoint{3.075000in}{3.016667in}}%
\pgfpathlineto{\pgfqpoint{3.075000in}{3.520000in}}%
\pgfusepath{fill}%
\end{pgfscope}%
\begin{pgfscope}%
\pgfpathrectangle{\pgfqpoint{1.565000in}{0.500000in}}{\pgfqpoint{3.020000in}{3.020000in}}%
\pgfusepath{clip}%
\pgfsetbuttcap%
\pgfsetroundjoin%
\definecolor{currentfill}{rgb}{0.996094,0.996094,0.996094}%
\pgfsetfillcolor{currentfill}%
\pgfsetlinewidth{0.000000pt}%
\definecolor{currentstroke}{rgb}{1.000000,1.000000,1.000000}%
\pgfsetstrokecolor{currentstroke}%
\pgfsetdash{}{0pt}%
\pgfpathmoveto{\pgfqpoint{3.578333in}{3.520000in}}%
\pgfpathlineto{\pgfqpoint{4.081667in}{3.520000in}}%
\pgfpathlineto{\pgfqpoint{4.081667in}{3.016667in}}%
\pgfpathlineto{\pgfqpoint{3.578333in}{3.016667in}}%
\pgfpathlineto{\pgfqpoint{3.578333in}{3.520000in}}%
\pgfusepath{fill}%
\end{pgfscope}%
\begin{pgfscope}%
\pgfpathrectangle{\pgfqpoint{1.565000in}{0.500000in}}{\pgfqpoint{3.020000in}{3.020000in}}%
\pgfusepath{clip}%
\pgfsetbuttcap%
\pgfsetroundjoin%
\definecolor{currentfill}{rgb}{0.996094,0.996094,0.996094}%
\pgfsetfillcolor{currentfill}%
\pgfsetlinewidth{0.000000pt}%
\definecolor{currentstroke}{rgb}{1.000000,1.000000,1.000000}%
\pgfsetstrokecolor{currentstroke}%
\pgfsetdash{}{0pt}%
\pgfpathmoveto{\pgfqpoint{4.081667in}{3.520000in}}%
\pgfpathlineto{\pgfqpoint{4.585000in}{3.520000in}}%
\pgfpathlineto{\pgfqpoint{4.585000in}{3.016667in}}%
\pgfpathlineto{\pgfqpoint{4.081667in}{3.016667in}}%
\pgfpathlineto{\pgfqpoint{4.081667in}{3.520000in}}%
\pgfusepath{fill}%
\end{pgfscope}%
\begin{pgfscope}%
\pgfpathrectangle{\pgfqpoint{1.565000in}{0.500000in}}{\pgfqpoint{3.020000in}{3.020000in}}%
\pgfusepath{clip}%
\pgfsetbuttcap%
\pgfsetroundjoin%
\definecolor{currentfill}{rgb}{0.900301,0.819521,0.866702}%
\pgfsetfillcolor{currentfill}%
\pgfsetlinewidth{0.000000pt}%
\definecolor{currentstroke}{rgb}{1.000000,1.000000,1.000000}%
\pgfsetstrokecolor{currentstroke}%
\pgfsetdash{}{0pt}%
\pgfpathmoveto{\pgfqpoint{1.565000in}{3.016667in}}%
\pgfpathlineto{\pgfqpoint{2.068333in}{3.016667in}}%
\pgfpathlineto{\pgfqpoint{2.068333in}{2.513333in}}%
\pgfpathlineto{\pgfqpoint{1.565000in}{2.513333in}}%
\pgfpathlineto{\pgfqpoint{1.565000in}{3.016667in}}%
\pgfusepath{fill}%
\end{pgfscope}%
\begin{pgfscope}%
\pgfpathrectangle{\pgfqpoint{1.565000in}{0.500000in}}{\pgfqpoint{3.020000in}{3.020000in}}%
\pgfusepath{clip}%
\pgfsetbuttcap%
\pgfsetroundjoin%
\definecolor{currentfill}{rgb}{0.472656,0.031250,0.289062}%
\pgfsetfillcolor{currentfill}%
\pgfsetlinewidth{0.000000pt}%
\definecolor{currentstroke}{rgb}{1.000000,1.000000,1.000000}%
\pgfsetstrokecolor{currentstroke}%
\pgfsetdash{}{0pt}%
\pgfpathmoveto{\pgfqpoint{2.068333in}{3.016667in}}%
\pgfpathlineto{\pgfqpoint{2.571667in}{3.016667in}}%
\pgfpathlineto{\pgfqpoint{2.571667in}{2.513333in}}%
\pgfpathlineto{\pgfqpoint{2.068333in}{2.513333in}}%
\pgfpathlineto{\pgfqpoint{2.068333in}{3.016667in}}%
\pgfusepath{fill}%
\end{pgfscope}%
\begin{pgfscope}%
\pgfpathrectangle{\pgfqpoint{1.565000in}{0.500000in}}{\pgfqpoint{3.020000in}{3.020000in}}%
\pgfusepath{clip}%
\pgfsetbuttcap%
\pgfsetroundjoin%
\definecolor{currentfill}{rgb}{0.681347,0.415926,0.570951}%
\pgfsetfillcolor{currentfill}%
\pgfsetlinewidth{0.000000pt}%
\definecolor{currentstroke}{rgb}{1.000000,1.000000,1.000000}%
\pgfsetstrokecolor{currentstroke}%
\pgfsetdash{}{0pt}%
\pgfpathmoveto{\pgfqpoint{2.571667in}{3.016667in}}%
\pgfpathlineto{\pgfqpoint{3.075000in}{3.016667in}}%
\pgfpathlineto{\pgfqpoint{3.075000in}{2.513333in}}%
\pgfpathlineto{\pgfqpoint{2.571667in}{2.513333in}}%
\pgfpathlineto{\pgfqpoint{2.571667in}{3.016667in}}%
\pgfusepath{fill}%
\end{pgfscope}%
\begin{pgfscope}%
\pgfpathrectangle{\pgfqpoint{1.565000in}{0.500000in}}{\pgfqpoint{3.020000in}{3.020000in}}%
\pgfusepath{clip}%
\pgfsetbuttcap%
\pgfsetroundjoin%
\definecolor{currentfill}{rgb}{0.968724,0.945644,0.959125}%
\pgfsetfillcolor{currentfill}%
\pgfsetlinewidth{0.000000pt}%
\definecolor{currentstroke}{rgb}{1.000000,1.000000,1.000000}%
\pgfsetstrokecolor{currentstroke}%
\pgfsetdash{}{0pt}%
\pgfpathmoveto{\pgfqpoint{3.075000in}{3.016667in}}%
\pgfpathlineto{\pgfqpoint{3.578333in}{3.016667in}}%
\pgfpathlineto{\pgfqpoint{3.578333in}{2.513333in}}%
\pgfpathlineto{\pgfqpoint{3.075000in}{2.513333in}}%
\pgfpathlineto{\pgfqpoint{3.075000in}{3.016667in}}%
\pgfusepath{fill}%
\end{pgfscope}%
\begin{pgfscope}%
\pgfpathrectangle{\pgfqpoint{1.565000in}{0.500000in}}{\pgfqpoint{3.020000in}{3.020000in}}%
\pgfusepath{clip}%
\pgfsetbuttcap%
\pgfsetroundjoin%
\definecolor{currentfill}{rgb}{0.996094,0.996094,0.996094}%
\pgfsetfillcolor{currentfill}%
\pgfsetlinewidth{0.000000pt}%
\definecolor{currentstroke}{rgb}{1.000000,1.000000,1.000000}%
\pgfsetstrokecolor{currentstroke}%
\pgfsetdash{}{0pt}%
\pgfpathmoveto{\pgfqpoint{3.578333in}{3.016667in}}%
\pgfpathlineto{\pgfqpoint{4.081667in}{3.016667in}}%
\pgfpathlineto{\pgfqpoint{4.081667in}{2.513333in}}%
\pgfpathlineto{\pgfqpoint{3.578333in}{2.513333in}}%
\pgfpathlineto{\pgfqpoint{3.578333in}{3.016667in}}%
\pgfusepath{fill}%
\end{pgfscope}%
\begin{pgfscope}%
\pgfpathrectangle{\pgfqpoint{1.565000in}{0.500000in}}{\pgfqpoint{3.020000in}{3.020000in}}%
\pgfusepath{clip}%
\pgfsetbuttcap%
\pgfsetroundjoin%
\definecolor{currentfill}{rgb}{0.996094,0.996094,0.996094}%
\pgfsetfillcolor{currentfill}%
\pgfsetlinewidth{0.000000pt}%
\definecolor{currentstroke}{rgb}{1.000000,1.000000,1.000000}%
\pgfsetstrokecolor{currentstroke}%
\pgfsetdash{}{0pt}%
\pgfpathmoveto{\pgfqpoint{4.081667in}{3.016667in}}%
\pgfpathlineto{\pgfqpoint{4.585000in}{3.016667in}}%
\pgfpathlineto{\pgfqpoint{4.585000in}{2.513333in}}%
\pgfpathlineto{\pgfqpoint{4.081667in}{2.513333in}}%
\pgfpathlineto{\pgfqpoint{4.081667in}{3.016667in}}%
\pgfusepath{fill}%
\end{pgfscope}%
\begin{pgfscope}%
\pgfpathrectangle{\pgfqpoint{1.565000in}{0.500000in}}{\pgfqpoint{3.020000in}{3.020000in}}%
\pgfusepath{clip}%
\pgfsetbuttcap%
\pgfsetroundjoin%
\definecolor{currentfill}{rgb}{0.927671,0.869970,0.903671}%
\pgfsetfillcolor{currentfill}%
\pgfsetlinewidth{0.000000pt}%
\definecolor{currentstroke}{rgb}{1.000000,1.000000,1.000000}%
\pgfsetstrokecolor{currentstroke}%
\pgfsetdash{}{0pt}%
\pgfpathmoveto{\pgfqpoint{1.565000in}{2.513333in}}%
\pgfpathlineto{\pgfqpoint{2.068333in}{2.513333in}}%
\pgfpathlineto{\pgfqpoint{2.068333in}{2.010000in}}%
\pgfpathlineto{\pgfqpoint{1.565000in}{2.010000in}}%
\pgfpathlineto{\pgfqpoint{1.565000in}{2.513333in}}%
\pgfusepath{fill}%
\end{pgfscope}%
\begin{pgfscope}%
\pgfpathrectangle{\pgfqpoint{1.565000in}{0.500000in}}{\pgfqpoint{3.020000in}{3.020000in}}%
\pgfusepath{clip}%
\pgfsetbuttcap%
\pgfsetroundjoin%
\definecolor{currentfill}{rgb}{0.476077,0.037556,0.293684}%
\pgfsetfillcolor{currentfill}%
\pgfsetlinewidth{0.000000pt}%
\definecolor{currentstroke}{rgb}{1.000000,1.000000,1.000000}%
\pgfsetstrokecolor{currentstroke}%
\pgfsetdash{}{0pt}%
\pgfpathmoveto{\pgfqpoint{2.068333in}{2.513333in}}%
\pgfpathlineto{\pgfqpoint{2.571667in}{2.513333in}}%
\pgfpathlineto{\pgfqpoint{2.571667in}{2.010000in}}%
\pgfpathlineto{\pgfqpoint{2.068333in}{2.010000in}}%
\pgfpathlineto{\pgfqpoint{2.068333in}{2.513333in}}%
\pgfusepath{fill}%
\end{pgfscope}%
\begin{pgfscope}%
\pgfpathrectangle{\pgfqpoint{1.565000in}{0.500000in}}{\pgfqpoint{3.020000in}{3.020000in}}%
\pgfusepath{clip}%
\pgfsetbuttcap%
\pgfsetroundjoin%
\definecolor{currentfill}{rgb}{0.472656,0.031250,0.289062}%
\pgfsetfillcolor{currentfill}%
\pgfsetlinewidth{0.000000pt}%
\definecolor{currentstroke}{rgb}{1.000000,1.000000,1.000000}%
\pgfsetstrokecolor{currentstroke}%
\pgfsetdash{}{0pt}%
\pgfpathmoveto{\pgfqpoint{2.571667in}{2.513333in}}%
\pgfpathlineto{\pgfqpoint{3.075000in}{2.513333in}}%
\pgfpathlineto{\pgfqpoint{3.075000in}{2.010000in}}%
\pgfpathlineto{\pgfqpoint{2.571667in}{2.010000in}}%
\pgfpathlineto{\pgfqpoint{2.571667in}{2.513333in}}%
\pgfusepath{fill}%
\end{pgfscope}%
\begin{pgfscope}%
\pgfpathrectangle{\pgfqpoint{1.565000in}{0.500000in}}{\pgfqpoint{3.020000in}{3.020000in}}%
\pgfusepath{clip}%
\pgfsetbuttcap%
\pgfsetroundjoin%
\definecolor{currentfill}{rgb}{0.790824,0.617724,0.718827}%
\pgfsetfillcolor{currentfill}%
\pgfsetlinewidth{0.000000pt}%
\definecolor{currentstroke}{rgb}{1.000000,1.000000,1.000000}%
\pgfsetstrokecolor{currentstroke}%
\pgfsetdash{}{0pt}%
\pgfpathmoveto{\pgfqpoint{3.075000in}{2.513333in}}%
\pgfpathlineto{\pgfqpoint{3.578333in}{2.513333in}}%
\pgfpathlineto{\pgfqpoint{3.578333in}{2.010000in}}%
\pgfpathlineto{\pgfqpoint{3.075000in}{2.010000in}}%
\pgfpathlineto{\pgfqpoint{3.075000in}{2.513333in}}%
\pgfusepath{fill}%
\end{pgfscope}%
\begin{pgfscope}%
\pgfpathrectangle{\pgfqpoint{1.565000in}{0.500000in}}{\pgfqpoint{3.020000in}{3.020000in}}%
\pgfusepath{clip}%
\pgfsetbuttcap%
\pgfsetroundjoin%
\definecolor{currentfill}{rgb}{0.900301,0.819521,0.866702}%
\pgfsetfillcolor{currentfill}%
\pgfsetlinewidth{0.000000pt}%
\definecolor{currentstroke}{rgb}{1.000000,1.000000,1.000000}%
\pgfsetstrokecolor{currentstroke}%
\pgfsetdash{}{0pt}%
\pgfpathmoveto{\pgfqpoint{3.578333in}{2.513333in}}%
\pgfpathlineto{\pgfqpoint{4.081667in}{2.513333in}}%
\pgfpathlineto{\pgfqpoint{4.081667in}{2.010000in}}%
\pgfpathlineto{\pgfqpoint{3.578333in}{2.010000in}}%
\pgfpathlineto{\pgfqpoint{3.578333in}{2.513333in}}%
\pgfusepath{fill}%
\end{pgfscope}%
\begin{pgfscope}%
\pgfpathrectangle{\pgfqpoint{1.565000in}{0.500000in}}{\pgfqpoint{3.020000in}{3.020000in}}%
\pgfusepath{clip}%
\pgfsetbuttcap%
\pgfsetroundjoin%
\definecolor{currentfill}{rgb}{0.968724,0.945644,0.959125}%
\pgfsetfillcolor{currentfill}%
\pgfsetlinewidth{0.000000pt}%
\definecolor{currentstroke}{rgb}{1.000000,1.000000,1.000000}%
\pgfsetstrokecolor{currentstroke}%
\pgfsetdash{}{0pt}%
\pgfpathmoveto{\pgfqpoint{4.081667in}{2.513333in}}%
\pgfpathlineto{\pgfqpoint{4.585000in}{2.513333in}}%
\pgfpathlineto{\pgfqpoint{4.585000in}{2.010000in}}%
\pgfpathlineto{\pgfqpoint{4.081667in}{2.010000in}}%
\pgfpathlineto{\pgfqpoint{4.081667in}{2.513333in}}%
\pgfusepath{fill}%
\end{pgfscope}%
\begin{pgfscope}%
\pgfpathrectangle{\pgfqpoint{1.565000in}{0.500000in}}{\pgfqpoint{3.020000in}{3.020000in}}%
\pgfusepath{clip}%
\pgfsetbuttcap%
\pgfsetroundjoin%
\definecolor{currentfill}{rgb}{0.968724,0.945644,0.959125}%
\pgfsetfillcolor{currentfill}%
\pgfsetlinewidth{0.000000pt}%
\definecolor{currentstroke}{rgb}{1.000000,1.000000,1.000000}%
\pgfsetstrokecolor{currentstroke}%
\pgfsetdash{}{0pt}%
\pgfpathmoveto{\pgfqpoint{1.565000in}{2.010000in}}%
\pgfpathlineto{\pgfqpoint{2.068333in}{2.010000in}}%
\pgfpathlineto{\pgfqpoint{2.068333in}{1.506667in}}%
\pgfpathlineto{\pgfqpoint{1.565000in}{1.506667in}}%
\pgfpathlineto{\pgfqpoint{1.565000in}{2.010000in}}%
\pgfusepath{fill}%
\end{pgfscope}%
\begin{pgfscope}%
\pgfpathrectangle{\pgfqpoint{1.565000in}{0.500000in}}{\pgfqpoint{3.020000in}{3.020000in}}%
\pgfusepath{clip}%
\pgfsetbuttcap%
\pgfsetroundjoin%
\definecolor{currentfill}{rgb}{0.996094,0.996094,0.996094}%
\pgfsetfillcolor{currentfill}%
\pgfsetlinewidth{0.000000pt}%
\definecolor{currentstroke}{rgb}{1.000000,1.000000,1.000000}%
\pgfsetstrokecolor{currentstroke}%
\pgfsetdash{}{0pt}%
\pgfpathmoveto{\pgfqpoint{2.068333in}{2.010000in}}%
\pgfpathlineto{\pgfqpoint{2.571667in}{2.010000in}}%
\pgfpathlineto{\pgfqpoint{2.571667in}{1.506667in}}%
\pgfpathlineto{\pgfqpoint{2.068333in}{1.506667in}}%
\pgfpathlineto{\pgfqpoint{2.068333in}{2.010000in}}%
\pgfusepath{fill}%
\end{pgfscope}%
\begin{pgfscope}%
\pgfpathrectangle{\pgfqpoint{1.565000in}{0.500000in}}{\pgfqpoint{3.020000in}{3.020000in}}%
\pgfusepath{clip}%
\pgfsetbuttcap%
\pgfsetroundjoin%
\definecolor{currentfill}{rgb}{0.749770,0.542050,0.663373}%
\pgfsetfillcolor{currentfill}%
\pgfsetlinewidth{0.000000pt}%
\definecolor{currentstroke}{rgb}{1.000000,1.000000,1.000000}%
\pgfsetstrokecolor{currentstroke}%
\pgfsetdash{}{0pt}%
\pgfpathmoveto{\pgfqpoint{2.571667in}{2.010000in}}%
\pgfpathlineto{\pgfqpoint{3.075000in}{2.010000in}}%
\pgfpathlineto{\pgfqpoint{3.075000in}{1.506667in}}%
\pgfpathlineto{\pgfqpoint{2.571667in}{1.506667in}}%
\pgfpathlineto{\pgfqpoint{2.571667in}{2.010000in}}%
\pgfusepath{fill}%
\end{pgfscope}%
\begin{pgfscope}%
\pgfpathrectangle{\pgfqpoint{1.565000in}{0.500000in}}{\pgfqpoint{3.020000in}{3.020000in}}%
\pgfusepath{clip}%
\pgfsetbuttcap%
\pgfsetroundjoin%
\definecolor{currentfill}{rgb}{0.472656,0.031250,0.289062}%
\pgfsetfillcolor{currentfill}%
\pgfsetlinewidth{0.000000pt}%
\definecolor{currentstroke}{rgb}{1.000000,1.000000,1.000000}%
\pgfsetstrokecolor{currentstroke}%
\pgfsetdash{}{0pt}%
\pgfpathmoveto{\pgfqpoint{3.075000in}{2.010000in}}%
\pgfpathlineto{\pgfqpoint{3.578333in}{2.010000in}}%
\pgfpathlineto{\pgfqpoint{3.578333in}{1.506667in}}%
\pgfpathlineto{\pgfqpoint{3.075000in}{1.506667in}}%
\pgfpathlineto{\pgfqpoint{3.075000in}{2.010000in}}%
\pgfusepath{fill}%
\end{pgfscope}%
\begin{pgfscope}%
\pgfpathrectangle{\pgfqpoint{1.565000in}{0.500000in}}{\pgfqpoint{3.020000in}{3.020000in}}%
\pgfusepath{clip}%
\pgfsetbuttcap%
\pgfsetroundjoin%
\definecolor{currentfill}{rgb}{0.472656,0.031250,0.289062}%
\pgfsetfillcolor{currentfill}%
\pgfsetlinewidth{0.000000pt}%
\definecolor{currentstroke}{rgb}{1.000000,1.000000,1.000000}%
\pgfsetstrokecolor{currentstroke}%
\pgfsetdash{}{0pt}%
\pgfpathmoveto{\pgfqpoint{3.578333in}{2.010000in}}%
\pgfpathlineto{\pgfqpoint{4.081667in}{2.010000in}}%
\pgfpathlineto{\pgfqpoint{4.081667in}{1.506667in}}%
\pgfpathlineto{\pgfqpoint{3.578333in}{1.506667in}}%
\pgfpathlineto{\pgfqpoint{3.578333in}{2.010000in}}%
\pgfusepath{fill}%
\end{pgfscope}%
\begin{pgfscope}%
\pgfpathrectangle{\pgfqpoint{1.565000in}{0.500000in}}{\pgfqpoint{3.020000in}{3.020000in}}%
\pgfusepath{clip}%
\pgfsetbuttcap%
\pgfsetroundjoin%
\definecolor{currentfill}{rgb}{0.859247,0.743847,0.811249}%
\pgfsetfillcolor{currentfill}%
\pgfsetlinewidth{0.000000pt}%
\definecolor{currentstroke}{rgb}{1.000000,1.000000,1.000000}%
\pgfsetstrokecolor{currentstroke}%
\pgfsetdash{}{0pt}%
\pgfpathmoveto{\pgfqpoint{4.081667in}{2.010000in}}%
\pgfpathlineto{\pgfqpoint{4.585000in}{2.010000in}}%
\pgfpathlineto{\pgfqpoint{4.585000in}{1.506667in}}%
\pgfpathlineto{\pgfqpoint{4.081667in}{1.506667in}}%
\pgfpathlineto{\pgfqpoint{4.081667in}{2.010000in}}%
\pgfusepath{fill}%
\end{pgfscope}%
\begin{pgfscope}%
\pgfpathrectangle{\pgfqpoint{1.565000in}{0.500000in}}{\pgfqpoint{3.020000in}{3.020000in}}%
\pgfusepath{clip}%
\pgfsetbuttcap%
\pgfsetroundjoin%
\definecolor{currentfill}{rgb}{0.968724,0.945644,0.959125}%
\pgfsetfillcolor{currentfill}%
\pgfsetlinewidth{0.000000pt}%
\definecolor{currentstroke}{rgb}{1.000000,1.000000,1.000000}%
\pgfsetstrokecolor{currentstroke}%
\pgfsetdash{}{0pt}%
\pgfpathmoveto{\pgfqpoint{1.565000in}{1.506667in}}%
\pgfpathlineto{\pgfqpoint{2.068333in}{1.506667in}}%
\pgfpathlineto{\pgfqpoint{2.068333in}{1.003333in}}%
\pgfpathlineto{\pgfqpoint{1.565000in}{1.003333in}}%
\pgfpathlineto{\pgfqpoint{1.565000in}{1.506667in}}%
\pgfusepath{fill}%
\end{pgfscope}%
\begin{pgfscope}%
\pgfpathrectangle{\pgfqpoint{1.565000in}{0.500000in}}{\pgfqpoint{3.020000in}{3.020000in}}%
\pgfusepath{clip}%
\pgfsetbuttcap%
\pgfsetroundjoin%
\definecolor{currentfill}{rgb}{0.968724,0.945644,0.959125}%
\pgfsetfillcolor{currentfill}%
\pgfsetlinewidth{0.000000pt}%
\definecolor{currentstroke}{rgb}{1.000000,1.000000,1.000000}%
\pgfsetstrokecolor{currentstroke}%
\pgfsetdash{}{0pt}%
\pgfpathmoveto{\pgfqpoint{2.068333in}{1.506667in}}%
\pgfpathlineto{\pgfqpoint{2.571667in}{1.506667in}}%
\pgfpathlineto{\pgfqpoint{2.571667in}{1.003333in}}%
\pgfpathlineto{\pgfqpoint{2.068333in}{1.003333in}}%
\pgfpathlineto{\pgfqpoint{2.068333in}{1.506667in}}%
\pgfusepath{fill}%
\end{pgfscope}%
\begin{pgfscope}%
\pgfpathrectangle{\pgfqpoint{1.565000in}{0.500000in}}{\pgfqpoint{3.020000in}{3.020000in}}%
\pgfusepath{clip}%
\pgfsetbuttcap%
\pgfsetroundjoin%
\definecolor{currentfill}{rgb}{0.996094,0.996094,0.996094}%
\pgfsetfillcolor{currentfill}%
\pgfsetlinewidth{0.000000pt}%
\definecolor{currentstroke}{rgb}{1.000000,1.000000,1.000000}%
\pgfsetstrokecolor{currentstroke}%
\pgfsetdash{}{0pt}%
\pgfpathmoveto{\pgfqpoint{2.571667in}{1.506667in}}%
\pgfpathlineto{\pgfqpoint{3.075000in}{1.506667in}}%
\pgfpathlineto{\pgfqpoint{3.075000in}{1.003333in}}%
\pgfpathlineto{\pgfqpoint{2.571667in}{1.003333in}}%
\pgfpathlineto{\pgfqpoint{2.571667in}{1.506667in}}%
\pgfusepath{fill}%
\end{pgfscope}%
\begin{pgfscope}%
\pgfpathrectangle{\pgfqpoint{1.565000in}{0.500000in}}{\pgfqpoint{3.020000in}{3.020000in}}%
\pgfusepath{clip}%
\pgfsetbuttcap%
\pgfsetroundjoin%
\definecolor{currentfill}{rgb}{0.996094,0.996094,0.996094}%
\pgfsetfillcolor{currentfill}%
\pgfsetlinewidth{0.000000pt}%
\definecolor{currentstroke}{rgb}{1.000000,1.000000,1.000000}%
\pgfsetstrokecolor{currentstroke}%
\pgfsetdash{}{0pt}%
\pgfpathmoveto{\pgfqpoint{3.075000in}{1.506667in}}%
\pgfpathlineto{\pgfqpoint{3.578333in}{1.506667in}}%
\pgfpathlineto{\pgfqpoint{3.578333in}{1.003333in}}%
\pgfpathlineto{\pgfqpoint{3.075000in}{1.003333in}}%
\pgfpathlineto{\pgfqpoint{3.075000in}{1.506667in}}%
\pgfusepath{fill}%
\end{pgfscope}%
\begin{pgfscope}%
\pgfpathrectangle{\pgfqpoint{1.565000in}{0.500000in}}{\pgfqpoint{3.020000in}{3.020000in}}%
\pgfusepath{clip}%
\pgfsetbuttcap%
\pgfsetroundjoin%
\definecolor{currentfill}{rgb}{0.472656,0.031250,0.289062}%
\pgfsetfillcolor{currentfill}%
\pgfsetlinewidth{0.000000pt}%
\definecolor{currentstroke}{rgb}{1.000000,1.000000,1.000000}%
\pgfsetstrokecolor{currentstroke}%
\pgfsetdash{}{0pt}%
\pgfpathmoveto{\pgfqpoint{3.578333in}{1.506667in}}%
\pgfpathlineto{\pgfqpoint{4.081667in}{1.506667in}}%
\pgfpathlineto{\pgfqpoint{4.081667in}{1.003333in}}%
\pgfpathlineto{\pgfqpoint{3.578333in}{1.003333in}}%
\pgfpathlineto{\pgfqpoint{3.578333in}{1.506667in}}%
\pgfusepath{fill}%
\end{pgfscope}%
\begin{pgfscope}%
\pgfpathrectangle{\pgfqpoint{1.565000in}{0.500000in}}{\pgfqpoint{3.020000in}{3.020000in}}%
\pgfusepath{clip}%
\pgfsetbuttcap%
\pgfsetroundjoin%
\definecolor{currentfill}{rgb}{0.968724,0.945644,0.959125}%
\pgfsetfillcolor{currentfill}%
\pgfsetlinewidth{0.000000pt}%
\definecolor{currentstroke}{rgb}{1.000000,1.000000,1.000000}%
\pgfsetstrokecolor{currentstroke}%
\pgfsetdash{}{0pt}%
\pgfpathmoveto{\pgfqpoint{4.081667in}{1.506667in}}%
\pgfpathlineto{\pgfqpoint{4.585000in}{1.506667in}}%
\pgfpathlineto{\pgfqpoint{4.585000in}{1.003333in}}%
\pgfpathlineto{\pgfqpoint{4.081667in}{1.003333in}}%
\pgfpathlineto{\pgfqpoint{4.081667in}{1.506667in}}%
\pgfusepath{fill}%
\end{pgfscope}%
\begin{pgfscope}%
\pgfpathrectangle{\pgfqpoint{1.565000in}{0.500000in}}{\pgfqpoint{3.020000in}{3.020000in}}%
\pgfusepath{clip}%
\pgfsetbuttcap%
\pgfsetroundjoin%
\definecolor{currentfill}{rgb}{0.968724,0.945644,0.959125}%
\pgfsetfillcolor{currentfill}%
\pgfsetlinewidth{0.000000pt}%
\definecolor{currentstroke}{rgb}{1.000000,1.000000,1.000000}%
\pgfsetstrokecolor{currentstroke}%
\pgfsetdash{}{0pt}%
\pgfpathmoveto{\pgfqpoint{1.565000in}{1.003333in}}%
\pgfpathlineto{\pgfqpoint{2.068333in}{1.003333in}}%
\pgfpathlineto{\pgfqpoint{2.068333in}{0.500000in}}%
\pgfpathlineto{\pgfqpoint{1.565000in}{0.500000in}}%
\pgfpathlineto{\pgfqpoint{1.565000in}{1.003333in}}%
\pgfusepath{fill}%
\end{pgfscope}%
\begin{pgfscope}%
\pgfpathrectangle{\pgfqpoint{1.565000in}{0.500000in}}{\pgfqpoint{3.020000in}{3.020000in}}%
\pgfusepath{clip}%
\pgfsetbuttcap%
\pgfsetroundjoin%
\definecolor{currentfill}{rgb}{0.996094,0.996094,0.996094}%
\pgfsetfillcolor{currentfill}%
\pgfsetlinewidth{0.000000pt}%
\definecolor{currentstroke}{rgb}{1.000000,1.000000,1.000000}%
\pgfsetstrokecolor{currentstroke}%
\pgfsetdash{}{0pt}%
\pgfpathmoveto{\pgfqpoint{2.068333in}{1.003333in}}%
\pgfpathlineto{\pgfqpoint{2.571667in}{1.003333in}}%
\pgfpathlineto{\pgfqpoint{2.571667in}{0.500000in}}%
\pgfpathlineto{\pgfqpoint{2.068333in}{0.500000in}}%
\pgfpathlineto{\pgfqpoint{2.068333in}{1.003333in}}%
\pgfusepath{fill}%
\end{pgfscope}%
\begin{pgfscope}%
\pgfpathrectangle{\pgfqpoint{1.565000in}{0.500000in}}{\pgfqpoint{3.020000in}{3.020000in}}%
\pgfusepath{clip}%
\pgfsetbuttcap%
\pgfsetroundjoin%
\definecolor{currentfill}{rgb}{0.996094,0.996094,0.996094}%
\pgfsetfillcolor{currentfill}%
\pgfsetlinewidth{0.000000pt}%
\definecolor{currentstroke}{rgb}{1.000000,1.000000,1.000000}%
\pgfsetstrokecolor{currentstroke}%
\pgfsetdash{}{0pt}%
\pgfpathmoveto{\pgfqpoint{2.571667in}{1.003333in}}%
\pgfpathlineto{\pgfqpoint{3.075000in}{1.003333in}}%
\pgfpathlineto{\pgfqpoint{3.075000in}{0.500000in}}%
\pgfpathlineto{\pgfqpoint{2.571667in}{0.500000in}}%
\pgfpathlineto{\pgfqpoint{2.571667in}{1.003333in}}%
\pgfusepath{fill}%
\end{pgfscope}%
\begin{pgfscope}%
\pgfpathrectangle{\pgfqpoint{1.565000in}{0.500000in}}{\pgfqpoint{3.020000in}{3.020000in}}%
\pgfusepath{clip}%
\pgfsetbuttcap%
\pgfsetroundjoin%
\definecolor{currentfill}{rgb}{0.996094,0.996094,0.996094}%
\pgfsetfillcolor{currentfill}%
\pgfsetlinewidth{0.000000pt}%
\definecolor{currentstroke}{rgb}{1.000000,1.000000,1.000000}%
\pgfsetstrokecolor{currentstroke}%
\pgfsetdash{}{0pt}%
\pgfpathmoveto{\pgfqpoint{3.075000in}{1.003333in}}%
\pgfpathlineto{\pgfqpoint{3.578333in}{1.003333in}}%
\pgfpathlineto{\pgfqpoint{3.578333in}{0.500000in}}%
\pgfpathlineto{\pgfqpoint{3.075000in}{0.500000in}}%
\pgfpathlineto{\pgfqpoint{3.075000in}{1.003333in}}%
\pgfusepath{fill}%
\end{pgfscope}%
\begin{pgfscope}%
\pgfpathrectangle{\pgfqpoint{1.565000in}{0.500000in}}{\pgfqpoint{3.020000in}{3.020000in}}%
\pgfusepath{clip}%
\pgfsetbuttcap%
\pgfsetroundjoin%
\definecolor{currentfill}{rgb}{0.472656,0.031250,0.289062}%
\pgfsetfillcolor{currentfill}%
\pgfsetlinewidth{0.000000pt}%
\definecolor{currentstroke}{rgb}{1.000000,1.000000,1.000000}%
\pgfsetstrokecolor{currentstroke}%
\pgfsetdash{}{0pt}%
\pgfpathmoveto{\pgfqpoint{3.578333in}{1.003333in}}%
\pgfpathlineto{\pgfqpoint{4.081667in}{1.003333in}}%
\pgfpathlineto{\pgfqpoint{4.081667in}{0.500000in}}%
\pgfpathlineto{\pgfqpoint{3.578333in}{0.500000in}}%
\pgfpathlineto{\pgfqpoint{3.578333in}{1.003333in}}%
\pgfusepath{fill}%
\end{pgfscope}%
\begin{pgfscope}%
\pgfpathrectangle{\pgfqpoint{1.565000in}{0.500000in}}{\pgfqpoint{3.020000in}{3.020000in}}%
\pgfusepath{clip}%
\pgfsetbuttcap%
\pgfsetroundjoin%
\definecolor{currentfill}{rgb}{0.681347,0.415926,0.570951}%
\pgfsetfillcolor{currentfill}%
\pgfsetlinewidth{0.000000pt}%
\definecolor{currentstroke}{rgb}{1.000000,1.000000,1.000000}%
\pgfsetstrokecolor{currentstroke}%
\pgfsetdash{}{0pt}%
\pgfpathmoveto{\pgfqpoint{4.081667in}{1.003333in}}%
\pgfpathlineto{\pgfqpoint{4.585000in}{1.003333in}}%
\pgfpathlineto{\pgfqpoint{4.585000in}{0.500000in}}%
\pgfpathlineto{\pgfqpoint{4.081667in}{0.500000in}}%
\pgfpathlineto{\pgfqpoint{4.081667in}{1.003333in}}%
\pgfusepath{fill}%
\end{pgfscope}%
\begin{pgfscope}%
\definecolor{textcolor}{rgb}{0.000000,0.000000,0.000000}%
\pgfsetstrokecolor{textcolor}%
\pgfsetfillcolor{textcolor}%
\pgftext[x=1.816667in,y=0.402778in,,top]{\color{textcolor}\rmfamily\fontsize{10.000000}{12.000000}\selectfont 0}%
\end{pgfscope}%
\begin{pgfscope}%
\definecolor{textcolor}{rgb}{0.000000,0.000000,0.000000}%
\pgfsetstrokecolor{textcolor}%
\pgfsetfillcolor{textcolor}%
\pgftext[x=2.320000in,y=0.402778in,,top]{\color{textcolor}\rmfamily\fontsize{10.000000}{12.000000}\selectfont 1}%
\end{pgfscope}%
\begin{pgfscope}%
\definecolor{textcolor}{rgb}{0.000000,0.000000,0.000000}%
\pgfsetstrokecolor{textcolor}%
\pgfsetfillcolor{textcolor}%
\pgftext[x=2.823333in,y=0.402778in,,top]{\color{textcolor}\rmfamily\fontsize{10.000000}{12.000000}\selectfont 2}%
\end{pgfscope}%
\begin{pgfscope}%
\definecolor{textcolor}{rgb}{0.000000,0.000000,0.000000}%
\pgfsetstrokecolor{textcolor}%
\pgfsetfillcolor{textcolor}%
\pgftext[x=3.326667in,y=0.402778in,,top]{\color{textcolor}\rmfamily\fontsize{10.000000}{12.000000}\selectfont 3}%
\end{pgfscope}%
\begin{pgfscope}%
\definecolor{textcolor}{rgb}{0.000000,0.000000,0.000000}%
\pgfsetstrokecolor{textcolor}%
\pgfsetfillcolor{textcolor}%
\pgftext[x=3.830000in,y=0.402778in,,top]{\color{textcolor}\rmfamily\fontsize{10.000000}{12.000000}\selectfont 4}%
\end{pgfscope}%
\begin{pgfscope}%
\definecolor{textcolor}{rgb}{0.000000,0.000000,0.000000}%
\pgfsetstrokecolor{textcolor}%
\pgfsetfillcolor{textcolor}%
\pgftext[x=4.333333in,y=0.402778in,,top]{\color{textcolor}\rmfamily\fontsize{10.000000}{12.000000}\selectfont 5}%
\end{pgfscope}%
\begin{pgfscope}%
\definecolor{textcolor}{rgb}{0.000000,0.000000,0.000000}%
\pgfsetstrokecolor{textcolor}%
\pgfsetfillcolor{textcolor}%
\pgftext[x=3.075000in,y=0.195988in,,top]{\color{textcolor}\rmfamily\fontsize{15.000000}{18.000000}\selectfont Predicted ISUP grade}%
\end{pgfscope}%
\begin{pgfscope}%
\definecolor{textcolor}{rgb}{0.000000,0.000000,0.000000}%
\pgfsetstrokecolor{textcolor}%
\pgfsetfillcolor{textcolor}%
\pgftext[x=1.440772in, y=3.247114in, left, base,rotate=90.000000]{\color{textcolor}\rmfamily\fontsize{10.000000}{12.000000}\selectfont 0}%
\end{pgfscope}%
\begin{pgfscope}%
\definecolor{textcolor}{rgb}{0.000000,0.000000,0.000000}%
\pgfsetstrokecolor{textcolor}%
\pgfsetfillcolor{textcolor}%
\pgftext[x=1.440772in, y=2.743781in, left, base,rotate=90.000000]{\color{textcolor}\rmfamily\fontsize{10.000000}{12.000000}\selectfont 1}%
\end{pgfscope}%
\begin{pgfscope}%
\definecolor{textcolor}{rgb}{0.000000,0.000000,0.000000}%
\pgfsetstrokecolor{textcolor}%
\pgfsetfillcolor{textcolor}%
\pgftext[x=1.440772in, y=2.240447in, left, base,rotate=90.000000]{\color{textcolor}\rmfamily\fontsize{10.000000}{12.000000}\selectfont 2}%
\end{pgfscope}%
\begin{pgfscope}%
\definecolor{textcolor}{rgb}{0.000000,0.000000,0.000000}%
\pgfsetstrokecolor{textcolor}%
\pgfsetfillcolor{textcolor}%
\pgftext[x=1.440772in, y=1.737114in, left, base,rotate=90.000000]{\color{textcolor}\rmfamily\fontsize{10.000000}{12.000000}\selectfont 3}%
\end{pgfscope}%
\begin{pgfscope}%
\definecolor{textcolor}{rgb}{0.000000,0.000000,0.000000}%
\pgfsetstrokecolor{textcolor}%
\pgfsetfillcolor{textcolor}%
\pgftext[x=1.440772in, y=1.233781in, left, base,rotate=90.000000]{\color{textcolor}\rmfamily\fontsize{10.000000}{12.000000}\selectfont 4}%
\end{pgfscope}%
\begin{pgfscope}%
\definecolor{textcolor}{rgb}{0.000000,0.000000,0.000000}%
\pgfsetstrokecolor{textcolor}%
\pgfsetfillcolor{textcolor}%
\pgftext[x=1.440772in, y=0.730447in, left, base,rotate=90.000000]{\color{textcolor}\rmfamily\fontsize{10.000000}{12.000000}\selectfont 5}%
\end{pgfscope}%
\begin{pgfscope}%
\definecolor{textcolor}{rgb}{0.000000,0.000000,0.000000}%
\pgfsetstrokecolor{textcolor}%
\pgfsetfillcolor{textcolor}%
\pgftext[x=1.260988in,y=2.010000in,,bottom,rotate=90.000000]{\color{textcolor}\rmfamily\fontsize{15.000000}{18.000000}\selectfont Ground truth ISUP grade}%
\end{pgfscope}%
\begin{pgfscope}%
\pgfsetrectcap%
\pgfsetmiterjoin%
\pgfsetlinewidth{0.803000pt}%
\definecolor{currentstroke}{rgb}{0.000000,0.000000,0.000000}%
\pgfsetstrokecolor{currentstroke}%
\pgfsetdash{}{0pt}%
\pgfpathmoveto{\pgfqpoint{1.565000in}{0.500000in}}%
\pgfpathlineto{\pgfqpoint{1.565000in}{3.520000in}}%
\pgfusepath{stroke}%
\end{pgfscope}%
\begin{pgfscope}%
\pgfsetrectcap%
\pgfsetmiterjoin%
\pgfsetlinewidth{0.803000pt}%
\definecolor{currentstroke}{rgb}{0.000000,0.000000,0.000000}%
\pgfsetstrokecolor{currentstroke}%
\pgfsetdash{}{0pt}%
\pgfpathmoveto{\pgfqpoint{4.585000in}{0.500000in}}%
\pgfpathlineto{\pgfqpoint{4.585000in}{3.520000in}}%
\pgfusepath{stroke}%
\end{pgfscope}%
\begin{pgfscope}%
\pgfsetrectcap%
\pgfsetmiterjoin%
\pgfsetlinewidth{0.803000pt}%
\definecolor{currentstroke}{rgb}{0.000000,0.000000,0.000000}%
\pgfsetstrokecolor{currentstroke}%
\pgfsetdash{}{0pt}%
\pgfpathmoveto{\pgfqpoint{1.565000in}{0.500000in}}%
\pgfpathlineto{\pgfqpoint{4.585000in}{0.500000in}}%
\pgfusepath{stroke}%
\end{pgfscope}%
\begin{pgfscope}%
\pgfsetrectcap%
\pgfsetmiterjoin%
\pgfsetlinewidth{0.803000pt}%
\definecolor{currentstroke}{rgb}{0.000000,0.000000,0.000000}%
\pgfsetstrokecolor{currentstroke}%
\pgfsetdash{}{0pt}%
\pgfpathmoveto{\pgfqpoint{1.565000in}{3.520000in}}%
\pgfpathlineto{\pgfqpoint{4.585000in}{3.520000in}}%
\pgfusepath{stroke}%
\end{pgfscope}%
\begin{pgfscope}%
\definecolor{textcolor}{rgb}{1.000000,1.000000,1.000000}%
\pgfsetstrokecolor{textcolor}%
\pgfsetfillcolor{textcolor}%
\pgftext[x=1.816667in,y=3.268333in,,]{\color{textcolor}\rmfamily\fontsize{10.000000}{12.000000}\selectfont 99}%
\end{pgfscope}%
\begin{pgfscope}%
\definecolor{textcolor}{rgb}{1.000000,1.000000,1.000000}%
\pgfsetstrokecolor{textcolor}%
\pgfsetfillcolor{textcolor}%
\pgftext[x=2.320000in,y=3.268333in,,]{\color{textcolor}\rmfamily\fontsize{10.000000}{12.000000}\selectfont 8}%
\end{pgfscope}%
\begin{pgfscope}%
\definecolor{textcolor}{rgb}{0.150000,0.150000,0.150000}%
\pgfsetstrokecolor{textcolor}%
\pgfsetfillcolor{textcolor}%
\pgftext[x=2.823333in,y=3.268333in,,]{\color{textcolor}\rmfamily\fontsize{10.000000}{12.000000}\selectfont 1}%
\end{pgfscope}%
\begin{pgfscope}%
\definecolor{textcolor}{rgb}{0.150000,0.150000,0.150000}%
\pgfsetstrokecolor{textcolor}%
\pgfsetfillcolor{textcolor}%
\pgftext[x=3.326667in,y=3.268333in,,]{\color{textcolor}\rmfamily\fontsize{10.000000}{12.000000}\selectfont 0}%
\end{pgfscope}%
\begin{pgfscope}%
\definecolor{textcolor}{rgb}{0.150000,0.150000,0.150000}%
\pgfsetstrokecolor{textcolor}%
\pgfsetfillcolor{textcolor}%
\pgftext[x=3.830000in,y=3.268333in,,]{\color{textcolor}\rmfamily\fontsize{10.000000}{12.000000}\selectfont 0}%
\end{pgfscope}%
\begin{pgfscope}%
\definecolor{textcolor}{rgb}{0.150000,0.150000,0.150000}%
\pgfsetstrokecolor{textcolor}%
\pgfsetfillcolor{textcolor}%
\pgftext[x=4.333333in,y=3.268333in,,]{\color{textcolor}\rmfamily\fontsize{10.000000}{12.000000}\selectfont 0}%
\end{pgfscope}%
\begin{pgfscope}%
\definecolor{textcolor}{rgb}{0.150000,0.150000,0.150000}%
\pgfsetstrokecolor{textcolor}%
\pgfsetfillcolor{textcolor}%
\pgftext[x=1.816667in,y=2.765000in,,]{\color{textcolor}\rmfamily\fontsize{10.000000}{12.000000}\selectfont 3}%
\end{pgfscope}%
\begin{pgfscope}%
\definecolor{textcolor}{rgb}{1.000000,1.000000,1.000000}%
\pgfsetstrokecolor{textcolor}%
\pgfsetfillcolor{textcolor}%
\pgftext[x=2.320000in,y=2.765000in,,]{\color{textcolor}\rmfamily\fontsize{10.000000}{12.000000}\selectfont 52}%
\end{pgfscope}%
\begin{pgfscope}%
\definecolor{textcolor}{rgb}{1.000000,1.000000,1.000000}%
\pgfsetstrokecolor{textcolor}%
\pgfsetfillcolor{textcolor}%
\pgftext[x=2.823333in,y=2.765000in,,]{\color{textcolor}\rmfamily\fontsize{10.000000}{12.000000}\selectfont 9}%
\end{pgfscope}%
\begin{pgfscope}%
\definecolor{textcolor}{rgb}{0.150000,0.150000,0.150000}%
\pgfsetstrokecolor{textcolor}%
\pgfsetfillcolor{textcolor}%
\pgftext[x=3.326667in,y=2.765000in,,]{\color{textcolor}\rmfamily\fontsize{10.000000}{12.000000}\selectfont 1}%
\end{pgfscope}%
\begin{pgfscope}%
\definecolor{textcolor}{rgb}{0.150000,0.150000,0.150000}%
\pgfsetstrokecolor{textcolor}%
\pgfsetfillcolor{textcolor}%
\pgftext[x=3.830000in,y=2.765000in,,]{\color{textcolor}\rmfamily\fontsize{10.000000}{12.000000}\selectfont 0}%
\end{pgfscope}%
\begin{pgfscope}%
\definecolor{textcolor}{rgb}{0.150000,0.150000,0.150000}%
\pgfsetstrokecolor{textcolor}%
\pgfsetfillcolor{textcolor}%
\pgftext[x=4.333333in,y=2.765000in,,]{\color{textcolor}\rmfamily\fontsize{10.000000}{12.000000}\selectfont 0}%
\end{pgfscope}%
\begin{pgfscope}%
\definecolor{textcolor}{rgb}{0.150000,0.150000,0.150000}%
\pgfsetstrokecolor{textcolor}%
\pgfsetfillcolor{textcolor}%
\pgftext[x=1.816667in,y=2.261667in,,]{\color{textcolor}\rmfamily\fontsize{10.000000}{12.000000}\selectfont 2}%
\end{pgfscope}%
\begin{pgfscope}%
\definecolor{textcolor}{rgb}{1.000000,1.000000,1.000000}%
\pgfsetstrokecolor{textcolor}%
\pgfsetfillcolor{textcolor}%
\pgftext[x=2.320000in,y=2.261667in,,]{\color{textcolor}\rmfamily\fontsize{10.000000}{12.000000}\selectfont 15}%
\end{pgfscope}%
\begin{pgfscope}%
\definecolor{textcolor}{rgb}{1.000000,1.000000,1.000000}%
\pgfsetstrokecolor{textcolor}%
\pgfsetfillcolor{textcolor}%
\pgftext[x=2.823333in,y=2.261667in,,]{\color{textcolor}\rmfamily\fontsize{10.000000}{12.000000}\selectfont 36}%
\end{pgfscope}%
\begin{pgfscope}%
\definecolor{textcolor}{rgb}{1.000000,1.000000,1.000000}%
\pgfsetstrokecolor{textcolor}%
\pgfsetfillcolor{textcolor}%
\pgftext[x=3.326667in,y=2.261667in,,]{\color{textcolor}\rmfamily\fontsize{10.000000}{12.000000}\selectfont 6}%
\end{pgfscope}%
\begin{pgfscope}%
\definecolor{textcolor}{rgb}{0.150000,0.150000,0.150000}%
\pgfsetstrokecolor{textcolor}%
\pgfsetfillcolor{textcolor}%
\pgftext[x=3.830000in,y=2.261667in,,]{\color{textcolor}\rmfamily\fontsize{10.000000}{12.000000}\selectfont 3}%
\end{pgfscope}%
\begin{pgfscope}%
\definecolor{textcolor}{rgb}{0.150000,0.150000,0.150000}%
\pgfsetstrokecolor{textcolor}%
\pgfsetfillcolor{textcolor}%
\pgftext[x=4.333333in,y=2.261667in,,]{\color{textcolor}\rmfamily\fontsize{10.000000}{12.000000}\selectfont 1}%
\end{pgfscope}%
\begin{pgfscope}%
\definecolor{textcolor}{rgb}{0.150000,0.150000,0.150000}%
\pgfsetstrokecolor{textcolor}%
\pgfsetfillcolor{textcolor}%
\pgftext[x=1.816667in,y=1.758333in,,]{\color{textcolor}\rmfamily\fontsize{10.000000}{12.000000}\selectfont 1}%
\end{pgfscope}%
\begin{pgfscope}%
\definecolor{textcolor}{rgb}{0.150000,0.150000,0.150000}%
\pgfsetstrokecolor{textcolor}%
\pgfsetfillcolor{textcolor}%
\pgftext[x=2.320000in,y=1.758333in,,]{\color{textcolor}\rmfamily\fontsize{10.000000}{12.000000}\selectfont 0}%
\end{pgfscope}%
\begin{pgfscope}%
\definecolor{textcolor}{rgb}{1.000000,1.000000,1.000000}%
\pgfsetstrokecolor{textcolor}%
\pgfsetfillcolor{textcolor}%
\pgftext[x=2.823333in,y=1.758333in,,]{\color{textcolor}\rmfamily\fontsize{10.000000}{12.000000}\selectfont 7}%
\end{pgfscope}%
\begin{pgfscope}%
\definecolor{textcolor}{rgb}{1.000000,1.000000,1.000000}%
\pgfsetstrokecolor{textcolor}%
\pgfsetfillcolor{textcolor}%
\pgftext[x=3.326667in,y=1.758333in,,]{\color{textcolor}\rmfamily\fontsize{10.000000}{12.000000}\selectfont 20}%
\end{pgfscope}%
\begin{pgfscope}%
\definecolor{textcolor}{rgb}{1.000000,1.000000,1.000000}%
\pgfsetstrokecolor{textcolor}%
\pgfsetfillcolor{textcolor}%
\pgftext[x=3.830000in,y=1.758333in,,]{\color{textcolor}\rmfamily\fontsize{10.000000}{12.000000}\selectfont 17}%
\end{pgfscope}%
\begin{pgfscope}%
\definecolor{textcolor}{rgb}{0.150000,0.150000,0.150000}%
\pgfsetstrokecolor{textcolor}%
\pgfsetfillcolor{textcolor}%
\pgftext[x=4.333333in,y=1.758333in,,]{\color{textcolor}\rmfamily\fontsize{10.000000}{12.000000}\selectfont 4}%
\end{pgfscope}%
\begin{pgfscope}%
\definecolor{textcolor}{rgb}{0.150000,0.150000,0.150000}%
\pgfsetstrokecolor{textcolor}%
\pgfsetfillcolor{textcolor}%
\pgftext[x=1.816667in,y=1.255000in,,]{\color{textcolor}\rmfamily\fontsize{10.000000}{12.000000}\selectfont 1}%
\end{pgfscope}%
\begin{pgfscope}%
\definecolor{textcolor}{rgb}{0.150000,0.150000,0.150000}%
\pgfsetstrokecolor{textcolor}%
\pgfsetfillcolor{textcolor}%
\pgftext[x=2.320000in,y=1.255000in,,]{\color{textcolor}\rmfamily\fontsize{10.000000}{12.000000}\selectfont 1}%
\end{pgfscope}%
\begin{pgfscope}%
\definecolor{textcolor}{rgb}{0.150000,0.150000,0.150000}%
\pgfsetstrokecolor{textcolor}%
\pgfsetfillcolor{textcolor}%
\pgftext[x=2.823333in,y=1.255000in,,]{\color{textcolor}\rmfamily\fontsize{10.000000}{12.000000}\selectfont 0}%
\end{pgfscope}%
\begin{pgfscope}%
\definecolor{textcolor}{rgb}{0.150000,0.150000,0.150000}%
\pgfsetstrokecolor{textcolor}%
\pgfsetfillcolor{textcolor}%
\pgftext[x=3.326667in,y=1.255000in,,]{\color{textcolor}\rmfamily\fontsize{10.000000}{12.000000}\selectfont 0}%
\end{pgfscope}%
\begin{pgfscope}%
\definecolor{textcolor}{rgb}{1.000000,1.000000,1.000000}%
\pgfsetstrokecolor{textcolor}%
\pgfsetfillcolor{textcolor}%
\pgftext[x=3.830000in,y=1.255000in,,]{\color{textcolor}\rmfamily\fontsize{10.000000}{12.000000}\selectfont 16}%
\end{pgfscope}%
\begin{pgfscope}%
\definecolor{textcolor}{rgb}{0.150000,0.150000,0.150000}%
\pgfsetstrokecolor{textcolor}%
\pgfsetfillcolor{textcolor}%
\pgftext[x=4.333333in,y=1.255000in,,]{\color{textcolor}\rmfamily\fontsize{10.000000}{12.000000}\selectfont 1}%
\end{pgfscope}%
\begin{pgfscope}%
\definecolor{textcolor}{rgb}{0.150000,0.150000,0.150000}%
\pgfsetstrokecolor{textcolor}%
\pgfsetfillcolor{textcolor}%
\pgftext[x=1.816667in,y=0.751667in,,]{\color{textcolor}\rmfamily\fontsize{10.000000}{12.000000}\selectfont 1}%
\end{pgfscope}%
\begin{pgfscope}%
\definecolor{textcolor}{rgb}{0.150000,0.150000,0.150000}%
\pgfsetstrokecolor{textcolor}%
\pgfsetfillcolor{textcolor}%
\pgftext[x=2.320000in,y=0.751667in,,]{\color{textcolor}\rmfamily\fontsize{10.000000}{12.000000}\selectfont 0}%
\end{pgfscope}%
\begin{pgfscope}%
\definecolor{textcolor}{rgb}{0.150000,0.150000,0.150000}%
\pgfsetstrokecolor{textcolor}%
\pgfsetfillcolor{textcolor}%
\pgftext[x=2.823333in,y=0.751667in,,]{\color{textcolor}\rmfamily\fontsize{10.000000}{12.000000}\selectfont 0}%
\end{pgfscope}%
\begin{pgfscope}%
\definecolor{textcolor}{rgb}{0.150000,0.150000,0.150000}%
\pgfsetstrokecolor{textcolor}%
\pgfsetfillcolor{textcolor}%
\pgftext[x=3.326667in,y=0.751667in,,]{\color{textcolor}\rmfamily\fontsize{10.000000}{12.000000}\selectfont 0}%
\end{pgfscope}%
\begin{pgfscope}%
\definecolor{textcolor}{rgb}{1.000000,1.000000,1.000000}%
\pgfsetstrokecolor{textcolor}%
\pgfsetfillcolor{textcolor}%
\pgftext[x=3.830000in,y=0.751667in,,]{\color{textcolor}\rmfamily\fontsize{10.000000}{12.000000}\selectfont 16}%
\end{pgfscope}%
\begin{pgfscope}%
\definecolor{textcolor}{rgb}{1.000000,1.000000,1.000000}%
\pgfsetstrokecolor{textcolor}%
\pgfsetfillcolor{textcolor}%
\pgftext[x=4.333333in,y=0.751667in,,]{\color{textcolor}\rmfamily\fontsize{10.000000}{12.000000}\selectfont 9}%
\end{pgfscope}%
\begin{pgfscope}%
\definecolor{textcolor}{rgb}{0.000000,0.000000,0.000000}%
\pgfsetstrokecolor{textcolor}%
\pgfsetfillcolor{textcolor}%
\pgftext[x=3.075000in,y=3.686667in,,base]{\color{textcolor}\rmfamily\fontsize{18.000000}{21.600000}\selectfont External test set}%
\end{pgfscope}%
\end{pgfpicture}%
\makeatother%
\endgroup%
}
    \end{subfigure}%
    \begin{subfigure}[b]{.5\linewidth}
      \centering
      \resizebox{1.5\textwidth}{!}{%% Creator: Matplotlib, PGF backend
%%
%% To include the figure in your LaTeX document, write
%%   \input{<filename>.pgf}
%%
%% Make sure the required packages are loaded in your preamble
%%   \usepackage{pgf}
%%
%% Figures using additional raster images can only be included by \input if
%% they are in the same directory as the main LaTeX file. For loading figures
%% from other directories you can use the `import` package
%%   \usepackage{import}
%%
%% and then include the figures with
%%   \import{<path to file>}{<filename>.pgf}
%%
%% Matplotlib used the following preamble
%%
\begingroup%
\makeatletter%
\begin{pgfpicture}%
\pgfpathrectangle{\pgfpointorigin}{\pgfqpoint{6.000000in}{4.000000in}}%
\pgfusepath{use as bounding box, clip}%
\begin{pgfscope}%
\pgfsetbuttcap%
\pgfsetmiterjoin%
\pgfsetlinewidth{0.000000pt}%
\definecolor{currentstroke}{rgb}{1.000000,1.000000,1.000000}%
\pgfsetstrokecolor{currentstroke}%
\pgfsetstrokeopacity{0.000000}%
\pgfsetdash{}{0pt}%
\pgfpathmoveto{\pgfqpoint{0.000000in}{0.000000in}}%
\pgfpathlineto{\pgfqpoint{6.000000in}{0.000000in}}%
\pgfpathlineto{\pgfqpoint{6.000000in}{4.000000in}}%
\pgfpathlineto{\pgfqpoint{0.000000in}{4.000000in}}%
\pgfpathclose%
\pgfusepath{}%
\end{pgfscope}%
\begin{pgfscope}%
\pgfsetbuttcap%
\pgfsetmiterjoin%
\definecolor{currentfill}{rgb}{1.000000,1.000000,1.000000}%
\pgfsetfillcolor{currentfill}%
\pgfsetlinewidth{0.000000pt}%
\definecolor{currentstroke}{rgb}{0.000000,0.000000,0.000000}%
\pgfsetstrokecolor{currentstroke}%
\pgfsetstrokeopacity{0.000000}%
\pgfsetdash{}{0pt}%
\pgfpathmoveto{\pgfqpoint{1.565000in}{0.500000in}}%
\pgfpathlineto{\pgfqpoint{4.585000in}{0.500000in}}%
\pgfpathlineto{\pgfqpoint{4.585000in}{3.520000in}}%
\pgfpathlineto{\pgfqpoint{1.565000in}{3.520000in}}%
\pgfpathclose%
\pgfusepath{fill}%
\end{pgfscope}%
\begin{pgfscope}%
\pgfpathrectangle{\pgfqpoint{1.565000in}{0.500000in}}{\pgfqpoint{3.020000in}{3.020000in}}%
\pgfusepath{clip}%
\pgfsetbuttcap%
\pgfsetroundjoin%
\definecolor{currentfill}{rgb}{0.472656,0.031250,0.289062}%
\pgfsetfillcolor{currentfill}%
\pgfsetlinewidth{0.000000pt}%
\definecolor{currentstroke}{rgb}{1.000000,1.000000,1.000000}%
\pgfsetstrokecolor{currentstroke}%
\pgfsetdash{}{0pt}%
\pgfpathmoveto{\pgfqpoint{1.565000in}{3.520000in}}%
\pgfpathlineto{\pgfqpoint{2.068333in}{3.520000in}}%
\pgfpathlineto{\pgfqpoint{2.068333in}{3.016667in}}%
\pgfpathlineto{\pgfqpoint{1.565000in}{3.016667in}}%
\pgfpathlineto{\pgfqpoint{1.565000in}{3.520000in}}%
\pgfusepath{fill}%
\end{pgfscope}%
\begin{pgfscope}%
\pgfpathrectangle{\pgfqpoint{1.565000in}{0.500000in}}{\pgfqpoint{3.020000in}{3.020000in}}%
\pgfusepath{clip}%
\pgfsetbuttcap%
\pgfsetroundjoin%
\definecolor{currentfill}{rgb}{0.955040,0.920420,0.940640}%
\pgfsetfillcolor{currentfill}%
\pgfsetlinewidth{0.000000pt}%
\definecolor{currentstroke}{rgb}{1.000000,1.000000,1.000000}%
\pgfsetstrokecolor{currentstroke}%
\pgfsetdash{}{0pt}%
\pgfpathmoveto{\pgfqpoint{2.068333in}{3.520000in}}%
\pgfpathlineto{\pgfqpoint{2.571667in}{3.520000in}}%
\pgfpathlineto{\pgfqpoint{2.571667in}{3.016667in}}%
\pgfpathlineto{\pgfqpoint{2.068333in}{3.016667in}}%
\pgfpathlineto{\pgfqpoint{2.068333in}{3.520000in}}%
\pgfusepath{fill}%
\end{pgfscope}%
\begin{pgfscope}%
\pgfpathrectangle{\pgfqpoint{1.565000in}{0.500000in}}{\pgfqpoint{3.020000in}{3.020000in}}%
\pgfusepath{clip}%
\pgfsetbuttcap%
\pgfsetroundjoin%
\definecolor{currentfill}{rgb}{0.991988,0.988526,0.990548}%
\pgfsetfillcolor{currentfill}%
\pgfsetlinewidth{0.000000pt}%
\definecolor{currentstroke}{rgb}{1.000000,1.000000,1.000000}%
\pgfsetstrokecolor{currentstroke}%
\pgfsetdash{}{0pt}%
\pgfpathmoveto{\pgfqpoint{2.571667in}{3.520000in}}%
\pgfpathlineto{\pgfqpoint{3.075000in}{3.520000in}}%
\pgfpathlineto{\pgfqpoint{3.075000in}{3.016667in}}%
\pgfpathlineto{\pgfqpoint{2.571667in}{3.016667in}}%
\pgfpathlineto{\pgfqpoint{2.571667in}{3.520000in}}%
\pgfusepath{fill}%
\end{pgfscope}%
\begin{pgfscope}%
\pgfpathrectangle{\pgfqpoint{1.565000in}{0.500000in}}{\pgfqpoint{3.020000in}{3.020000in}}%
\pgfusepath{clip}%
\pgfsetbuttcap%
\pgfsetroundjoin%
\definecolor{currentfill}{rgb}{0.996094,0.996094,0.996094}%
\pgfsetfillcolor{currentfill}%
\pgfsetlinewidth{0.000000pt}%
\definecolor{currentstroke}{rgb}{1.000000,1.000000,1.000000}%
\pgfsetstrokecolor{currentstroke}%
\pgfsetdash{}{0pt}%
\pgfpathmoveto{\pgfqpoint{3.075000in}{3.520000in}}%
\pgfpathlineto{\pgfqpoint{3.578333in}{3.520000in}}%
\pgfpathlineto{\pgfqpoint{3.578333in}{3.016667in}}%
\pgfpathlineto{\pgfqpoint{3.075000in}{3.016667in}}%
\pgfpathlineto{\pgfqpoint{3.075000in}{3.520000in}}%
\pgfusepath{fill}%
\end{pgfscope}%
\begin{pgfscope}%
\pgfpathrectangle{\pgfqpoint{1.565000in}{0.500000in}}{\pgfqpoint{3.020000in}{3.020000in}}%
\pgfusepath{clip}%
\pgfsetbuttcap%
\pgfsetroundjoin%
\definecolor{currentfill}{rgb}{0.996094,0.996094,0.996094}%
\pgfsetfillcolor{currentfill}%
\pgfsetlinewidth{0.000000pt}%
\definecolor{currentstroke}{rgb}{1.000000,1.000000,1.000000}%
\pgfsetstrokecolor{currentstroke}%
\pgfsetdash{}{0pt}%
\pgfpathmoveto{\pgfqpoint{3.578333in}{3.520000in}}%
\pgfpathlineto{\pgfqpoint{4.081667in}{3.520000in}}%
\pgfpathlineto{\pgfqpoint{4.081667in}{3.016667in}}%
\pgfpathlineto{\pgfqpoint{3.578333in}{3.016667in}}%
\pgfpathlineto{\pgfqpoint{3.578333in}{3.520000in}}%
\pgfusepath{fill}%
\end{pgfscope}%
\begin{pgfscope}%
\pgfpathrectangle{\pgfqpoint{1.565000in}{0.500000in}}{\pgfqpoint{3.020000in}{3.020000in}}%
\pgfusepath{clip}%
\pgfsetbuttcap%
\pgfsetroundjoin%
\definecolor{currentfill}{rgb}{0.996094,0.996094,0.996094}%
\pgfsetfillcolor{currentfill}%
\pgfsetlinewidth{0.000000pt}%
\definecolor{currentstroke}{rgb}{1.000000,1.000000,1.000000}%
\pgfsetstrokecolor{currentstroke}%
\pgfsetdash{}{0pt}%
\pgfpathmoveto{\pgfqpoint{4.081667in}{3.520000in}}%
\pgfpathlineto{\pgfqpoint{4.585000in}{3.520000in}}%
\pgfpathlineto{\pgfqpoint{4.585000in}{3.016667in}}%
\pgfpathlineto{\pgfqpoint{4.081667in}{3.016667in}}%
\pgfpathlineto{\pgfqpoint{4.081667in}{3.520000in}}%
\pgfusepath{fill}%
\end{pgfscope}%
\begin{pgfscope}%
\pgfpathrectangle{\pgfqpoint{1.565000in}{0.500000in}}{\pgfqpoint{3.020000in}{3.020000in}}%
\pgfusepath{clip}%
\pgfsetbuttcap%
\pgfsetroundjoin%
\definecolor{currentfill}{rgb}{0.971461,0.950689,0.962822}%
\pgfsetfillcolor{currentfill}%
\pgfsetlinewidth{0.000000pt}%
\definecolor{currentstroke}{rgb}{1.000000,1.000000,1.000000}%
\pgfsetstrokecolor{currentstroke}%
\pgfsetdash{}{0pt}%
\pgfpathmoveto{\pgfqpoint{1.565000in}{3.016667in}}%
\pgfpathlineto{\pgfqpoint{2.068333in}{3.016667in}}%
\pgfpathlineto{\pgfqpoint{2.068333in}{2.513333in}}%
\pgfpathlineto{\pgfqpoint{1.565000in}{2.513333in}}%
\pgfpathlineto{\pgfqpoint{1.565000in}{3.016667in}}%
\pgfusepath{fill}%
\end{pgfscope}%
\begin{pgfscope}%
\pgfpathrectangle{\pgfqpoint{1.565000in}{0.500000in}}{\pgfqpoint{3.020000in}{3.020000in}}%
\pgfusepath{clip}%
\pgfsetbuttcap%
\pgfsetroundjoin%
\definecolor{currentfill}{rgb}{0.538343,0.152328,0.377788}%
\pgfsetfillcolor{currentfill}%
\pgfsetlinewidth{0.000000pt}%
\definecolor{currentstroke}{rgb}{1.000000,1.000000,1.000000}%
\pgfsetstrokecolor{currentstroke}%
\pgfsetdash{}{0pt}%
\pgfpathmoveto{\pgfqpoint{2.068333in}{3.016667in}}%
\pgfpathlineto{\pgfqpoint{2.571667in}{3.016667in}}%
\pgfpathlineto{\pgfqpoint{2.571667in}{2.513333in}}%
\pgfpathlineto{\pgfqpoint{2.068333in}{2.513333in}}%
\pgfpathlineto{\pgfqpoint{2.068333in}{3.016667in}}%
\pgfusepath{fill}%
\end{pgfscope}%
\begin{pgfscope}%
\pgfpathrectangle{\pgfqpoint{1.565000in}{0.500000in}}{\pgfqpoint{3.020000in}{3.020000in}}%
\pgfusepath{clip}%
\pgfsetbuttcap%
\pgfsetroundjoin%
\definecolor{currentfill}{rgb}{0.918091,0.852313,0.890732}%
\pgfsetfillcolor{currentfill}%
\pgfsetlinewidth{0.000000pt}%
\definecolor{currentstroke}{rgb}{1.000000,1.000000,1.000000}%
\pgfsetstrokecolor{currentstroke}%
\pgfsetdash{}{0pt}%
\pgfpathmoveto{\pgfqpoint{2.571667in}{3.016667in}}%
\pgfpathlineto{\pgfqpoint{3.075000in}{3.016667in}}%
\pgfpathlineto{\pgfqpoint{3.075000in}{2.513333in}}%
\pgfpathlineto{\pgfqpoint{2.571667in}{2.513333in}}%
\pgfpathlineto{\pgfqpoint{2.571667in}{3.016667in}}%
\pgfusepath{fill}%
\end{pgfscope}%
\begin{pgfscope}%
\pgfpathrectangle{\pgfqpoint{1.565000in}{0.500000in}}{\pgfqpoint{3.020000in}{3.020000in}}%
\pgfusepath{clip}%
\pgfsetbuttcap%
\pgfsetroundjoin%
\definecolor{currentfill}{rgb}{0.987883,0.980959,0.985003}%
\pgfsetfillcolor{currentfill}%
\pgfsetlinewidth{0.000000pt}%
\definecolor{currentstroke}{rgb}{1.000000,1.000000,1.000000}%
\pgfsetstrokecolor{currentstroke}%
\pgfsetdash{}{0pt}%
\pgfpathmoveto{\pgfqpoint{3.075000in}{3.016667in}}%
\pgfpathlineto{\pgfqpoint{3.578333in}{3.016667in}}%
\pgfpathlineto{\pgfqpoint{3.578333in}{2.513333in}}%
\pgfpathlineto{\pgfqpoint{3.075000in}{2.513333in}}%
\pgfpathlineto{\pgfqpoint{3.075000in}{3.016667in}}%
\pgfusepath{fill}%
\end{pgfscope}%
\begin{pgfscope}%
\pgfpathrectangle{\pgfqpoint{1.565000in}{0.500000in}}{\pgfqpoint{3.020000in}{3.020000in}}%
\pgfusepath{clip}%
\pgfsetbuttcap%
\pgfsetroundjoin%
\definecolor{currentfill}{rgb}{0.996094,0.996094,0.996094}%
\pgfsetfillcolor{currentfill}%
\pgfsetlinewidth{0.000000pt}%
\definecolor{currentstroke}{rgb}{1.000000,1.000000,1.000000}%
\pgfsetstrokecolor{currentstroke}%
\pgfsetdash{}{0pt}%
\pgfpathmoveto{\pgfqpoint{3.578333in}{3.016667in}}%
\pgfpathlineto{\pgfqpoint{4.081667in}{3.016667in}}%
\pgfpathlineto{\pgfqpoint{4.081667in}{2.513333in}}%
\pgfpathlineto{\pgfqpoint{3.578333in}{2.513333in}}%
\pgfpathlineto{\pgfqpoint{3.578333in}{3.016667in}}%
\pgfusepath{fill}%
\end{pgfscope}%
\begin{pgfscope}%
\pgfpathrectangle{\pgfqpoint{1.565000in}{0.500000in}}{\pgfqpoint{3.020000in}{3.020000in}}%
\pgfusepath{clip}%
\pgfsetbuttcap%
\pgfsetroundjoin%
\definecolor{currentfill}{rgb}{0.996094,0.996094,0.996094}%
\pgfsetfillcolor{currentfill}%
\pgfsetlinewidth{0.000000pt}%
\definecolor{currentstroke}{rgb}{1.000000,1.000000,1.000000}%
\pgfsetstrokecolor{currentstroke}%
\pgfsetdash{}{0pt}%
\pgfpathmoveto{\pgfqpoint{4.081667in}{3.016667in}}%
\pgfpathlineto{\pgfqpoint{4.585000in}{3.016667in}}%
\pgfpathlineto{\pgfqpoint{4.585000in}{2.513333in}}%
\pgfpathlineto{\pgfqpoint{4.081667in}{2.513333in}}%
\pgfpathlineto{\pgfqpoint{4.081667in}{3.016667in}}%
\pgfusepath{fill}%
\end{pgfscope}%
\begin{pgfscope}%
\pgfpathrectangle{\pgfqpoint{1.565000in}{0.500000in}}{\pgfqpoint{3.020000in}{3.020000in}}%
\pgfusepath{clip}%
\pgfsetbuttcap%
\pgfsetroundjoin%
\definecolor{currentfill}{rgb}{0.979672,0.965824,0.973912}%
\pgfsetfillcolor{currentfill}%
\pgfsetlinewidth{0.000000pt}%
\definecolor{currentstroke}{rgb}{1.000000,1.000000,1.000000}%
\pgfsetstrokecolor{currentstroke}%
\pgfsetdash{}{0pt}%
\pgfpathmoveto{\pgfqpoint{1.565000in}{2.513333in}}%
\pgfpathlineto{\pgfqpoint{2.068333in}{2.513333in}}%
\pgfpathlineto{\pgfqpoint{2.068333in}{2.010000in}}%
\pgfpathlineto{\pgfqpoint{1.565000in}{2.010000in}}%
\pgfpathlineto{\pgfqpoint{1.565000in}{2.513333in}}%
\pgfusepath{fill}%
\end{pgfscope}%
\begin{pgfscope}%
\pgfpathrectangle{\pgfqpoint{1.565000in}{0.500000in}}{\pgfqpoint{3.020000in}{3.020000in}}%
\pgfusepath{clip}%
\pgfsetbuttcap%
\pgfsetroundjoin%
\definecolor{currentfill}{rgb}{0.860616,0.746369,0.813097}%
\pgfsetfillcolor{currentfill}%
\pgfsetlinewidth{0.000000pt}%
\definecolor{currentstroke}{rgb}{1.000000,1.000000,1.000000}%
\pgfsetstrokecolor{currentstroke}%
\pgfsetdash{}{0pt}%
\pgfpathmoveto{\pgfqpoint{2.068333in}{2.513333in}}%
\pgfpathlineto{\pgfqpoint{2.571667in}{2.513333in}}%
\pgfpathlineto{\pgfqpoint{2.571667in}{2.010000in}}%
\pgfpathlineto{\pgfqpoint{2.068333in}{2.010000in}}%
\pgfpathlineto{\pgfqpoint{2.068333in}{2.513333in}}%
\pgfusepath{fill}%
\end{pgfscope}%
\begin{pgfscope}%
\pgfpathrectangle{\pgfqpoint{1.565000in}{0.500000in}}{\pgfqpoint{3.020000in}{3.020000in}}%
\pgfusepath{clip}%
\pgfsetbuttcap%
\pgfsetroundjoin%
\definecolor{currentfill}{rgb}{0.669715,0.394485,0.555239}%
\pgfsetfillcolor{currentfill}%
\pgfsetlinewidth{0.000000pt}%
\definecolor{currentstroke}{rgb}{1.000000,1.000000,1.000000}%
\pgfsetstrokecolor{currentstroke}%
\pgfsetdash{}{0pt}%
\pgfpathmoveto{\pgfqpoint{2.571667in}{2.513333in}}%
\pgfpathlineto{\pgfqpoint{3.075000in}{2.513333in}}%
\pgfpathlineto{\pgfqpoint{3.075000in}{2.010000in}}%
\pgfpathlineto{\pgfqpoint{2.571667in}{2.010000in}}%
\pgfpathlineto{\pgfqpoint{2.571667in}{2.513333in}}%
\pgfusepath{fill}%
\end{pgfscope}%
\begin{pgfscope}%
\pgfpathrectangle{\pgfqpoint{1.565000in}{0.500000in}}{\pgfqpoint{3.020000in}{3.020000in}}%
\pgfusepath{clip}%
\pgfsetbuttcap%
\pgfsetroundjoin%
\definecolor{currentfill}{rgb}{0.942724,0.897718,0.924004}%
\pgfsetfillcolor{currentfill}%
\pgfsetlinewidth{0.000000pt}%
\definecolor{currentstroke}{rgb}{1.000000,1.000000,1.000000}%
\pgfsetstrokecolor{currentstroke}%
\pgfsetdash{}{0pt}%
\pgfpathmoveto{\pgfqpoint{3.075000in}{2.513333in}}%
\pgfpathlineto{\pgfqpoint{3.578333in}{2.513333in}}%
\pgfpathlineto{\pgfqpoint{3.578333in}{2.010000in}}%
\pgfpathlineto{\pgfqpoint{3.075000in}{2.010000in}}%
\pgfpathlineto{\pgfqpoint{3.075000in}{2.513333in}}%
\pgfusepath{fill}%
\end{pgfscope}%
\begin{pgfscope}%
\pgfpathrectangle{\pgfqpoint{1.565000in}{0.500000in}}{\pgfqpoint{3.020000in}{3.020000in}}%
\pgfusepath{clip}%
\pgfsetbuttcap%
\pgfsetroundjoin%
\definecolor{currentfill}{rgb}{0.969409,0.946906,0.960049}%
\pgfsetfillcolor{currentfill}%
\pgfsetlinewidth{0.000000pt}%
\definecolor{currentstroke}{rgb}{1.000000,1.000000,1.000000}%
\pgfsetstrokecolor{currentstroke}%
\pgfsetdash{}{0pt}%
\pgfpathmoveto{\pgfqpoint{3.578333in}{2.513333in}}%
\pgfpathlineto{\pgfqpoint{4.081667in}{2.513333in}}%
\pgfpathlineto{\pgfqpoint{4.081667in}{2.010000in}}%
\pgfpathlineto{\pgfqpoint{3.578333in}{2.010000in}}%
\pgfpathlineto{\pgfqpoint{3.578333in}{2.513333in}}%
\pgfusepath{fill}%
\end{pgfscope}%
\begin{pgfscope}%
\pgfpathrectangle{\pgfqpoint{1.565000in}{0.500000in}}{\pgfqpoint{3.020000in}{3.020000in}}%
\pgfusepath{clip}%
\pgfsetbuttcap%
\pgfsetroundjoin%
\definecolor{currentfill}{rgb}{0.987883,0.980959,0.985003}%
\pgfsetfillcolor{currentfill}%
\pgfsetlinewidth{0.000000pt}%
\definecolor{currentstroke}{rgb}{1.000000,1.000000,1.000000}%
\pgfsetstrokecolor{currentstroke}%
\pgfsetdash{}{0pt}%
\pgfpathmoveto{\pgfqpoint{4.081667in}{2.513333in}}%
\pgfpathlineto{\pgfqpoint{4.585000in}{2.513333in}}%
\pgfpathlineto{\pgfqpoint{4.585000in}{2.010000in}}%
\pgfpathlineto{\pgfqpoint{4.081667in}{2.010000in}}%
\pgfpathlineto{\pgfqpoint{4.081667in}{2.513333in}}%
\pgfusepath{fill}%
\end{pgfscope}%
\begin{pgfscope}%
\pgfpathrectangle{\pgfqpoint{1.565000in}{0.500000in}}{\pgfqpoint{3.020000in}{3.020000in}}%
\pgfusepath{clip}%
\pgfsetbuttcap%
\pgfsetroundjoin%
\definecolor{currentfill}{rgb}{0.985830,0.977175,0.982230}%
\pgfsetfillcolor{currentfill}%
\pgfsetlinewidth{0.000000pt}%
\definecolor{currentstroke}{rgb}{1.000000,1.000000,1.000000}%
\pgfsetstrokecolor{currentstroke}%
\pgfsetdash{}{0pt}%
\pgfpathmoveto{\pgfqpoint{1.565000in}{2.010000in}}%
\pgfpathlineto{\pgfqpoint{2.068333in}{2.010000in}}%
\pgfpathlineto{\pgfqpoint{2.068333in}{1.506667in}}%
\pgfpathlineto{\pgfqpoint{1.565000in}{1.506667in}}%
\pgfpathlineto{\pgfqpoint{1.565000in}{2.010000in}}%
\pgfusepath{fill}%
\end{pgfscope}%
\begin{pgfscope}%
\pgfpathrectangle{\pgfqpoint{1.565000in}{0.500000in}}{\pgfqpoint{3.020000in}{3.020000in}}%
\pgfusepath{clip}%
\pgfsetbuttcap%
\pgfsetroundjoin%
\definecolor{currentfill}{rgb}{0.996094,0.996094,0.996094}%
\pgfsetfillcolor{currentfill}%
\pgfsetlinewidth{0.000000pt}%
\definecolor{currentstroke}{rgb}{1.000000,1.000000,1.000000}%
\pgfsetstrokecolor{currentstroke}%
\pgfsetdash{}{0pt}%
\pgfpathmoveto{\pgfqpoint{2.068333in}{2.010000in}}%
\pgfpathlineto{\pgfqpoint{2.571667in}{2.010000in}}%
\pgfpathlineto{\pgfqpoint{2.571667in}{1.506667in}}%
\pgfpathlineto{\pgfqpoint{2.068333in}{1.506667in}}%
\pgfpathlineto{\pgfqpoint{2.068333in}{2.010000in}}%
\pgfusepath{fill}%
\end{pgfscope}%
\begin{pgfscope}%
\pgfpathrectangle{\pgfqpoint{1.565000in}{0.500000in}}{\pgfqpoint{3.020000in}{3.020000in}}%
\pgfusepath{clip}%
\pgfsetbuttcap%
\pgfsetroundjoin%
\definecolor{currentfill}{rgb}{0.916039,0.848529,0.887960}%
\pgfsetfillcolor{currentfill}%
\pgfsetlinewidth{0.000000pt}%
\definecolor{currentstroke}{rgb}{1.000000,1.000000,1.000000}%
\pgfsetstrokecolor{currentstroke}%
\pgfsetdash{}{0pt}%
\pgfpathmoveto{\pgfqpoint{2.571667in}{2.010000in}}%
\pgfpathlineto{\pgfqpoint{3.075000in}{2.010000in}}%
\pgfpathlineto{\pgfqpoint{3.075000in}{1.506667in}}%
\pgfpathlineto{\pgfqpoint{2.571667in}{1.506667in}}%
\pgfpathlineto{\pgfqpoint{2.571667in}{2.010000in}}%
\pgfusepath{fill}%
\end{pgfscope}%
\begin{pgfscope}%
\pgfpathrectangle{\pgfqpoint{1.565000in}{0.500000in}}{\pgfqpoint{3.020000in}{3.020000in}}%
\pgfusepath{clip}%
\pgfsetbuttcap%
\pgfsetroundjoin%
\definecolor{currentfill}{rgb}{0.764139,0.568536,0.682782}%
\pgfsetfillcolor{currentfill}%
\pgfsetlinewidth{0.000000pt}%
\definecolor{currentstroke}{rgb}{1.000000,1.000000,1.000000}%
\pgfsetstrokecolor{currentstroke}%
\pgfsetdash{}{0pt}%
\pgfpathmoveto{\pgfqpoint{3.075000in}{2.010000in}}%
\pgfpathlineto{\pgfqpoint{3.578333in}{2.010000in}}%
\pgfpathlineto{\pgfqpoint{3.578333in}{1.506667in}}%
\pgfpathlineto{\pgfqpoint{3.075000in}{1.506667in}}%
\pgfpathlineto{\pgfqpoint{3.075000in}{2.010000in}}%
\pgfusepath{fill}%
\end{pgfscope}%
\begin{pgfscope}%
\pgfpathrectangle{\pgfqpoint{1.565000in}{0.500000in}}{\pgfqpoint{3.020000in}{3.020000in}}%
\pgfusepath{clip}%
\pgfsetbuttcap%
\pgfsetroundjoin%
\definecolor{currentfill}{rgb}{0.799035,0.632858,0.729917}%
\pgfsetfillcolor{currentfill}%
\pgfsetlinewidth{0.000000pt}%
\definecolor{currentstroke}{rgb}{1.000000,1.000000,1.000000}%
\pgfsetstrokecolor{currentstroke}%
\pgfsetdash{}{0pt}%
\pgfpathmoveto{\pgfqpoint{3.578333in}{2.010000in}}%
\pgfpathlineto{\pgfqpoint{4.081667in}{2.010000in}}%
\pgfpathlineto{\pgfqpoint{4.081667in}{1.506667in}}%
\pgfpathlineto{\pgfqpoint{3.578333in}{1.506667in}}%
\pgfpathlineto{\pgfqpoint{3.578333in}{2.010000in}}%
\pgfusepath{fill}%
\end{pgfscope}%
\begin{pgfscope}%
\pgfpathrectangle{\pgfqpoint{1.565000in}{0.500000in}}{\pgfqpoint{3.020000in}{3.020000in}}%
\pgfusepath{clip}%
\pgfsetbuttcap%
\pgfsetroundjoin%
\definecolor{currentfill}{rgb}{0.950934,0.912852,0.935095}%
\pgfsetfillcolor{currentfill}%
\pgfsetlinewidth{0.000000pt}%
\definecolor{currentstroke}{rgb}{1.000000,1.000000,1.000000}%
\pgfsetstrokecolor{currentstroke}%
\pgfsetdash{}{0pt}%
\pgfpathmoveto{\pgfqpoint{4.081667in}{2.010000in}}%
\pgfpathlineto{\pgfqpoint{4.585000in}{2.010000in}}%
\pgfpathlineto{\pgfqpoint{4.585000in}{1.506667in}}%
\pgfpathlineto{\pgfqpoint{4.081667in}{1.506667in}}%
\pgfpathlineto{\pgfqpoint{4.081667in}{2.010000in}}%
\pgfusepath{fill}%
\end{pgfscope}%
\begin{pgfscope}%
\pgfpathrectangle{\pgfqpoint{1.565000in}{0.500000in}}{\pgfqpoint{3.020000in}{3.020000in}}%
\pgfusepath{clip}%
\pgfsetbuttcap%
\pgfsetroundjoin%
\definecolor{currentfill}{rgb}{0.967356,0.943122,0.957276}%
\pgfsetfillcolor{currentfill}%
\pgfsetlinewidth{0.000000pt}%
\definecolor{currentstroke}{rgb}{1.000000,1.000000,1.000000}%
\pgfsetstrokecolor{currentstroke}%
\pgfsetdash{}{0pt}%
\pgfpathmoveto{\pgfqpoint{1.565000in}{1.506667in}}%
\pgfpathlineto{\pgfqpoint{2.068333in}{1.506667in}}%
\pgfpathlineto{\pgfqpoint{2.068333in}{1.003333in}}%
\pgfpathlineto{\pgfqpoint{1.565000in}{1.003333in}}%
\pgfpathlineto{\pgfqpoint{1.565000in}{1.506667in}}%
\pgfusepath{fill}%
\end{pgfscope}%
\begin{pgfscope}%
\pgfpathrectangle{\pgfqpoint{1.565000in}{0.500000in}}{\pgfqpoint{3.020000in}{3.020000in}}%
\pgfusepath{clip}%
\pgfsetbuttcap%
\pgfsetroundjoin%
\definecolor{currentfill}{rgb}{0.967356,0.943122,0.957276}%
\pgfsetfillcolor{currentfill}%
\pgfsetlinewidth{0.000000pt}%
\definecolor{currentstroke}{rgb}{1.000000,1.000000,1.000000}%
\pgfsetstrokecolor{currentstroke}%
\pgfsetdash{}{0pt}%
\pgfpathmoveto{\pgfqpoint{2.068333in}{1.506667in}}%
\pgfpathlineto{\pgfqpoint{2.571667in}{1.506667in}}%
\pgfpathlineto{\pgfqpoint{2.571667in}{1.003333in}}%
\pgfpathlineto{\pgfqpoint{2.068333in}{1.003333in}}%
\pgfpathlineto{\pgfqpoint{2.068333in}{1.506667in}}%
\pgfusepath{fill}%
\end{pgfscope}%
\begin{pgfscope}%
\pgfpathrectangle{\pgfqpoint{1.565000in}{0.500000in}}{\pgfqpoint{3.020000in}{3.020000in}}%
\pgfusepath{clip}%
\pgfsetbuttcap%
\pgfsetroundjoin%
\definecolor{currentfill}{rgb}{0.996094,0.996094,0.996094}%
\pgfsetfillcolor{currentfill}%
\pgfsetlinewidth{0.000000pt}%
\definecolor{currentstroke}{rgb}{1.000000,1.000000,1.000000}%
\pgfsetstrokecolor{currentstroke}%
\pgfsetdash{}{0pt}%
\pgfpathmoveto{\pgfqpoint{2.571667in}{1.506667in}}%
\pgfpathlineto{\pgfqpoint{3.075000in}{1.506667in}}%
\pgfpathlineto{\pgfqpoint{3.075000in}{1.003333in}}%
\pgfpathlineto{\pgfqpoint{2.571667in}{1.003333in}}%
\pgfpathlineto{\pgfqpoint{2.571667in}{1.506667in}}%
\pgfusepath{fill}%
\end{pgfscope}%
\begin{pgfscope}%
\pgfpathrectangle{\pgfqpoint{1.565000in}{0.500000in}}{\pgfqpoint{3.020000in}{3.020000in}}%
\pgfusepath{clip}%
\pgfsetbuttcap%
\pgfsetroundjoin%
\definecolor{currentfill}{rgb}{0.996094,0.996094,0.996094}%
\pgfsetfillcolor{currentfill}%
\pgfsetlinewidth{0.000000pt}%
\definecolor{currentstroke}{rgb}{1.000000,1.000000,1.000000}%
\pgfsetstrokecolor{currentstroke}%
\pgfsetdash{}{0pt}%
\pgfpathmoveto{\pgfqpoint{3.075000in}{1.506667in}}%
\pgfpathlineto{\pgfqpoint{3.578333in}{1.506667in}}%
\pgfpathlineto{\pgfqpoint{3.578333in}{1.003333in}}%
\pgfpathlineto{\pgfqpoint{3.075000in}{1.003333in}}%
\pgfpathlineto{\pgfqpoint{3.075000in}{1.506667in}}%
\pgfusepath{fill}%
\end{pgfscope}%
\begin{pgfscope}%
\pgfpathrectangle{\pgfqpoint{1.565000in}{0.500000in}}{\pgfqpoint{3.020000in}{3.020000in}}%
\pgfusepath{clip}%
\pgfsetbuttcap%
\pgfsetroundjoin%
\definecolor{currentfill}{rgb}{0.513710,0.106924,0.344516}%
\pgfsetfillcolor{currentfill}%
\pgfsetlinewidth{0.000000pt}%
\definecolor{currentstroke}{rgb}{1.000000,1.000000,1.000000}%
\pgfsetstrokecolor{currentstroke}%
\pgfsetdash{}{0pt}%
\pgfpathmoveto{\pgfqpoint{3.578333in}{1.506667in}}%
\pgfpathlineto{\pgfqpoint{4.081667in}{1.506667in}}%
\pgfpathlineto{\pgfqpoint{4.081667in}{1.003333in}}%
\pgfpathlineto{\pgfqpoint{3.578333in}{1.003333in}}%
\pgfpathlineto{\pgfqpoint{3.578333in}{1.506667in}}%
\pgfusepath{fill}%
\end{pgfscope}%
\begin{pgfscope}%
\pgfpathrectangle{\pgfqpoint{1.565000in}{0.500000in}}{\pgfqpoint{3.020000in}{3.020000in}}%
\pgfusepath{clip}%
\pgfsetbuttcap%
\pgfsetroundjoin%
\definecolor{currentfill}{rgb}{0.967356,0.943122,0.957276}%
\pgfsetfillcolor{currentfill}%
\pgfsetlinewidth{0.000000pt}%
\definecolor{currentstroke}{rgb}{1.000000,1.000000,1.000000}%
\pgfsetstrokecolor{currentstroke}%
\pgfsetdash{}{0pt}%
\pgfpathmoveto{\pgfqpoint{4.081667in}{1.506667in}}%
\pgfpathlineto{\pgfqpoint{4.585000in}{1.506667in}}%
\pgfpathlineto{\pgfqpoint{4.585000in}{1.003333in}}%
\pgfpathlineto{\pgfqpoint{4.081667in}{1.003333in}}%
\pgfpathlineto{\pgfqpoint{4.081667in}{1.506667in}}%
\pgfusepath{fill}%
\end{pgfscope}%
\begin{pgfscope}%
\pgfpathrectangle{\pgfqpoint{1.565000in}{0.500000in}}{\pgfqpoint{3.020000in}{3.020000in}}%
\pgfusepath{clip}%
\pgfsetbuttcap%
\pgfsetroundjoin%
\definecolor{currentfill}{rgb}{0.975567,0.958257,0.968367}%
\pgfsetfillcolor{currentfill}%
\pgfsetlinewidth{0.000000pt}%
\definecolor{currentstroke}{rgb}{1.000000,1.000000,1.000000}%
\pgfsetstrokecolor{currentstroke}%
\pgfsetdash{}{0pt}%
\pgfpathmoveto{\pgfqpoint{1.565000in}{1.003333in}}%
\pgfpathlineto{\pgfqpoint{2.068333in}{1.003333in}}%
\pgfpathlineto{\pgfqpoint{2.068333in}{0.500000in}}%
\pgfpathlineto{\pgfqpoint{1.565000in}{0.500000in}}%
\pgfpathlineto{\pgfqpoint{1.565000in}{1.003333in}}%
\pgfusepath{fill}%
\end{pgfscope}%
\begin{pgfscope}%
\pgfpathrectangle{\pgfqpoint{1.565000in}{0.500000in}}{\pgfqpoint{3.020000in}{3.020000in}}%
\pgfusepath{clip}%
\pgfsetbuttcap%
\pgfsetroundjoin%
\definecolor{currentfill}{rgb}{0.996094,0.996094,0.996094}%
\pgfsetfillcolor{currentfill}%
\pgfsetlinewidth{0.000000pt}%
\definecolor{currentstroke}{rgb}{1.000000,1.000000,1.000000}%
\pgfsetstrokecolor{currentstroke}%
\pgfsetdash{}{0pt}%
\pgfpathmoveto{\pgfqpoint{2.068333in}{1.003333in}}%
\pgfpathlineto{\pgfqpoint{2.571667in}{1.003333in}}%
\pgfpathlineto{\pgfqpoint{2.571667in}{0.500000in}}%
\pgfpathlineto{\pgfqpoint{2.068333in}{0.500000in}}%
\pgfpathlineto{\pgfqpoint{2.068333in}{1.003333in}}%
\pgfusepath{fill}%
\end{pgfscope}%
\begin{pgfscope}%
\pgfpathrectangle{\pgfqpoint{1.565000in}{0.500000in}}{\pgfqpoint{3.020000in}{3.020000in}}%
\pgfusepath{clip}%
\pgfsetbuttcap%
\pgfsetroundjoin%
\definecolor{currentfill}{rgb}{0.996094,0.996094,0.996094}%
\pgfsetfillcolor{currentfill}%
\pgfsetlinewidth{0.000000pt}%
\definecolor{currentstroke}{rgb}{1.000000,1.000000,1.000000}%
\pgfsetstrokecolor{currentstroke}%
\pgfsetdash{}{0pt}%
\pgfpathmoveto{\pgfqpoint{2.571667in}{1.003333in}}%
\pgfpathlineto{\pgfqpoint{3.075000in}{1.003333in}}%
\pgfpathlineto{\pgfqpoint{3.075000in}{0.500000in}}%
\pgfpathlineto{\pgfqpoint{2.571667in}{0.500000in}}%
\pgfpathlineto{\pgfqpoint{2.571667in}{1.003333in}}%
\pgfusepath{fill}%
\end{pgfscope}%
\begin{pgfscope}%
\pgfpathrectangle{\pgfqpoint{1.565000in}{0.500000in}}{\pgfqpoint{3.020000in}{3.020000in}}%
\pgfusepath{clip}%
\pgfsetbuttcap%
\pgfsetroundjoin%
\definecolor{currentfill}{rgb}{0.996094,0.996094,0.996094}%
\pgfsetfillcolor{currentfill}%
\pgfsetlinewidth{0.000000pt}%
\definecolor{currentstroke}{rgb}{1.000000,1.000000,1.000000}%
\pgfsetstrokecolor{currentstroke}%
\pgfsetdash{}{0pt}%
\pgfpathmoveto{\pgfqpoint{3.075000in}{1.003333in}}%
\pgfpathlineto{\pgfqpoint{3.578333in}{1.003333in}}%
\pgfpathlineto{\pgfqpoint{3.578333in}{0.500000in}}%
\pgfpathlineto{\pgfqpoint{3.075000in}{0.500000in}}%
\pgfpathlineto{\pgfqpoint{3.075000in}{1.003333in}}%
\pgfusepath{fill}%
\end{pgfscope}%
\begin{pgfscope}%
\pgfpathrectangle{\pgfqpoint{1.565000in}{0.500000in}}{\pgfqpoint{3.020000in}{3.020000in}}%
\pgfusepath{clip}%
\pgfsetbuttcap%
\pgfsetroundjoin%
\definecolor{currentfill}{rgb}{0.645083,0.349081,0.521967}%
\pgfsetfillcolor{currentfill}%
\pgfsetlinewidth{0.000000pt}%
\definecolor{currentstroke}{rgb}{1.000000,1.000000,1.000000}%
\pgfsetstrokecolor{currentstroke}%
\pgfsetdash{}{0pt}%
\pgfpathmoveto{\pgfqpoint{3.578333in}{1.003333in}}%
\pgfpathlineto{\pgfqpoint{4.081667in}{1.003333in}}%
\pgfpathlineto{\pgfqpoint{4.081667in}{0.500000in}}%
\pgfpathlineto{\pgfqpoint{3.578333in}{0.500000in}}%
\pgfpathlineto{\pgfqpoint{3.578333in}{1.003333in}}%
\pgfusepath{fill}%
\end{pgfscope}%
\begin{pgfscope}%
\pgfpathrectangle{\pgfqpoint{1.565000in}{0.500000in}}{\pgfqpoint{3.020000in}{3.020000in}}%
\pgfusepath{clip}%
\pgfsetbuttcap%
\pgfsetroundjoin%
\definecolor{currentfill}{rgb}{0.799035,0.632858,0.729917}%
\pgfsetfillcolor{currentfill}%
\pgfsetlinewidth{0.000000pt}%
\definecolor{currentstroke}{rgb}{1.000000,1.000000,1.000000}%
\pgfsetstrokecolor{currentstroke}%
\pgfsetdash{}{0pt}%
\pgfpathmoveto{\pgfqpoint{4.081667in}{1.003333in}}%
\pgfpathlineto{\pgfqpoint{4.585000in}{1.003333in}}%
\pgfpathlineto{\pgfqpoint{4.585000in}{0.500000in}}%
\pgfpathlineto{\pgfqpoint{4.081667in}{0.500000in}}%
\pgfpathlineto{\pgfqpoint{4.081667in}{1.003333in}}%
\pgfusepath{fill}%
\end{pgfscope}%
\begin{pgfscope}%
\definecolor{textcolor}{rgb}{0.000000,0.000000,0.000000}%
\pgfsetstrokecolor{textcolor}%
\pgfsetfillcolor{textcolor}%
\pgftext[x=1.816667in,y=0.402778in,,top]{\color{textcolor}\rmfamily\fontsize{10.000000}{12.000000}\selectfont 0}%
\end{pgfscope}%
\begin{pgfscope}%
\definecolor{textcolor}{rgb}{0.000000,0.000000,0.000000}%
\pgfsetstrokecolor{textcolor}%
\pgfsetfillcolor{textcolor}%
\pgftext[x=2.320000in,y=0.402778in,,top]{\color{textcolor}\rmfamily\fontsize{10.000000}{12.000000}\selectfont 1}%
\end{pgfscope}%
\begin{pgfscope}%
\definecolor{textcolor}{rgb}{0.000000,0.000000,0.000000}%
\pgfsetstrokecolor{textcolor}%
\pgfsetfillcolor{textcolor}%
\pgftext[x=2.823333in,y=0.402778in,,top]{\color{textcolor}\rmfamily\fontsize{10.000000}{12.000000}\selectfont 2}%
\end{pgfscope}%
\begin{pgfscope}%
\definecolor{textcolor}{rgb}{0.000000,0.000000,0.000000}%
\pgfsetstrokecolor{textcolor}%
\pgfsetfillcolor{textcolor}%
\pgftext[x=3.326667in,y=0.402778in,,top]{\color{textcolor}\rmfamily\fontsize{10.000000}{12.000000}\selectfont 3}%
\end{pgfscope}%
\begin{pgfscope}%
\definecolor{textcolor}{rgb}{0.000000,0.000000,0.000000}%
\pgfsetstrokecolor{textcolor}%
\pgfsetfillcolor{textcolor}%
\pgftext[x=3.830000in,y=0.402778in,,top]{\color{textcolor}\rmfamily\fontsize{10.000000}{12.000000}\selectfont 4}%
\end{pgfscope}%
\begin{pgfscope}%
\definecolor{textcolor}{rgb}{0.000000,0.000000,0.000000}%
\pgfsetstrokecolor{textcolor}%
\pgfsetfillcolor{textcolor}%
\pgftext[x=4.333333in,y=0.402778in,,top]{\color{textcolor}\rmfamily\fontsize{10.000000}{12.000000}\selectfont 5}%
\end{pgfscope}%
\begin{pgfscope}%
\definecolor{textcolor}{rgb}{0.000000,0.000000,0.000000}%
\pgfsetstrokecolor{textcolor}%
\pgfsetfillcolor{textcolor}%
\pgftext[x=3.075000in,y=0.195988in,,top]{\color{textcolor}\rmfamily\fontsize{15.000000}{18.000000}\selectfont Predicted ISUP grade}%
\end{pgfscope}%
\begin{pgfscope}%
\definecolor{textcolor}{rgb}{0.000000,0.000000,0.000000}%
\pgfsetstrokecolor{textcolor}%
\pgfsetfillcolor{textcolor}%
\pgftext[x=1.440772in, y=3.247114in, left, base,rotate=90.000000]{\color{textcolor}\rmfamily\fontsize{10.000000}{12.000000}\selectfont 0}%
\end{pgfscope}%
\begin{pgfscope}%
\definecolor{textcolor}{rgb}{0.000000,0.000000,0.000000}%
\pgfsetstrokecolor{textcolor}%
\pgfsetfillcolor{textcolor}%
\pgftext[x=1.440772in, y=2.743781in, left, base,rotate=90.000000]{\color{textcolor}\rmfamily\fontsize{10.000000}{12.000000}\selectfont 1}%
\end{pgfscope}%
\begin{pgfscope}%
\definecolor{textcolor}{rgb}{0.000000,0.000000,0.000000}%
\pgfsetstrokecolor{textcolor}%
\pgfsetfillcolor{textcolor}%
\pgftext[x=1.440772in, y=2.240447in, left, base,rotate=90.000000]{\color{textcolor}\rmfamily\fontsize{10.000000}{12.000000}\selectfont 2}%
\end{pgfscope}%
\begin{pgfscope}%
\definecolor{textcolor}{rgb}{0.000000,0.000000,0.000000}%
\pgfsetstrokecolor{textcolor}%
\pgfsetfillcolor{textcolor}%
\pgftext[x=1.440772in, y=1.737114in, left, base,rotate=90.000000]{\color{textcolor}\rmfamily\fontsize{10.000000}{12.000000}\selectfont 3}%
\end{pgfscope}%
\begin{pgfscope}%
\definecolor{textcolor}{rgb}{0.000000,0.000000,0.000000}%
\pgfsetstrokecolor{textcolor}%
\pgfsetfillcolor{textcolor}%
\pgftext[x=1.440772in, y=1.233781in, left, base,rotate=90.000000]{\color{textcolor}\rmfamily\fontsize{10.000000}{12.000000}\selectfont 4}%
\end{pgfscope}%
\begin{pgfscope}%
\definecolor{textcolor}{rgb}{0.000000,0.000000,0.000000}%
\pgfsetstrokecolor{textcolor}%
\pgfsetfillcolor{textcolor}%
\pgftext[x=1.440772in, y=0.730447in, left, base,rotate=90.000000]{\color{textcolor}\rmfamily\fontsize{10.000000}{12.000000}\selectfont 5}%
\end{pgfscope}%
\begin{pgfscope}%
\definecolor{textcolor}{rgb}{0.000000,0.000000,0.000000}%
\pgfsetstrokecolor{textcolor}%
\pgfsetfillcolor{textcolor}%
\pgftext[x=1.260988in,y=2.010000in,,bottom,rotate=90.000000]{\color{textcolor}\rmfamily\fontsize{15.000000}{18.000000}\selectfont Ground truth ISUP grade}%
\end{pgfscope}%
\begin{pgfscope}%
\pgfsetrectcap%
\pgfsetmiterjoin%
\pgfsetlinewidth{0.803000pt}%
\definecolor{currentstroke}{rgb}{0.000000,0.000000,0.000000}%
\pgfsetstrokecolor{currentstroke}%
\pgfsetdash{}{0pt}%
\pgfpathmoveto{\pgfqpoint{1.565000in}{0.500000in}}%
\pgfpathlineto{\pgfqpoint{1.565000in}{3.520000in}}%
\pgfusepath{stroke}%
\end{pgfscope}%
\begin{pgfscope}%
\pgfsetrectcap%
\pgfsetmiterjoin%
\pgfsetlinewidth{0.803000pt}%
\definecolor{currentstroke}{rgb}{0.000000,0.000000,0.000000}%
\pgfsetstrokecolor{currentstroke}%
\pgfsetdash{}{0pt}%
\pgfpathmoveto{\pgfqpoint{4.585000in}{0.500000in}}%
\pgfpathlineto{\pgfqpoint{4.585000in}{3.520000in}}%
\pgfusepath{stroke}%
\end{pgfscope}%
\begin{pgfscope}%
\pgfsetrectcap%
\pgfsetmiterjoin%
\pgfsetlinewidth{0.803000pt}%
\definecolor{currentstroke}{rgb}{0.000000,0.000000,0.000000}%
\pgfsetstrokecolor{currentstroke}%
\pgfsetdash{}{0pt}%
\pgfpathmoveto{\pgfqpoint{1.565000in}{0.500000in}}%
\pgfpathlineto{\pgfqpoint{4.585000in}{0.500000in}}%
\pgfusepath{stroke}%
\end{pgfscope}%
\begin{pgfscope}%
\pgfsetrectcap%
\pgfsetmiterjoin%
\pgfsetlinewidth{0.803000pt}%
\definecolor{currentstroke}{rgb}{0.000000,0.000000,0.000000}%
\pgfsetstrokecolor{currentstroke}%
\pgfsetdash{}{0pt}%
\pgfpathmoveto{\pgfqpoint{1.565000in}{3.520000in}}%
\pgfpathlineto{\pgfqpoint{4.585000in}{3.520000in}}%
\pgfusepath{stroke}%
\end{pgfscope}%
\begin{pgfscope}%
\definecolor{textcolor}{rgb}{1.000000,1.000000,1.000000}%
\pgfsetstrokecolor{textcolor}%
\pgfsetfillcolor{textcolor}%
\pgftext[x=1.816667in,y=3.268333in,,]{\color{textcolor}\rmfamily\fontsize{10.000000}{12.000000}\selectfont 92\%}%
\end{pgfscope}%
\begin{pgfscope}%
\definecolor{textcolor}{rgb}{0.150000,0.150000,0.150000}%
\pgfsetstrokecolor{textcolor}%
\pgfsetfillcolor{textcolor}%
\pgftext[x=2.320000in,y=3.268333in,,]{\color{textcolor}\rmfamily\fontsize{10.000000}{12.000000}\selectfont 7\%}%
\end{pgfscope}%
\begin{pgfscope}%
\definecolor{textcolor}{rgb}{0.150000,0.150000,0.150000}%
\pgfsetstrokecolor{textcolor}%
\pgfsetfillcolor{textcolor}%
\pgftext[x=2.823333in,y=3.268333in,,]{\color{textcolor}\rmfamily\fontsize{10.000000}{12.000000}\selectfont 1\%}%
\end{pgfscope}%
\begin{pgfscope}%
\definecolor{textcolor}{rgb}{0.150000,0.150000,0.150000}%
\pgfsetstrokecolor{textcolor}%
\pgfsetfillcolor{textcolor}%
\pgftext[x=3.326667in,y=3.268333in,,]{\color{textcolor}\rmfamily\fontsize{10.000000}{12.000000}\selectfont 0\%}%
\end{pgfscope}%
\begin{pgfscope}%
\definecolor{textcolor}{rgb}{0.150000,0.150000,0.150000}%
\pgfsetstrokecolor{textcolor}%
\pgfsetfillcolor{textcolor}%
\pgftext[x=3.830000in,y=3.268333in,,]{\color{textcolor}\rmfamily\fontsize{10.000000}{12.000000}\selectfont 0\%}%
\end{pgfscope}%
\begin{pgfscope}%
\definecolor{textcolor}{rgb}{0.150000,0.150000,0.150000}%
\pgfsetstrokecolor{textcolor}%
\pgfsetfillcolor{textcolor}%
\pgftext[x=4.333333in,y=3.268333in,,]{\color{textcolor}\rmfamily\fontsize{10.000000}{12.000000}\selectfont 0\%}%
\end{pgfscope}%
\begin{pgfscope}%
\definecolor{textcolor}{rgb}{0.150000,0.150000,0.150000}%
\pgfsetstrokecolor{textcolor}%
\pgfsetfillcolor{textcolor}%
\pgftext[x=1.816667in,y=2.765000in,,]{\color{textcolor}\rmfamily\fontsize{10.000000}{12.000000}\selectfont 5\%}%
\end{pgfscope}%
\begin{pgfscope}%
\definecolor{textcolor}{rgb}{1.000000,1.000000,1.000000}%
\pgfsetstrokecolor{textcolor}%
\pgfsetfillcolor{textcolor}%
\pgftext[x=2.320000in,y=2.765000in,,]{\color{textcolor}\rmfamily\fontsize{10.000000}{12.000000}\selectfont 80\%}%
\end{pgfscope}%
\begin{pgfscope}%
\definecolor{textcolor}{rgb}{0.150000,0.150000,0.150000}%
\pgfsetstrokecolor{textcolor}%
\pgfsetfillcolor{textcolor}%
\pgftext[x=2.823333in,y=2.765000in,,]{\color{textcolor}\rmfamily\fontsize{10.000000}{12.000000}\selectfont 14\%}%
\end{pgfscope}%
\begin{pgfscope}%
\definecolor{textcolor}{rgb}{0.150000,0.150000,0.150000}%
\pgfsetstrokecolor{textcolor}%
\pgfsetfillcolor{textcolor}%
\pgftext[x=3.326667in,y=2.765000in,,]{\color{textcolor}\rmfamily\fontsize{10.000000}{12.000000}\selectfont 2\%}%
\end{pgfscope}%
\begin{pgfscope}%
\definecolor{textcolor}{rgb}{0.150000,0.150000,0.150000}%
\pgfsetstrokecolor{textcolor}%
\pgfsetfillcolor{textcolor}%
\pgftext[x=3.830000in,y=2.765000in,,]{\color{textcolor}\rmfamily\fontsize{10.000000}{12.000000}\selectfont 0\%}%
\end{pgfscope}%
\begin{pgfscope}%
\definecolor{textcolor}{rgb}{0.150000,0.150000,0.150000}%
\pgfsetstrokecolor{textcolor}%
\pgfsetfillcolor{textcolor}%
\pgftext[x=4.333333in,y=2.765000in,,]{\color{textcolor}\rmfamily\fontsize{10.000000}{12.000000}\selectfont 0\%}%
\end{pgfscope}%
\begin{pgfscope}%
\definecolor{textcolor}{rgb}{0.150000,0.150000,0.150000}%
\pgfsetstrokecolor{textcolor}%
\pgfsetfillcolor{textcolor}%
\pgftext[x=1.816667in,y=2.261667in,,]{\color{textcolor}\rmfamily\fontsize{10.000000}{12.000000}\selectfont 3\%}%
\end{pgfscope}%
\begin{pgfscope}%
\definecolor{textcolor}{rgb}{0.150000,0.150000,0.150000}%
\pgfsetstrokecolor{textcolor}%
\pgfsetfillcolor{textcolor}%
\pgftext[x=2.320000in,y=2.261667in,,]{\color{textcolor}\rmfamily\fontsize{10.000000}{12.000000}\selectfont 24\%}%
\end{pgfscope}%
\begin{pgfscope}%
\definecolor{textcolor}{rgb}{1.000000,1.000000,1.000000}%
\pgfsetstrokecolor{textcolor}%
\pgfsetfillcolor{textcolor}%
\pgftext[x=2.823333in,y=2.261667in,,]{\color{textcolor}\rmfamily\fontsize{10.000000}{12.000000}\selectfont 57\%}%
\end{pgfscope}%
\begin{pgfscope}%
\definecolor{textcolor}{rgb}{0.150000,0.150000,0.150000}%
\pgfsetstrokecolor{textcolor}%
\pgfsetfillcolor{textcolor}%
\pgftext[x=3.326667in,y=2.261667in,,]{\color{textcolor}\rmfamily\fontsize{10.000000}{12.000000}\selectfont 10\%}%
\end{pgfscope}%
\begin{pgfscope}%
\definecolor{textcolor}{rgb}{0.150000,0.150000,0.150000}%
\pgfsetstrokecolor{textcolor}%
\pgfsetfillcolor{textcolor}%
\pgftext[x=3.830000in,y=2.261667in,,]{\color{textcolor}\rmfamily\fontsize{10.000000}{12.000000}\selectfont 5\%}%
\end{pgfscope}%
\begin{pgfscope}%
\definecolor{textcolor}{rgb}{0.150000,0.150000,0.150000}%
\pgfsetstrokecolor{textcolor}%
\pgfsetfillcolor{textcolor}%
\pgftext[x=4.333333in,y=2.261667in,,]{\color{textcolor}\rmfamily\fontsize{10.000000}{12.000000}\selectfont 2\%}%
\end{pgfscope}%
\begin{pgfscope}%
\definecolor{textcolor}{rgb}{0.150000,0.150000,0.150000}%
\pgfsetstrokecolor{textcolor}%
\pgfsetfillcolor{textcolor}%
\pgftext[x=1.816667in,y=1.758333in,,]{\color{textcolor}\rmfamily\fontsize{10.000000}{12.000000}\selectfont 2\%}%
\end{pgfscope}%
\begin{pgfscope}%
\definecolor{textcolor}{rgb}{0.150000,0.150000,0.150000}%
\pgfsetstrokecolor{textcolor}%
\pgfsetfillcolor{textcolor}%
\pgftext[x=2.320000in,y=1.758333in,,]{\color{textcolor}\rmfamily\fontsize{10.000000}{12.000000}\selectfont 0\%}%
\end{pgfscope}%
\begin{pgfscope}%
\definecolor{textcolor}{rgb}{0.150000,0.150000,0.150000}%
\pgfsetstrokecolor{textcolor}%
\pgfsetfillcolor{textcolor}%
\pgftext[x=2.823333in,y=1.758333in,,]{\color{textcolor}\rmfamily\fontsize{10.000000}{12.000000}\selectfont 14\%}%
\end{pgfscope}%
\begin{pgfscope}%
\definecolor{textcolor}{rgb}{1.000000,1.000000,1.000000}%
\pgfsetstrokecolor{textcolor}%
\pgfsetfillcolor{textcolor}%
\pgftext[x=3.326667in,y=1.758333in,,]{\color{textcolor}\rmfamily\fontsize{10.000000}{12.000000}\selectfont 41\%}%
\end{pgfscope}%
\begin{pgfscope}%
\definecolor{textcolor}{rgb}{0.150000,0.150000,0.150000}%
\pgfsetstrokecolor{textcolor}%
\pgfsetfillcolor{textcolor}%
\pgftext[x=3.830000in,y=1.758333in,,]{\color{textcolor}\rmfamily\fontsize{10.000000}{12.000000}\selectfont 35\%}%
\end{pgfscope}%
\begin{pgfscope}%
\definecolor{textcolor}{rgb}{0.150000,0.150000,0.150000}%
\pgfsetstrokecolor{textcolor}%
\pgfsetfillcolor{textcolor}%
\pgftext[x=4.333333in,y=1.758333in,,]{\color{textcolor}\rmfamily\fontsize{10.000000}{12.000000}\selectfont 8\%}%
\end{pgfscope}%
\begin{pgfscope}%
\definecolor{textcolor}{rgb}{0.150000,0.150000,0.150000}%
\pgfsetstrokecolor{textcolor}%
\pgfsetfillcolor{textcolor}%
\pgftext[x=1.816667in,y=1.255000in,,]{\color{textcolor}\rmfamily\fontsize{10.000000}{12.000000}\selectfont 5\%}%
\end{pgfscope}%
\begin{pgfscope}%
\definecolor{textcolor}{rgb}{0.150000,0.150000,0.150000}%
\pgfsetstrokecolor{textcolor}%
\pgfsetfillcolor{textcolor}%
\pgftext[x=2.320000in,y=1.255000in,,]{\color{textcolor}\rmfamily\fontsize{10.000000}{12.000000}\selectfont 5\%}%
\end{pgfscope}%
\begin{pgfscope}%
\definecolor{textcolor}{rgb}{0.150000,0.150000,0.150000}%
\pgfsetstrokecolor{textcolor}%
\pgfsetfillcolor{textcolor}%
\pgftext[x=2.823333in,y=1.255000in,,]{\color{textcolor}\rmfamily\fontsize{10.000000}{12.000000}\selectfont 0\%}%
\end{pgfscope}%
\begin{pgfscope}%
\definecolor{textcolor}{rgb}{0.150000,0.150000,0.150000}%
\pgfsetstrokecolor{textcolor}%
\pgfsetfillcolor{textcolor}%
\pgftext[x=3.326667in,y=1.255000in,,]{\color{textcolor}\rmfamily\fontsize{10.000000}{12.000000}\selectfont 0\%}%
\end{pgfscope}%
\begin{pgfscope}%
\definecolor{textcolor}{rgb}{1.000000,1.000000,1.000000}%
\pgfsetstrokecolor{textcolor}%
\pgfsetfillcolor{textcolor}%
\pgftext[x=3.830000in,y=1.255000in,,]{\color{textcolor}\rmfamily\fontsize{10.000000}{12.000000}\selectfont 84\%}%
\end{pgfscope}%
\begin{pgfscope}%
\definecolor{textcolor}{rgb}{0.150000,0.150000,0.150000}%
\pgfsetstrokecolor{textcolor}%
\pgfsetfillcolor{textcolor}%
\pgftext[x=4.333333in,y=1.255000in,,]{\color{textcolor}\rmfamily\fontsize{10.000000}{12.000000}\selectfont 5\%}%
\end{pgfscope}%
\begin{pgfscope}%
\definecolor{textcolor}{rgb}{0.150000,0.150000,0.150000}%
\pgfsetstrokecolor{textcolor}%
\pgfsetfillcolor{textcolor}%
\pgftext[x=1.816667in,y=0.751667in,,]{\color{textcolor}\rmfamily\fontsize{10.000000}{12.000000}\selectfont 4\%}%
\end{pgfscope}%
\begin{pgfscope}%
\definecolor{textcolor}{rgb}{0.150000,0.150000,0.150000}%
\pgfsetstrokecolor{textcolor}%
\pgfsetfillcolor{textcolor}%
\pgftext[x=2.320000in,y=0.751667in,,]{\color{textcolor}\rmfamily\fontsize{10.000000}{12.000000}\selectfont 0\%}%
\end{pgfscope}%
\begin{pgfscope}%
\definecolor{textcolor}{rgb}{0.150000,0.150000,0.150000}%
\pgfsetstrokecolor{textcolor}%
\pgfsetfillcolor{textcolor}%
\pgftext[x=2.823333in,y=0.751667in,,]{\color{textcolor}\rmfamily\fontsize{10.000000}{12.000000}\selectfont 0\%}%
\end{pgfscope}%
\begin{pgfscope}%
\definecolor{textcolor}{rgb}{0.150000,0.150000,0.150000}%
\pgfsetstrokecolor{textcolor}%
\pgfsetfillcolor{textcolor}%
\pgftext[x=3.326667in,y=0.751667in,,]{\color{textcolor}\rmfamily\fontsize{10.000000}{12.000000}\selectfont 0\%}%
\end{pgfscope}%
\begin{pgfscope}%
\definecolor{textcolor}{rgb}{1.000000,1.000000,1.000000}%
\pgfsetstrokecolor{textcolor}%
\pgfsetfillcolor{textcolor}%
\pgftext[x=3.830000in,y=0.751667in,,]{\color{textcolor}\rmfamily\fontsize{10.000000}{12.000000}\selectfont 62\%}%
\end{pgfscope}%
\begin{pgfscope}%
\definecolor{textcolor}{rgb}{0.150000,0.150000,0.150000}%
\pgfsetstrokecolor{textcolor}%
\pgfsetfillcolor{textcolor}%
\pgftext[x=4.333333in,y=0.751667in,,]{\color{textcolor}\rmfamily\fontsize{10.000000}{12.000000}\selectfont 35\%}%
\end{pgfscope}%
\begin{pgfscope}%
\definecolor{textcolor}{rgb}{0.000000,0.000000,0.000000}%
\pgfsetstrokecolor{textcolor}%
\pgfsetfillcolor{textcolor}%
\pgftext[x=3.075000in,y=3.686667in,,base]{\color{textcolor}\rmfamily\fontsize{18.000000}{21.600000}\selectfont External test set}%
\end{pgfscope}%
\end{pgfpicture}%
\makeatother%
\endgroup%
}
    \end{subfigure}%
    \caption{}
  \end{subfigure}%
  \caption{Confusion matrices for ISUP grading on (a) the internal test set and (b) the external test set. The matrix on the left of each set shows the total amount of predictions per ground truth grade. The matrix on the right of each set displays the percentage predicted grades per ground truth grade.}
  \label{fig:heatmaps}
\end{figure*}

\begin{table*}[b]
  \centering
  \caption{The \acrshort{qwk} and accuracy results for both the internal and external dataset. The star (*) indicates grouping of ISUP grades such as: benign, ISUP 1, ISUP 2-3, and ISUP 4-5.}
  \begin{tabular}{@{}lcccc@{}}
    \toprule
    \textbf{Test set} & \textbf{QWK} & \textbf{QWK*} & \textbf{Accuracy} & \textbf{Accuracy*} \\ \midrule
    Internal          & 0.909        & 0.914         & 73.7\%            & 82.3\%             \\
    External          & 0.882        & 0.875         & 70.3\%            & 79.3\%             \\ \bottomrule
  \end{tabular}
  \label{table:results}
\end{table*}

To train the model on the complete \acrshort{panda} dataset took a total of 2 hours and 18 minutes. Inference took roughly 30 minutes (on one GPU) for both the internal and external sets.

The \acrlong{qwk} achieved by the network on the internal set was 0.909 and the accuracy reached 73.7\%. Grouping the ISUP grades, the network performed a little better and achieved a \acrshort{qwk} of 0.914 and accuracy of 82.3\%.

The external dataset performed with a \acrshort{qwk} of 0.882 and accuracy of 70.3\%. Grouping the ISUP grades showed different results compared to the internal dataset, a slightly worse performance with a \acrshort{qwk} of 0.875 and improved accuracy of 79.3\%.

A summary of these results, for both the internal and external test sets are seen in table \ref{table:results}.


\section{Discussion}
The \textit{simple} model created displayed high performant results on both the internal and external validation sets. As expected, there is a drop in performance between the internal and external sets. This drop was not as large as expected and can be due to several reasons. One theory proposed is that several augmentations done (such as blurring, hue, and saturation) that do not effect the internal test set performance helps the model to regularize to other test sets. This is due to the nature of microscopes, having different colour profiles, and focusing capapilities. Also normalising the images to ImageNet values most likely helped to keep this standard. Though, more research is needed to support this hypothesis.

The model seems to have minor trouble with some outliers (e.g. distinguishing between ISUP 5 and benign tissue), this is most likely due to not enough training data. Ensembling the model across all 5 folds would most likely eliminate this problem.

Interesting to note is that already after 25 minutes of training the model achieved a \acrshort{qwk} of 0.935 on the holdout fold, increasing to 0.953 after an additional two hours. This displays how fast the network is actually learning, mainly due to super-converging with the cyclical learning rate.

\section{Conclusion}
In this paper, we have displayed that training a state-of-the-art pathologist-level model for grading prostate biopsies on the ISUP scale does not require a tremendous amount of resources; it can be done in roughly two hours on two consumer-grade graphics cards. We believe that this facilitates the application of support grading systems in regions where neither expert level pathology resources or computational resources are sufficient. 

Though, this is study used quite a strict definition of insufficient resources. Corners were cut to keep this definition, e.g. no ensembling was made to drastically reduce training and inference time. Such strict definition may prove premature, and simply ensembling all folds may be a worthwhile compromise to make in comparison to the performance increase.

\begin{ack}
I would like to thank my supervisor \textbf{Martin Eklund} for letting me take on this project and welcoming me into the group. I would also like to thank \textbf{Kimmo Kartasalo} and \textbf{Nita Mulliqi} for the immense support and openness to discuss ideas.
\end{ack}

\printbibliography

\end{document}
